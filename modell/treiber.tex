\section{SSH-Treiber}\label{sec:sshDriver}

Im Einführungstext zu diesem Kaptiel wurde bereits auf den grundlegenden Aufbau des Treibers eingegangen. Der SSH-Treiber besteht aus den drei einzelnen Komponenten Parser, Connector und SSH-Verbindung. Der Parser selbst besteht neben dem eigentlichen Parser zudem aus Datenhaltungs-Klassen für die relevanten YARN-Komponenten. Sie sind außerdem so aufgebaut, dass sie für beide hier implementierten Parser bzw. Verbindungen für die Kommandozeilen-Befehle und die REST-API genutzt werden können.

\subsection{Integration im Modell}\label{sec:modelIntegration}

Hadoop besitzt zwei primäre Wege, um die Daten vom \ac{RM} bzw. dem \ac{TLS} ausgeben zu können. Dies ist zum einen die Kommandozeile, mithilfe der die Daten vom \ac{RM} und vom \ac{TLS} kombiniert ausgegeben werden, und die REST-API. Auf beiden Wegen können \uA die Daten zu folgenden Komponenten ausgegeben werden \cite{HadoopYarnTlServer271,HadoopYarnCmds271,HadoopRmApi271,HadoopNmApi271}:

\begin{description}[noitemsep]
    \item[Anwendungen] als nach dem Status gefilterte Liste oder der Report einer Anwendung
    \item[Ausführungen] als Liste aller Ausführungen einer Anwendung oder der Report einer Ausführung
    \item[Container] als Liste aller Container einer Ausführung oder der Report eines Containers
    \item[Nodes] als Liste aller Nodes oder der Report eines Nodes
\end{description}

Zur Integration des Treibers wurden daher entsprechende Interfaces erstellt, über die das Modell auf den eigentlichen Treiber zugreifen kann.

Die vier Interfaces \texttt{IApplicationResult}, \texttt{IAppAttemptResult}, \texttt{IContainerResult} und \texttt{INodeResult} dienen der Übergabe der geparsten Daten der einzelnen Komponenten an die korrespondierenden Komponenten im \sS-Modell. Sie alle implementieren zudem das Interface \texttt{IParsedComponent}, welches wiederum als Basis für die Übergabe der ausgelesenen Daten an \texttt{IYarnReadable.SetStatus()} im Modell dient.

\subsection{Kommandozeilen-Parser}\label{sec:cmdParser}

% Was gibt Hadoop aus
Hadoop besitzt zur Steuerung einige Kommandozeilen-Befehle, mit denen \uA auch die Daten der YARN-Komponenten ausgelesen werden können. Die Daten werden mithilfe der Befehle immer vom \ac{RM} und, sofern gestartet, vom Timeline-Server ermittelt und ausgegeben.

% Wie wurde der Parser aufgebaut

\subsection{REST-API-Parser}\label{sec:restParser}

% Was gibt Hadoop aus
% Wie wurde der Parser aufgebaut

\subsection{Connector}\label{sec:Connector}

% Wie wird die Verbindung abstrahiert

\subsection{SSH-Verbindung}\label{sec:sshConnection}

% Wie funktioniert die Verbindung selbst
