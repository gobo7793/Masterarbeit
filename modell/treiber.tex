\section{SSH-Treiber}\label{sec:sshDriver}

Im Einführungstext zu diesem Kaptiel wurde bereits auf den grundlegenden Aufbau des Treibers eingegangen. Der SSH-Treiber besteht aus den drei einzelnen Komponenten Parser, Connector und SSH-Verbindung, von denen die ersten beiden mithilfe von Interfaces im YARN-Modell eingebunden sind. Dadurch ist es möglich, unterschiedliche Parser bzw. auch Verbindungen für unterschiedliche Komponenten zu nutzen.

Der Parser selbst besteht neben dem eigentlichen Parser zudem aus Datenhaltungs-Klassen für die relevanten YARN-Komponenten. Sie dienen dazu, die geparsten Rohdaten von Hadoop an das \sS-Modell zu übergeben und sind daher ebenfalls mithilfe von entsprechenden Interfaces im Modell eingebunden. Die Implementierungen der Klassen selbst sind außerdem so aufgebaut, dass sie für beide hier implementierten Parser genutzt werden können.

\subsection{Kommandozeilen-Parser}\label{sec:cmdParser}

% Was gibt Hadoop aus
Hadoop besitzt zur Steuerung einige Kommandozeilen-Befehle, mit denen \uA auch die Daten der YARN-Komponenten ausgelesen werden können. Die Daten werden mithilfe der Befehle immer vom \ac{RM} und, sofern gestartet, vom Timeline-Server ermittelt und ausgegeben. Ausgegeben werden können \uA die Daten zu:

\begin{description}[noitemsep]
    \item[Anwendungen] als nach dem Status gefilterte Liste oder der Report einer Anwendung
    \item[Ausführungen] als Liste aller Ausführungen einer Anwendung oder der Report einer Ausführung
    \item[Container] als Liste aller Container einer Ausführung oder der Report eines Containers
    \item[Nodes] als Liste aller Nodes oder der Report eines Nodes
\end{description}

Einige Beispiele für die dafür notwendigen Befehle sowie deren mögliche Ausgaben sind in \autoref{app:hadoopCmds} zu finden.

% Wie wurde der Parser aufgebaut

\subsection{REST-API-Parser}\label{sec:restParser}

% Was gibt Hadoop aus
% Wie wurde der Parser aufgebaut

\subsection{Connector}\label{sec:Connector}

% Wie wird die Verbindung abstrahiert

\subsection{SSH-Verbindung}\label{sec:sshConnection}

% Wie funktioniert die Verbindung selbst
