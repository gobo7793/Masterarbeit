\section{Implementierung des Treibers}
\label{sec:sshDriver}

Im Einführungstext zu diesem Kapitel wurde bereits auf den grundlegenden Aufbau des Treibers eingegangen, der aus den drei einzelnen Komponenten Parser, Connector und der eigentlichen SSH"=Verbindung besteht.
Der Parser selbst besteht neben dem eigentlichen Parser zudem aus Datenhaltungs"=Klassen für die relevanten YARN"=Komponenten.
Sie sind außerdem so aufgebaut, dass sie für beide hier implementierten Parser bzw. Connectoren für die Kommandozeilen"=Befehle und die REST"=API genutzt werden können.

\subsection{Integration im Modell}
\label{sec:modelIntegration}

Hadoop besitzt zwei primäre Wege, um die Daten vom \ac{RM} bzw. dem \ac{TLS} ausgeben zu können.
Dies ist zum einen die Kommandozeile, mithilfe der die Daten vom \ac{RM} und vom \ac{TLS} kombiniert ausgegeben werden, und die REST"=API.
Die benötigten Befehle für die Kommandozeile und deren Ausgaben sind in \autoref{app:hadoopCmds}, die für die REST"=API benötigten URLs und deren Rückgaben in \autoref{app:hadoopRestApi} gelistet.
Auf beiden Wegen können \uA die Daten zu folgenden Komponenten ausgegeben werden \cite{HadoopYarnTlServer271,HadoopYarnCmds271,HadoopRmApi271,HadoopNmApi271}:

\begin{description}
    \item[Anwendungen] als nach dem Status gefilterte Liste oder der Report einer Anwendung
    \item[Ausführungen] als Liste aller Ausführungen einer Anwendung oder der Report einer Ausführung
    \item[Container] als Liste aller Container einer Ausführung oder der Report eines Containers
    \item[Nodes] als Liste aller Nodes oder der Report eines Nodes
\end{description}

Zur Integration des Treibers wurden daher entsprechende Interfaces entwickelt, über die das Modell auf den eigentlichen Treiber zugreifen kann.

Die vier Interfaces \texttt{IApplicationResult}, \texttt{IAppAttemptResult}, \texttt{IContainerResult} und \texttt{INodeResult} dienen der Übergabe der geparsten Daten der einzelnen Komponenten an die korrespondierenden Komponenten im \sS"=Modell.
Sie enthalten jeweils alle relevanten Daten, die von Hadoop über die Kommandozeile oder die REST"=API ausgegeben werden.
Alle vier Interfaces implementieren zudem \texttt{IParsedComponent}, welches wiederum als Basis für die Übergabe der ausgelesenen Daten an \texttt{IYarnReadable.SetStatus()} im Modell dient.

Das Interface \texttt{IHadoopParser} dient als Einbindung des Parsers im Modell mithilfe von \texttt{IYarnReadable.Parser} und enthält für jede der acht relevanten Ausgaben von Hadoop entsprechende Methodendefinitionen.

Beim Interface \texttt{IHadoopConnector}, das im Modell den Connector über die \texttt{Fault"-Connector}"=Eigenschaften von \texttt{YarnApp} und \texttt{YarnNode} einbindet, besitzt ebenfalls für jede der acht Datenrückgaben entsprechende Deklarationen, für Ausführungen und Container dabei jeweils vom \ac{RM} (\ac{NM} für Container) und vom \ac{TLS}.
Auf die Nutzung des \ac{TLS} zum Ermitteln der Daten zu Anwendungen wird verzichtet.
Dies liegt darin begründet, dass bei Nutzung der REST"=API des \ac{RM} neben den vom \ac{TLS} bereitgestellten Daten einige weitere Informationen zu den Anwendungen ausgegeben werden \cite{HadoopRmApi271,HadoopYarnTlServer271}.
Das Connector"=Interface enthält darüber hinaus Deklarationen, um die im Modell implementierten Komponentenfehler im realen Cluster zu steuern und Anwendungen starten zu können.
Architektonisch ist der Treiber zudem so aufgebaut, dass das Modell keine Kontrolle über den vom Parser benötigten Connector besitzt und die SSH"=Verbindung ausschließlich vom Connector gesteuert werden kann.

\subsection{Implementierte Parser}
\label{sec:implementedParsers}

Da die Daten für die relevanten Komponenten auf zwei Arten ermittelt werden können und unterschiedliche Ausgaben erzeugen, wurden auch für beide Arten ein Parser (\texttt{CmdParser} und \texttt{RestParser}) entwickelt.
Da der Parser von außerhalb keinerlei weitere Informationen erhält außer der ID der zu parsenden YARN"=Komponente, ist der Parser selbst dafür verantwortlich, die Daten von einem korrespondierenden Connector zu erhalten.
Daher muss zur Initialisierung eines Parsers zunächst der korrespondierende Connector initialisiert werden.
Da für die Nutzung der REST"=API zum Teil die IDs der übergeordneten YARN"=Komponenten ebenfalls nötig sind, ist der \texttt{RestParser} zudem auch dafür verantwortlich, die entsprechenden IDs zu ermitteln, bei der Nutzung der Kommandozeile reichen aufgrund der Befehlsstruktur die IDs der Komponenten selbst.

Die konkreten Implementierungen der auf \texttt{IParsedComponent} basierenden Übergabe"=Interfaces können ebenfalls als Bestandteil des Parsers angesehen werden.
Sie wurden zudem so implementiert, dass sie für beide entwickelten Parser genutzt werden können.

Der grundlegende Ablauf ist bei jedem Parsing"=Vorgang gleich.
Zunächst werden, sofern benötigt, die benötigten YARN"=Komponenten"=IDs ermittelt und die Rohdaten mithilfe des Connectors von Hadoop abgefragt.
Auch vom Parser wird dabei analog zum Modell das Abrufen der Daten ausschließlich mithilfe des Interfaces \texttt{IHadoopConnector} durchgeführt.
Anschließend findet das eigentliche Parsing der Ausgabe von Hadoop statt, deren Daten direkt in der für die YARN"=Komponente vorgesehene \texttt{IParsed"-Compo"-nent}"=Implementierung gespeichert werden.
Da Hadoop über die Kommandozeile die Daten in keinem standardisierten Format zurückgibt, wurde das Parsing der Rohdaten von Hadoop beim \texttt{CmdParser} in eigenem Code mithilfe von \emph{Regular Expressions} realisiert.
Bei der Nutzung der REST"=API werden die Daten dagegen im JSON"=Format zurückgegeben \cite{HadoopYarnTlServer271,HadoopRmApi271,HadoopNmApi271}, wodurch diese mithilfe des \emph{Json.NET}"=Frameworks\footnote{\url{https://www.newtonsoft.com/json}} deserialisiert und direkt als die entsprechende \texttt{IParsedComponent}"=Implementierung gespeichert werden.
Da \ac{RM} und \ac{TLS} verschiedene Daten einer YARN"=Komponente ausgeben, werden, sofern nötig, \ac{RM} und \ac{TLS} abgefragt und die dabei ermittelten Daten zusammengeführt.

Eine erste Besonderheit bildet zudem das Abrufen und Parsen der Report"=Daten mittels REST"=API.
Da die Listen hierbei als Array der einzelnen Reports zurückgegeben werden \cite{HadoopYarnTlServer271,HadoopRmApi271,HadoopNmApi271}, wird beim Parsen eines Ausführungs- oder Container"=Reports die komplette Liste abgerufen und geparst.
Anschließend wird in dieser Liste basierend auf der ID die benötigte Komponente herausgefiltert.

Die zweite Besonderheit bei der Nutzung der REST"=API liegt darin, dass die Daten zu derzeit ausgeführten Container ausschließlich vom \ac{NM}, auf dem der Container ausgeführt wird, zurückgegeben werden können \cite{HadoopRmApi271,HadoopNmApi271}.
Daher werden zur Ermittlung der Container"=Listen alle Nodes abgefragt und anschließend die benötigten Container gefiltert.

Die geparsten Daten werden abschließend als das für die YARN"=Komponente vorgesehene Interface zurückgegeben, was anschließend im Modell zum Speichern der Daten genutzt werden kann.

\subsection{Implementierte Connectoren}
\label{sec:implementedConnectors}

Für die beiden Parser wurden die beiden korrespondieren Connectoren \texttt{CmdConnector} und \texttt{RestConnector} entwickelt.
Während der Connector für die REST"=API nur über eine SSH"=Verbindung verfügt, besteht beim Connector für die Kommandozeile die Möglichkeit, mehrere einzelne SSH"=Verbindungen zu nutzen.
Dies ist damit begründet, dass zum Steuern der Komponentenfehler, was nur über die Kommandozeile möglich ist, eine eigene SSH"=Verbindung genutzt wird.
Zum Starten von Anwendungen besteht zudem die Möglichkeit, eine beliebige Anzahl an einzelnen SSH"=Verbindungen aufzubauen, damit mehrere Anwendungen parallel gestartet werden können.
Da die Daten der einzelnen YARN"=Komponenten in der Fallstudie bevorzugt mithilfe der REST"=API ermittelt werden, kann die dafür vorgesehene SSH"=Verbindung des \texttt{CmdConnector} deaktiviert werden.

Da über die Kommandozeile die Befehle für die Daten vom \ac{TLS} die gleichen wie für die Daten vom \ac{RM} sind \cite{HadoopYarnTlServer271,HadoopYarnCmds271}, sind beim \texttt{CmdConnector} die \ac{TLS}"=Methoden von geringer Bedeutung und nutzen daher ebenfalls die \ac{RM}"=Methoden.

Der Connector ist beim Abrufen der Daten dafür zuständig, die dafür notwendigen Befehle auszuführen.
Während dies für die Kommandozeilen"=Befehle die entsprechenden Hadoop"=Befehle sind, wird dies zum Abrufen der Daten über die REST"=API mithilfe des Tools \emph{curl} durchgeführt.
Die dabei zurückgegebenen Daten werden vom Connector ohne Verarbeitung zurückgegeben und können dann vom Parser verarbeitet werden.

Beim Steuern der Komponentenfehler wird vom Connector das für die Fallstudie entwickelte Start"=Script verwendet.
Nach dem eigentlichen Start bzw. Aufheben eines Komponentenfehlers wird vom Connector zudem überprüft, ob die Injizierung bzw. Aufhebung erfolgreich war.
Während der Datenabruf sowie die Steuerung der Komponentenfehler synchron stattfindet, findet das Starten der Anwendungen asynchron und mithilfe des Benchmark"=Scriptes statt.
Da eine Ausführung einer YARN"=Anwendung längere Zeit in Anspruch nehmen kann, wird dadurch die Ausführung von \sS nicht behindert und es können mehrere Anwendungen parallel ausgeführt werden.

\subsection{SSH"=Verbindung}
\label{sec:sshConnection}

Die SSH"=Verbindung selbst ist der einzige Bestandteil des Treibers, welches kein entsprechendes Interface benötigt, die SSH"=Verbindung wird ausschließlich vom Connector genutzt.
Realisiert wird die Verbindung mithilfe des Frameworks SSH.NET,\footnote{\url{https://github.com/sshnet/SSH.NET}} weshalb die SSH"=Verbindung im Treiber nur entsprechende Funktionen zum Aufbauen, Nutzen und Beenden der Verbindung enthält.

Um die Verbindung mit dem Cluster"=PC aufzubauen, ist zudem ein dort installierter SSH"=Key nötig.
Ein Kommando auf dem Cluster"=PC kann mithilfe der Treiberkomponente synchron und asynchron ausgeführt werden.
