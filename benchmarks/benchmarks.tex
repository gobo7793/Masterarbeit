\chapter{Implementierung der Benchmarks}
\label{ch:benchmarks}

Neben dem \gls{YARN}"=Modell selbst sind auch die während der Testausführung genutzten \glspl{Anwendung} ein wichtiger Bestandteil des gesamten Testmodells.
Da Hadoop selbst sowie die Plattform Hadoop"=Benchmark bereits einige \glspl{Anwendung} und Benchmarks enthalten, konnten diese auch im Rahmen dieser Fallstudie genutzt werden.
Dazu wurde eine Auswahl an \glspl{Anwendung} in einem Transitionssystem in Form einer Markow"=Kette miteinander verbunden, mit dem die Ausführungsreihenfolge der einzelnen \glspl{Anwendung} basierend auf Wahrscheinlichkeiten bestimmt wird.
Verwaltet werden die implementierten Benchmarks mithilfe des Benchmark"=Controllers.

Da für die durchgeführte Fallstudie \cite{Eberhardinger2018} das für diese Fallstudie entwickelte Testmodell genutzt wurde, enthält es auch das hierfür entwickelte Transitionssystem.
Daher wurden Teile der Beiträge dieses Kapitels dort bereits publiziert.

\section{Übersicht möglicher Anwendungen}
\label{sec:appOverview}

Hadoop-Benchmark enthält bereits die Möglichkeit, unterschiedliche Benchmarks zu starten.
Folgende Benchmarks sind bereits in der Plattform integriert (vgl. \cref{sec:hadoopBenchmark}):

\begin{itemize}
    \item Hadoop"=Mapreduce"=Examples
    \item Intel HiBench
    \item \acrfull{SWIM}
\end{itemize}

Jede Benchmark enthält ein spezifisches Start"=Script, um die jeweiligen Benchmarks in einem Docker"=Container zu starten.
Dieser Container wird, abhängig vom \texttt{HostMode} (vgl. \cref{subsec:hostMode}), auf der Docker"=Machine des \gls{RM} oder direkt auf dem Host gestartet.
Da es in Docker"=Umgebungen \textit{best practice} ist, für jeden Einsatzzweck ein eigenes Image zu erstellen bzw. Container zu starten, werden für jede der drei Benchmark"=Suites eigene Container ausgeführt.
Um mehrere Benchmarks gleichzeitig ausführen zu können, wurden die Startscripte der Benchmarks zudem entsprechend angepasst.

\subsection{Mapreduce"=Examples}
\label{subsec:mapreduceExamples}

Die Hadoop"=Mapreduce"=Examples sind unterschiedliche und meist voneinander unabhängige Anwendungen, die beispielhaft für die meisten Anwendungsfälle in einem produktiv genutzten Cluster sind.
Die Examples sind Teil der Hadoop"=Installation und daher standardmäßig in jedem Hadoop"=Cluster verfügbar.
Einige der Anwendungen der Examples sind:

\begin{itemize}
    \item Generatoren für Text und Binärdaten, \zB \acrlong{rtw}
    \item Analysieren von Daten, \zB \acrlong{wc}
    \item Sortieren von Daten, \zB \acrlong{so}
    \item Ausführen von komplexen Berechnungen, \zB die Ausführung der \emph{Bailey-Borwein-Plouffe-Formel} \cite{Bailey1997} zur Berechnung einzelner Stellen von $\pi$
\end{itemize}

\subsection{Intel HiBench}
\label{subsec:hibench}

Intel HiBench\footnote{\url{https://github.com/intel-hadoop/HiBench/}} ist eine von Intel entwickelte Benchmark"=Suite mit \emph{Workloads} zu verschiedenen Anwendungszwecken mit jeweils unterschiedlichen einzelnen Anwendungen.
Die Suite enthielt anfangs nur wenige Anwendungen \cite{Huang2010}, wurde im Laufe der Zeit jedoch stetig um neue Anwendungen und auch Workloads erweitert.
Das zeigt sich auch darin, dass in Hadoop"=Benchmark die HiBench"=Version \mbox{2.2} integriert ist, die einen noch deutlich geringeren Umfang an Workloads und Anwendungen besitzt, wie \zB die aktuellere Version 7.
Aus diesem Grund wurde vor der Analyse der Anwendungen der HiBench"=Suite das Docker"=Image entsprechend auf Version 7 aktualisiert, um die in der Zwischenzeit hinzugefügten Workloads und Anwendungen der Suite nutzen zu können.
HiBench enthält damit folgende Workloads mit einer unterschiedlichen Anzahl an möglichen Anwendungen:

\begin{itemize}
    \item Micro"=Benchmarks (basierend auf den Examples und den Jobclient"=Tests)
    \item Maschinelles Lernen
    \item SQL/Datenbanken
    \item Websuche
    \item Graphen
    \item Streaming
\end{itemize}

\subsection{\glsentryshort{SWIM}}
\label{subsec:swim}

\gls{SWIM}\footnote{\url{https://github.com/SWIMProjectUCB/SWIM/}} ist eine aus 50 einzelnen Workloads bestehende Benchmark"=Suite.
Das besondere an \gls{SWIM} ist, dass die Suite im Rahmen der Studie \cite{Chen2012} entwickelt wurde, und dadurch anhand mehrerer tausend real ausgeführter \gls{MR}"=Jobs entwickelt wurde.
Die dabei enthaltenen Workloads stellen damit eine größere Vielfalt an ausgeführten Anwendungen und damit einen größeren Testumfang dar als vergleichbare Benchmarks \cite{SwimWikiHome}.

Bei der Ausführung auf dem in dieser Fallstudie verwendeten Cluster wurden jedoch nicht alle Workloads fehlerfrei ausgeführt.
Zudem wird in \cite{InriaTutorial} explizit erwähnt, dass die Ausführung auf einem Cluster auf einem Host sehr zeitintensiv ist, sofern die Workloads überhaupt ausgeführt werden können.
\gls{SWIM} ist außerdem für Benchmarks eines Clusters mit mehreren physischen Nodes ausgelegt, weshalb die Ausführung in dieser Fallstudie extrem viel Zeit benötigten würde.
Daher wurde die Nutzung des \gls{SWIM}"=Benchmarks nicht weiter verfolgt.

\subsection{Jobclient"=Tests}
\label{subsec:jobclient}

Ebenfalls im Installationsumfang von Hadoop enthalten sind die hier aufgrund ihres Dateinamens als Jobclient"=Tests bezeichneten Anwendungen.
Hauptbestandteil dieser Tests sind vor allem weitere, den Mapreduce"=Examples ergänzende, Benchmarks, welche das gesamte Cluster oder einzelne Nodes testen.
Der Fokus der Jobclient"=Tests liegt im Gegensatz zu den Examples nicht auf dem \gls{MR}- bzw. YARN"=Framework, sondern beim HDFS.
Da die Jobclient-Tests kein Teil von Hadoop"=Benchmark sind, wurde zur Ausführung der Jobclient"=Test zunächst ein eigenes Start"=Script analog zur Ausführung der Mapreduce"=Examples erstellt, damit diese ebenfalls im Rahmen der Plattform Hadoop"=Benchmark gestartet werden können.
Die Jobclient"=Tests enthalten \uA folgende Arten an Anwendungen:

\begin{itemize}
    \item HDFS"=Systemtests, \zB \texttt{SilveTest}
    \item Reine Lastgeneratoren, \zB \texttt{NNloadGenerator}
    \item Eingabe/Ausgabe"=Durchsatz"=Tests, \zB \texttt{TestDFSIO}
    \item DummyAnwendungen \acrlong{sl} und \acrlong{fl}
\end{itemize}


\section{Auswahl der verwendeten Anwendungen}\label{sec:appSelection}

Damit die Fallstudie die Realität abbilden kann, wurden von allen verfügbaren Anwendungen einige ausgewählt und in ein Transitionssystem in Form einer Markow"=Kette überführt.
Diese Kette definiert die Ausführungsreihenfolge zwischen den einzelnen Anwendungen.
Eine zufallsbasierte Markow"=Kette wurde aus dem Grund verwendet, dass auch in der Realität Anwendungen nicht immer in der gleichen Reihenfolge ausgeführt werden und daher auch in der Fallstudie eine unterschiedliche Ausführungsreihenfolge der Anwendungen gewährleistet werden soll.
Mithilfe der Festlegung eines bestimmten Seeds für den in der Fallstudie benötigten Pseudo"=Zufallsgenerator besteht bei Bedarf dennoch die Möglichkeit, einen Test mit den gleichen Anwendungen wiederholen zu können.

Einige der in \autoref{sec:appOverview} erwähnten Mapreduce Examples werden häufig als Benchmark verwendet.
Einige Beispiele dafür sind die Anwendungen \acl{so} und \texttt{grep} (ermittelt Anzahl von Regex"=Übereinstimmungen), die bereits im Referenzpapier zum MapReduce"=Algorithmus als Benchmarks verwendet wurden \cite{Dean2008}.
\acl{tsr} ist ebenfalls ein weit verbreiteter Benchmark, der die Hadoop"=Implementierung der standardisierten \emph{Sort Benchmarks}\footnote{\url{https://sortbenchmark.org/}} darstellt \cite{Graves2013}.
Ebenfalls als guter Benchmark dient die Anwendung \acl{wc}, mit der ein großer Datensatz stark verkleinert bzw. zusammengefasst wird und dient daher als gute Repräsentation für Anwendungsarten, bei denen Daten extrahiert werden \cite{Huang2010,Chen2012}.

Da in dieser Fallstudie ein realistisches Abbild der ausgeführten Anwendungen ausgeführt werden soll, ist es nicht sehr hilfreich, die einzelnen Übergangswahrscheinlichkeiten im Transitionssystem anzugleichen oder rein zufällig zu verteilen.
Einen realistischen Einblick, welche Anwendungs- und Datentypen in produktiv genutzten Hadoop"=Clustern genutzt werden, geben \uA \cite{Chen2012} und \cite{HadoopDataTypes}.
Auffällig ist hierbei, dass die meisten Anwendungen in einem Hadoop"=Cluster innerhalb weniger Sekunden oder Minuten abgeschlossen sind und/oder Datensätze im Größenbereich von wenigen Kilobyte bis hin zu wenigen Megabyte verarbeiten.
Zu einem ähnlichen Ergebnis kamen auch \citeauthor{Ren2013} in \cite{Ren2013} und folgerten daher, dass für kleine Jobs evtl. einfachere Frameworks abseits von Hadoop besser geeignet wären.
Die Autoren der Studie in \cite{HadoopDataTypes} bezeichneten Hadoop aufgrund ihrer Ergebnisse als \enquote{potentielle Technologie zum Verarbeiten aller Arten von Daten}, stellten aber eine ähnliche Vermutung an wie \citeauthor{Ren2013}, dass Hadoop primär Daten nutze, die auch mit \enquote{traditionellen Plattformen} verarbeiten werden könnten.

Basierend auf den Ergebnissen der Studien und der in den anderen Publikationen verwendeten Benchmark"=Anwendungen, wurden folgende Anwendungen der Mapreduce"=Examples und Jobclient"=Tests in das Transitionssystem übernommen:

\begin{itemize}
    \item Generieren von Eingabedaten für andere Anwendungen:
    \begin{itemize}
        \item Textdateien: \ac{rtw} und \ac{dfw}
        \item Binärdateien: \ac{rw} und \ac{tg}
    \end{itemize}

    \item Verarbeitung von Eingabedaten:
    \begin{itemize}
        \item Auslesen bzw. Zusammenfassen: \ac{wc} und \ac{dfr}
        \item Transformieren: \ac{so} für Textdaten und \ac{tsr} für Binärdaten
        \item Validierung: \ac{tms} und \ac{tvl} für die jeweiligen Sortier"=Anwendungen
    \end{itemize}

    \item Ausführen von Berechnungen:
    \begin{itemize}
        \item \acl{pi}\acused{pi}: Quasi-Monte-Carlo-Methode zur einfachen Berechnung von $\pi$ 
        \item \ac{pt}: Berechnung von Pentomino-Problemen
    \end{itemize}

    \item Dummy"=Anwendungen: \ac{sl} und \ac{fl}
\end{itemize}

Der Grund für die Berücksichtigung von mehreren gleichen bzw. ähnlichen Anwendungen für einige Kategorien liegt darin, dass die unterschiedlichen Anwendungen eine unterschiedliche Ausführungsdauer bzw. Datenrepräsentation (Text und Binär) repräsentieren.
So stehen die beiden \texttt{TestDFSIO}"=Varianten für eine umfangreichere Datennutzung, während die jeweils anderen Anwendungen einen kleineren Umfang repräsentieren.
Ähnlich verhält es sich bei den beiden Berechnungs"=Anwendungen, bei denen die \acl{pt}"=Anwendung die deutlich umfangreicheren Berechnungen durchführt.
\texttt{TestDFSIO} enthält zudem die Möglichkeit, Daten zu genieren und zu lesen, weshalb diese Anwendung in zwei Kategorien verwendet wurde.
Haupteinsatzzweck der Anwendung liegt vor allem darin, den Datendurchsatz des \ac{HDFS} zu testen.

Eine Besonderheit bilden die beiden Dummy"=Anwendungen.
Beide werden in dieser Fallstudie dafür genutzt, um zu simulieren, wenn auf dem Cluster \zB derzeit nichts ausgeführt wird, oder ein unerwarteter Fehler während der Ausführung auftaucht.
Daher können beide Anwendungen unabhängig von der derzeit ausgeführten Anwendung als nachfolgende Anwendung ausgewählt werden.
Als nachfolgende Anwendungen für die Dummy"=Anwendungen kommen nur Anwendungen in Betracht, welche ihrerseits keine Eingabedaten benötigen. Dies sind:

\begin{itemize}
    \item \acl{dfw}
    \item \acl{rtw}
    \item \acl{tg}
    \item \acl{rw}
    \item \acl{pi}
    \item \acl{pt}
\end{itemize}

\begin{table}
    \resizebox{\linewidth}{!}{\begin{tabular}{l|c|c|c|c|c|c|c|c|c|c|c|c|c|c}
    	             & \textit{dfw} & \textit{rtw} & \textit{tg} & \textit{dfr} & \textit{wc} & \textit{rw} & \textit{so} & \textit{tsr} & \textit{pi} & \textit{pt} & \textit{tms} & \textit{tvl} & \textit{sl} & \textit{fl} \\ \hline
    	\textit{dfw} &     .600     &     .073     &      0      &     .145     &      0      &      0      &      0      &      0       &    .073     &    .073     &      0       &      0       &    .018     &    .018     \\ \hline
    	\textit{rtw} &     .036     &     .600     &      0      &      0       &    .145     &    .036     &    .109     &      0       &    .036     &      0      &      0       &      0       &    .019     &    .019     \\ \hline
    	\textit{tg}  &      0       &     .036     &    .600     &      0       &      0      &      0      &      0      &     .255     &      0      &    .073     &      0       &      0       &    .018     &    .018     \\ \hline
    	\textit{dfr} &      0       &     .073     &      0      &     .600     &      0      &    .036     &      0      &      0       &    .145     &    .109     &      0       &      0       &    .018     &    .019     \\ \hline
    	\textit{wc}  &     .073     &     .109     &      0      &      0       &    .600     &      0      &    .073     &      0       &    .073     &    .036     &      0       &      0       &    .018     &    .018     \\ \hline
    	\textit{rw}  &      0       &     .073     &    .073     &      0       &      0      &    .600     &      0      &      0       &    .109     &    .109     &      0       &      0       &    .018     &    .018     \\ \hline
    	\textit{so}  &      0       &     .073     &    .036     &      0       &    .073     &    .036     &    .600     &      0       &    .073     &      0      &     .073     &      0       &    .018     &    .018     \\ \hline
    	\textit{tsr} &      0       &      0       &      0      &      0       &      0      &      0      &      0      &     .600     &    .109     &    .073     &      0       &     .182     &    .018     &    .018     \\ \hline
    	\textit{pi}  &     .145     &     .109     &      0      &      0       &      0      &      0      &      0      &      0       &    .600     &    .109     &      0       &      0       &    .018     &    .019     \\ \hline
    	\textit{pt}  &     .109     &     .109     &      0      &      0       &      0      &    .073     &      0      &      0       &    .073     &    .600     &      0       &      0       &    .018     &    .018     \\ \hline
    	\textit{tms} &      0       &     .145     &      0      &      0       &      0      &    .073     &      0      &      0       &    .036     &    .109     &     .600     &      0       &    .018     &    .019     \\ \hline
    	\textit{tvl} &     .073     &     .109     &      0      &      0       &      0      &      0      &      0      &      0       &    .109     &    .073     &      0       &     .600     &    .018     &    .018     \\ \hline
    	\textit{sl}  &     .167     &     .167     &    .167     &      0       &      0      &    .167     &      0      &      0       &    .167     &    .167     &      0       &      0       &      0      &      0      \\ \hline
    	\textit{fl}  &     .167     &     .167     &    .167     &      0       &      0      &    .167     &      0      &      0       &    .167     &    .167     &      0       &      0       &      0      &      0
    \end{tabular}}
    \caption
    {Verwendete Markov"=Kette für die Anwendungs"=Übergänge in Tabellenform.}
    \label{tab:transMatrix}
\end{table}

Für die in \autoref{tab:transMatrix} dargestellte Markow"=Kette der Übergänge zwischen den Anwendungen wurde neben den Ergebnissen aus den Studien zudem berücksichtigt, welche Anwendungen bestimmte Eingabedaten benötigen.
Dadurch wird sichergestellt, dass die für einige Anwendungen benötigten Eingabedaten immer vorhanden sind, da diese ebenfalls im Rahmen der Ausführung der Benchmarks generiert werden.
Anwendungen ohne Eingabedaten können dagegen fast jederzeit ausgeführt werden.


\section{Entwicklung des Benchmark"=Controllers}
\label{sec:benchmarkController}

Die im YARN"=Modell (vgl. \cref{sec:yarnModel}) implementierten Benchmarks und das zur Auswahl der Benchmarks entwickelte Transitionssystem bilden zusammen den Benchmark"=Controller.
Der Benchmark"=Controller wurde als eigene Klasse \texttt{BenchmarkController} im Modell implementiert und wird von den Clients genutzt, um die Auswahl der Benchmarks vorzunehmen.
Der Benchmark"=Controller verwaltet die implementierten Benchmarks und stellt diese den Clients zum Starten zur Verfügung.
Damit die Clients unabhängig voneinander sind, wird für jeden Client ein eigener Benchmark"=Controller instanziiert (vgl. \cref{subsec:yarnClient}).

\subsection{Implementierung von Benchmarks und Transitionssystem}
\label{subsec:appImplementation}

Die Benchmarks sind mithilfe der Klasse \texttt{Benchmark} implementiert.
Sie enthält alle zur Ausführung der Benchmarks benötigten Informationen und stellt diese der \texttt{BenchmarkController}"=Klasse bzw. dem Client bereit, um die Anwendung des Benchmarks zu starten.
Da mehrere Clients unabhängig voneinander agieren können müssen, erhält jeder Client ein spezifisches Unterverzeichnis im HDFS, in dem sich die Ein- und Ausgabeverzeichnisse für die von ihm gestarteten Anwendungen befinden.
Das muss auch bei der Definition der Startbefehle der Anwendungen berücksichtigt werden, weshalb hierfür entsprechende Platzhalter ersetzt werden müssen, wenn mithilfe der Methode \texttt{GetStartCmd()} der Start"=Befehl des Benchmarks zurückgegeben wird:

\begin{lstlisting}[label=lst:benchmarkClass,style=cs,
caption={[Wesentliche Methoden der Klasse Benchmark]
    Wesentliche Methoden der Klasse \texttt{Benchmark}}]
public class Benchmark
{
  public Benchmark(int id, string name, string startCmd,
     string outputDir, string inputDir)
  {
    _StartCmd = startCmd;
    _InDir = inputDir;
    HasInputDir = true;
  }
  
  public string GetStartCmd(string clientDir = "")
  {
    var result = _StartCmd
       .Replace(OutDirHolder, GetOutputDir(clientDir))
       .Replace(InDirHolder, GetInputDir(clientDir));
    if(result.Contains(BaseDirHolder))
      result = ReplaceClientDir(result, clientDir);
    return result;
  }
}
\end{lstlisting}

Die im Modell enthaltenen Benchmarks sind als Array in \texttt{BenchmarkController} gespeichert.
Hierbei werden mithilfe der Holder"=Variablen, die in \texttt{GetStartCmd()} ersetzt werden, die Startbefehle der Anwendungen definiert:

\begin{lstlisting}[label=lst:benchmarkDefinition,style=cs,
caption={[Definition der verfügbaren Benchmarks im BenchmarkController]
    Definition der verfügbaren Benchmarks im \texttt{BenchmarkController} (gekürzt)}]
public Benchmark[] Benchmarks => new[]
{
  new Benchmark(04, "wordcount",
     $"example wordcount {InDirHolder} {OutDirHolder}",
     $"{BaseDirHolder}/wcout"),
};
\end{lstlisting}

Der hierbei definierte und durch \texttt{GetStartCmd()} zurückgegebene vollständige Startbefehl wird beim Starten der Anwendungen vom Connector als Befehlsparameter des Benchmark"=Script genutzt (vgl. \cref{subsubsec:implCmdConnector}).
Damit kann durch das Benchmark"=Script die zu startende Anwendung identifiziert und das jeweilige Start"=Script ausgeführt werden (vgl. \cref{subsec:scripts}).

Das Transitionssystem selbst wurde im \texttt{BenchmarkController} als zweidimensionaler Array implementiert, auf das mithilfe von \texttt{Benchmark.ID} zugegriffen werden kann.
Entsprechend wichtig ist hierbei die Reihenfolge der jeweiligen Werte innerhalb des Arrays, welche immer gleich sein muss mit den jeweiligen IDs:

\begin{lstlisting}[label=lst:transitionSystemImpl,style=cs,
caption={[Implementierung des Transitionssystems im BenchmarkController]
    Implementierung des Transitionssystems im \texttt{BenchmarkController} (gekürzt)}]
public double[][] BenchTransitions => new[]
{
  /* f§§rom / to ->    00    01   02   ...*/
  /* f§§rom / to ->   dfw   rtw   tg   ...*/
  new[] /* 00 */ { .600, .073,  000, ... ,
  new[] /* 01 */ { .036, .600,  000, ... ,
  new[] /* 02 */ {  000, .036, .600, ... ,
};
\end{lstlisting}

\subsection{Auswahl der nachfolgenden Benchmarks}
\label{subsec:selectionNextBenchmark}

Zur Auswahl der Nachfolgenden Anwendung dient die Methode \texttt{ChangeBenchmark()} des \texttt{BenchmarkController}s.
Hier wird mithilfe des Transitionssystems und unabhängig von anderen Clients bestimmt, welcher Benchmark ausgeführt werden soll und dieser in \texttt{CurrentBenchmark} gespeichert:

\begin{lstlisting}[label=lst:benchmarkChanging,style=cs,
caption={[Auswahl des nachfolgenden Benchmarks]
    Auswahl des nachfolgenden Benchmarks (gekürzt).
    Dies stellt einen Ausschnitt der Methode \texttt{ChangeBenchmark()} dar, welche vom Client zur Bestimmung des nachfolgenden Benchmarks aufgerufen wird (vgl. \cref{subsec:yarnClient}).}]
// g§§et probabilities f§§rom current benchmark
var transitions = BenchTransitions[CurrentBenchmark.Id];

// calculate next benchmark
var ranNumber = RandomGen.NextDouble();
var cumulative = 0D;
for(int i = 0; i < transitions.Length; i++)
{
  cumulative += transitions[i];
  if(ranNumber >= cumulative)
    continue;
  
  // prevent saving current benchmark a§§s previous
  if(CurrentBenchmark == _BenchmarksInstance[i])
    break;
  
  // save benchmarks
  PreviousBenchmark = CurrentBenchmark;
  CurrentBenchmark = Benchmarks[i];
  return true;
}
\end{lstlisting}

Bevor auf die Daten des implementierten Transitionssystems zugegriffen wird, wird außerdem zunächst geprüft, ob die Markow"=Kette alle möglichen Übergänge für den aktuellen Benchmark enthält.
Ist das nicht der Fall, wird eine \texttt{InvalidOperationException} ausgelöst.
Wenn die Auswahl des Benchmarks dagegen erfolgreich war, wird an den aufrufenden Client \texttt{true} zurückgegeben und der ausgewählte Benchmark kann gestartet werden (vgl. \cref{subsec:yarnClient}).

Der zur Auswahl des nachfolgenden Benchmarks benötigte Zufallsgenerator \texttt{Random""Gen} wird bei der Initialisierung des Benchmark"=Controllers basierend auf dem in \cref{subsec:testcaseGeneration} spezifizierten Basisseed initialisiert.
Damit die Benchmark"=Controller aller Clients nicht die gleichen Benchmarks auswählen, wird zum Basisseed die numerische ID des zum Benchmark"=Controler zugehörigen Clients addiert und der Zufallsgenerator mit diesem Wert initialisiert.
Als initiale Anwendung wird immer der \acrlong{sl}"=Benchmark genutzt, sodass als erste Anwendung immer eine Anwendung ohne benötigte Eingabedaten gestartet wird (vgl. \cref{subsec:markovChain}).

\subsection{Vorabgenerierung von Eingabedaten}
\label{subsec:precreateInputData}

Neben der Generierung der für einige Anwendungen benötigten Eingabedaten während der Testausführung gibt es auch die Möglichkeit, die Eingabedaten vorab zu generieren und anschließend diese zu nutzen (vgl. \cref{sec:clusterSetup,subsec:testcaseGeneration,subsec:simulationModelInit}).
Um die vorab generierten Daten in einem Test zu nutzen, muss die entsprechende Eigenschaft \texttt{ModelSettings.IsPrecreateBenchInputs} auf \texttt{true} gesetzt werden.
Dadurch werden auch direkt die Eingabedaten generiert.

Die Vorabgenerierung der Daten startet folgende Anwendungen, die Eingabedaten für andere Anwendungen generieren und speichert diese in einem nur hierfür genutzten Verzeichnis im HDFS:

\begin{itemize}
    \item \acrlong{dfw}
    \item \acrlong{rtw}
    \item \acrlong{tg}
    \item \acrlong{so}
    \item \acrlong{tsr}
\end{itemize}

Gestartet werden kann die Vorabgenerierung mithilfe des Benchmark"=Controllers.
Hierbei werden die Anwendungen, sofern möglich, gleichzeitig ausgeführt und anschließend gewartet, bis alle fünf Anwendungen beendet sind.
Hierbei wird standardmäßig die Generierung der Eingabedaten einer Anwendung übersprungen, wenn das entsprechende Ausgabeverzeichnis der Anwendung bereits existiert.

Es ist jedoch auch möglich, vorhandene Verzeichnisse zu löschen, um somit die Daten vollständig neu zu generieren.
Hierbei wird zudem ein HDFS"=Filecheck ausgeführt, um fehlerhafte Daten zu finden und zu löschen.
Fehlerhafte Daten können im HDFS \zB dadurch entstehen, dass alle Nodes defekt sind, auf denen ein Block repliziert wurde, sich die Datei jedoch noch im Index des HDFS befindet (vgl. \cref{sec:hadoop}.
Die vollständige Neugenerierung der Eingabedaten kann mithilfe der Eigenschaft \texttt{ModelSettings.IsPrecreateBenchInputsRecreate} gesteuert werden (vgl. \cref{subsec:simulationModelInit}).

Um beim Starten der Anwendungen die vorab generierten Eingabedaten zu nutzen, wird beim Starten der Anwendungen das Verzeichnis der vorab generierten Daten als entsprechendes Client"=Verzeichnis genutzt (vgl. \cref{subsec:yarnClient,subsec:appImplementation}).
