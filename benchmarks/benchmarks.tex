\chapter{Implementierung der Benchmarks}
\label{ch:benchmarks}

Neben dem \gls{YARN}"=Modell selbst sind auch die während der Testausführung genutzten \glspl{Anwendung} ein wichtiger Bestandteil des gesamten Testmodells.
Da Hadoop selbst sowie die Plattform Hadoop"=Benchmark bereits einige \glspl{Anwendung} und Benchmarks enthalten, konnten diese auch im Rahmen dieser Fallstudie genutzt werden.
Dazu wurde eine Auswahl an \glspl{Anwendung} in einem Transitionssystem in Form einer Markow"=Kette miteinander verbunden, mit dem die Ausführungsreihenfolge der einzelnen \glspl{Anwendung} basierend auf Wahrscheinlichkeiten bestimmt wird.
Verwaltet werden die implementierten Benchmarks mithilfe des Benchmark"=Controllers.

Da für die durchgeführte Fallstudie \cite{Eberhardinger2018} das für diese Fallstudie entwickelte Testmodell genutzt wurde, enthält es auch das hierfür entwickelte Transitionssystem.
Daher wurden Teile der Beiträge dieses Kapitels dort bereits publiziert.

\section{Übersicht möglicher Anwendungen}\label{sec:appOverview}

Hadoop-Benchmark enthält bereits die Möglichkeit, unterschiedliche Benchmarks zu starten. Wie in \autoref{sec:hadoopBenchmark} erwähnt, sind folgende Benchmarks in der Plattform integriert:

\begin{itemize}
    \item Hadoop Mapreduce Examples
    \item Intel HiBench
    \item \ac{SWIM}
\end{itemize}

Jeder Benchmark enthält zum Starten ein jeweiliges Start"=Script, mit dem ein neuer Docker"=Container auf der Controller"=VM gestartet wird, mit dem der Benchmark auf dem Cluster gestartet wird. Dass dafür ein eigener Docker-Container genutzt wird liegt daran, dass es in Docker"=Umgebungen \emph{best practice} ist, einen Docker"=Container für nur einen Einsatzzweck zu erstellen und zu nutzen. Die Hauptgründe dafür sind, dass dadurch die Skalierbarkeit erhöht und die Wiederverwendbarkeit gesteigert werden \cite{DockerBestPractice}. Daher wurden auch die Startscripte für die Benchmarks so angepasst, dass die jeweiligen Benchmarks mehrfach gestartet werden können.

Die \textbf{Hadoop Mapreduce Examples} sind unterschiedliche und meist voneinander unabhängige Anwendungen, die beispielhaft für die meisten Anwendungsfällen in einem produktiv genutzten Cluster sind. Die Examples sind Teil von Hadoop und daher bei jeder Hadoop"=Installation enthalten. Einige der Anwendungen der Examples sind:

\begin{itemize}
    \item Generatoren für Text und Binärdaten, \zB \acl{rtw}
    \item Analysieren von Daten, \zB \acl{wc}
    \item Sortieren von Daten, \zB \acl{sort}
    \item Ausführen von komplexen Berechnungen, \zB \emph{Bailey-Borwein-Plouffe-Formel} zur Berechnung einzelner Stellen von $\pi$
\end{itemize}

\textbf{Intel HiBench} ist eine Benchmark-Suite mit \emph{Workloads} zu verschiedenen Anwendungszwecken mit jeweils unterschiedlichen einzelnen Anwendungen. Da in Hadoop"=Benchmark noch die HiBench"=Version \mbox{2.2} verwendet wird, wurde der Docker"=Container von HiBench zunächst auf Version 7 aktualisiert, die einige neuen Workloads und Anwendungen enthält. HiBench enthält damit folgende Workloads mit einer unterschiedlichen Anzahl an möglichen Anwendungen:

\begin{itemize}
    \item Micro"=Benchmarks (basieret auf den Mapreduce"=Examples und den Jobclient"=Tests)
    \item Maschinelles Lernen
    \item SQL/Datenbanken
    \item Websuche
    \item Graphen
    \item Streaming
\end{itemize}

\textbf{\ac{SWIM}} ist eine Benchmark-Suite, die aus 50 verschiedenen Workloads besteht. Das besondere dabei ist, dass die dabei verwendeten Mapreduce"=Jobs anhand mehrerer tausend Jobs erstellt wurden und im Vergleich zu anderen Benchmarks eine größere Vielfalt an Anwendungen und somit ein größerer Testumfang gewährleistet wird \cite{SwimWikiHome}. Bei der Ausführung auf dem in dieser Arbeit verwendete Cluster wurden jedoch nicht alle Workloads fehlerfrei ausgeführt. Zudem wird in \cite{InriaTutorial} explizit erwähnt, dass es bei der Ausführung auf einem Cluster auf einem einzelnen PC bzw. Laptop Probleme geben kann. Zudem ist SWIM für Benchmarks eines Clusters mit mehreren physischen Nodes ausgelegt ist und daher würde die Ausführung in dieser Fallstudie extrem viel Zeit benötigten. Daher wurde die Nutzung des SWIM-Benchmarks nicht weiter verfolgt.

Ebenfalls im Installationsumfang von Hadoop enthalten sind die sog. \textbf{Jobclient"=Tests}. Hauptbestandteil dieser Tests sind vor allem weitere Benchmarks, welche das gesamte Cluster oder einzelne Nodes testen, wobei die meisten Anwendungen das \ac{HDFS} betreffen. Da die Jobclient-Tests kein Teil von Hadoop"=Benchmark sind, wurde zur Ausführung der Jobclient"=Test zunächst ein eigenes Start"=Script analog zur Ausführung der Mapreduce"=Examples erstellt, damit hierfür ebenfalls ein eigener Docker"=Container gestartet wird. Die Jobclient"=Tests enthalten \uA folgende Arten an Anwendungen:

\begin{itemize}
    \item \ac{HDFS}"=Systemtests, \zB \texttt{SilveTest}
    \item Reine Lastgeneratoren, \zB \texttt{NNloadGenerator}
    \item Eingabe/Ausgabe"=Durchsatz"=Tests, \zB \texttt{TestDFSIO}
    \item Dummy"=Anwendungen \texttt{sleep} (blockiert Ressourcen, führt aber nichts aus) und \texttt{fail} (Anwendung schlägt immer fehl)
\end{itemize}

\section{Auswahl der verwendeten Anwendungen}
\label{sec:appSelection}

Damit die Fallstudie die Realität abbilden kann, wurden von allen verfügbaren Anwendungen einige ausgewählt und in ein Transitionssystem in Form einer Markow"=Kette überführt.
Diese Kette definiert die Ausführungsreihenfolge zwischen den einzelnen Anwendungen.
Eine zufallsbasierte Markow"=Kette wurde aus dem Grund verwendet, dass auch in der Realität Anwendungen nicht immer in der gleichen Reihenfolge ausgeführt werden und daher auch in der Fallstudie eine unterschiedliche Ausführungsreihenfolge der Anwendungen gewährleistet werden soll.
Mithilfe der Festlegung eines bestimmten Seeds für den in der Fallstudie benötigten Pseudo"=Zufallsgenerator besteht bei Bedarf dennoch die Möglichkeit, einen Test mit den gleichen Anwendungen wiederholen zu können.

Einige der in \cref{sec:appOverview} erwähnten Mapreduce Examples werden häufig als Benchmark verwendet.
Einige Beispiele dafür sind die Anwendungen \acl{so} und \texttt{grep}, die bereits im Referenzpapier des \ac{MR}"=Frameworks zum Testen genutzt wurden \cite{Dean2004}.
Zum Testen des \ac{HDFS} dient in \cite{Shvachko2010} der DFSIO"=Benchmark, um den Durchsatz beim Lesen und Schreiben einer großen Datenmenge auf dem \ac{HDFS} zu messen.
\acl{tsr} ist ebenfalls ein weit verbreiteter Benchmark, der die Hadoop"=Implementierung der standardisierten \emph{Sort Benchmarks}\footnote{\url{https://sortbenchmark.org/}} darstellt \cite{Graves2013}.
Ebenfalls als guter Benchmark dient die Anwendung \acl{wc}, mit der ein großer Datensatz stark verkleinert bzw. zusammengefasst wird und dient daher als gute Repräsentation für Anwendungsarten, bei denen Daten extrahiert werden \cite{Huang2010,Chen2012}.

Da in dieser Fallstudie ein realistisches Abbild der ausgeführten Anwendungen ausgeführt werden soll, ist es nicht sehr hilfreich, die einzelnen Übergangswahrscheinlichkeiten im Transitionssystem anzugleichen oder rein zufällig zu verteilen.
Einen realistischen Einblick, welche Anwendungs- und Datentypen in produktiv genutzten Hadoop"=Clustern genutzt werden, geben \uA \cite{Chen2012} und \cite{HadoopDataTypes}.
Auffällig ist hierbei, dass die meisten Anwendungen in einem Hadoop"=Cluster innerhalb weniger Sekunden oder Minuten abgeschlossen sind und/oder Datensätze im Größenbereich von wenigen Kilobyte bis hin zu wenigen Megabyte verarbeiten.
Zu einem ähnlichen Ergebnis kamen auch \citeauthor{Ren2013} in \cite{Ren2013} und folgerten daher, dass für kleine Jobs evtl. einfachere Frameworks abseits von Hadoop besser geeignet wären.
Die Autoren der Studie in \cite{HadoopDataTypes} bezeichneten Hadoop aufgrund ihrer Ergebnisse als \enquote{potentielle Technologie zum Verarbeiten aller Arten von Daten}, stellten aber eine ähnliche Vermutung an wie \citeauthor{Ren2013}, dass Hadoop primär Daten nutze, die auch mit \enquote{traditionellen Plattformen} verarbeiten werden könnten.

Basierend auf den Ergebnissen der Studien in \cite{Huang2010,Chen2012,HadoopDataTypes,Ren2013} und der in den Publikationen \cite{Shvachko2010,Dean2004,Graves2013} verwendeten Benchmark"=Anwendungen, wurden folgende Anwendungen der Mapreduce Examples und Jobclient"=Tests in das Transitionssystem übernommen:

\begin{itemize}
    \item Generieren von Eingabedaten für andere Anwendungen:
    \begin{itemize}
        \item Textdateien:
        \begin{itemize}
            \item \ac{rtw}: Generierung von zufälligen Zeichenfolgen
            \item \ac{dfw}: Schreiben einer großen Datenmenge auf dem \ac{HDFS}
        \end{itemize}
        \item Binärdateien:
        \begin{itemize}
            \item \ac{rw}: Generierung von zufälligen Binärdaten
            \item \ac{tg}: Generierung der Eingabedaten für den \acl{tsr}"=Benchmark
       \end{itemize}
    \end{itemize}

    \item Verarbeitung von Eingabedaten:
    \begin{itemize}
        \item Auslesen bzw. Zusammenfassen:
        \begin{itemize}
            \item \ac{wc}: Auslesen einer Textdatei und Ermitteln der Anzahl der darin enthaltenen Wörter
            \item \ac{dfr}: Auslesen einer großen Datenmenge auf dem \ac{HDFS}
        \end{itemize}
        \item Transformieren:
        \begin{itemize}
            \item \ac{so}: Sortieren von Daten, wird in dieser Fallstudie zum Sortieren von Textdaten genutzt
            \item \ac{tsr}: Sortieren von großen Binärdatenmengen
        \end{itemize}
        \item Validierung der Transformationen:
        \begin{itemize}
            \item \ac{tms}: Validierung der von \acl{so} transformierten Daten
            \item \ac{tvl}: Validierung der vom \acl{tsr} sortierten Binärdaten
        \end{itemize}
    \end{itemize}

    \item Ausführen von Berechnungen:\todo{evtl. noch literatur dazu finden}
    \begin{itemize}
        \item \acl{pi}\acused{pi}: Ausführung der Quasi"=Monte"=Carlo"=Methode zur einfachen Berechnung von $\pi$
        \item \ac{pt}: Berechnung von Pentomino"=Problemen
    \end{itemize}

    \item Dummy"=Anwendungen:
    \begin{itemize}
        \item  \ac{sl}: Blockieren von Ressourcen
        \item  \ac{fl}: Fehlschlagen einer Anwendung
    \end{itemize}
\end{itemize}

Der Grund für die Berücksichtigung von mehreren gleichen bzw. ähnlichen Anwendungen für einige Kategorien liegt darin, dass die unterschiedlichen Anwendungen eine unterschiedliche Ausführungsdauer bzw. Datenrepräsentation (Text und Binär) repräsentieren.
So stehen die beiden \texttt{TestDFSIO}"=Varianten für eine umfangreichere Datennutzung, während die jeweils anderen Anwendungen einen kleineren Umfang repräsentieren.
Ähnlich verhält es sich bei den beiden Berechnungs"=Anwendungen, bei denen die \acl{pt}"=Anwendung die deutlich umfangreicheren Berechnungen durchführt.
\texttt{TestDFSIO} enthält zudem die Möglichkeit, Daten zu generieren und zu lesen, weshalb dieser Benchmark in zwei Kategorien als Anwendung genutzt wird.

Eine Besonderheit bilden die beiden Dummy"=Anwendungen.
Beide werden in dieser Fallstudie dafür genutzt, um zu simulieren, wenn auf dem Cluster \zB derzeit nichts ausgeführt wird, oder ein Fehler während der Ausführung einer Anwendung auftritt.
Daher können beide Anwendungen unabhängig von der derzeit ausgeführten Anwendung als nachfolgende Anwendung ausgewählt werden.
Als nachfolgende Anwendungen für die Dummy"=Anwendungen wurden nur Anwendungen definiert, die ihrerseits keine Eingabedaten benötigten bzw. diese für andere Anwendungen generieren:

\begin{itemize}
    \item \acl{dfw}
    \item \acl{rtw}
    \item \acl{tg}
    \item \acl{rw}
    \item \acl{pi}
    \item \acl{pt}
\end{itemize}

\begin{table}
    \resizebox{\linewidth}{!}{\begin{tabu}{l|[1.5pt]c|c|c|c|c|c|c|c|c|c|c|c|c|c}
    	                   & \textit{\acs{dfw}} & \textit{\acs{rtw}} & \textit{\acs{tg}} & \textit{\acs{dfr}} & \textit{\acs{wc}} & \textit{\acs{rw}} & \textit{\acs{so}} & \textit{\acs{tsr}} & \textit{\acs{pi}} & \textit{\acs{pt}} & \textit{\acs{tms}} & \textit{\acs{tvl}} & \textit{\acs{sl}} & \textit{\acs{fl}} \\ \tabucline[1.5pt]{-}
    	\textit{\acs{dfw}} &        .600        &        .073        &         0         &        .145        &         0         &         0         &         0         &         0          &       .073        &       .073        &         0          &         0          &       .018        &       .018        \\ \hline
    	\textit{\acs{rtw}} &        .036        &        .600        &         0         &         0          &       .145        &       .036        &       .109        &         0          &       .036        &         0         &         0          &         0          &       .019        &       .019        \\ \hline
    	\textit{\acs{tg}}  &         0          &        .036        &       .600        &         0          &         0         &         0         &         0         &        .255        &         0         &       .073        &         0          &         0          &       .018        &       .018        \\ \hline
    	\textit{\acs{dfr}} &         0          &        .073        &         0         &        .600        &         0         &       .036        &         0         &         0          &       .145        &       .109        &         0          &         0          &       .018        &       .019        \\ \hline
    	\textit{\acs{wc}}  &        .073        &        .109        &         0         &         0          &       .600        &         0         &       .073        &         0          &       .073        &       .036        &         0          &         0          &       .018        &       .018        \\ \hline
    	\textit{\acs{rw}}  &         0          &        .073        &       .073        &         0          &         0         &       .600        &         0         &         0          &       .109        &       .109        &         0          &         0          &       .018        &       .018        \\ \hline
    	\textit{\acs{so}}  &         0          &        .073        &       .036        &         0          &       .073        &       .036        &       .600        &         0          &       .073        &         0         &        .073        &         0          &       .018        &       .018        \\ \hline
    	\textit{\acs{tsr}} &         0          &         0          &         0         &         0          &         0         &         0         &         0         &        .600        &       .109        &       .073        &         0          &        .182        &       .018        &       .018        \\ \hline
    	\textit{\acs{pi}}  &        .145        &        .109        &         0         &         0          &         0         &         0         &         0         &         0          &       .600        &       .109        &         0          &         0          &       .018        &       .019        \\ \hline
    	\textit{\acs{pt}}  &        .109        &        .109        &         0         &         0          &         0         &       .073        &         0         &         0          &       .073        &       .600        &         0          &         0          &       .018        &       .018        \\ \hline
    	\textit{\acs{tms}} &         0          &        .145        &         0         &         0          &         0         &       .073        &         0         &         0          &       .036        &       .109        &        .600        &         0          &       .018        &       .019        \\ \hline
    	\textit{\acs{tvl}} &        .073        &        .109        &         0         &         0          &         0         &         0         &         0         &         0          &       .109        &       .073        &         0          &        .600        &       .018        &       .018        \\ \hline
    	\textit{\acs{sl}}  &        .167        &        .167        &       .167        &         0          &         0         &       .167        &         0         &         0          &       .167        &       .167        &         0          &         0          &         0         &         0         \\ \hline
    	\textit{\acs{fl}}  &        .167        &        .167        &       .167        &         0          &         0         &       .167        &         0         &         0          &       .167        &       .167        &         0          &         0          &         0         &         0
    \end{tabu}}
    \caption
    {Verwendete Markov"=Kette für die Anwendungs"=Übergänge in Tabellenform.}
    \label{tab:transMatrix}
\end{table}

Für die in \cref{tab:transMatrix} dargestellte Markow"=Kette der Übergänge zwischen den Anwendungen wurde neben den Ergebnissen aus den Studien zudem berücksichtigt, welche Anwendungen bestimmte Eingabedaten benötigen.
Dadurch wird sichergestellt, dass die für einige Anwendungen benötigten Eingabedaten immer vorhanden sind, da diese ebenfalls im Rahmen der Ausführung der Benchmarks generiert werden.
Anwendungen ohne Eingabedaten können dagegen fast jederzeit ausgeführt werden.


\section{Implementierung der Anwendungen im Modell}
\label{sec:appImplementation}

Die Verwaltung der auszuführenden Benchmarks wurde komplett vom restlichen YARN"=Modell getrennt.
Verbunden sind beide durch die Eigenschaft \texttt{Client.BenchController}, das den vom Client verwendeten \texttt{BenchmarkController} enthält, der zur Verwaltung der auszuführenden Anwendung dient.
Der Controller besteht aus zwei wesentlichen Teilen, einem statischen und einem dynamischen.

Der \textbf{statische Teil} des Controllers definiert die möglichen Anwendungen sowie das im Abschnitt zuvor definierte und in \cref{tab:transMatrix} dargestellte Transitionssystem.
Die einzelnen Anwendungen werden mithilfe der Klasse \texttt{Benchmark} repräsentiert, in der die benötigten Informationen wie \zB der Befehl zum Starten der Anwendung definiert werden.
Da mehrere Clients unabhängig voneinander agieren können müssen, erhält jeder Client zudem ein eigenes Unterverzeichnis im \ac{HDFS}, in dem sich die Ein- und Ausgabeverzeichnisse für die von ihm gestarteten Anwendungen befinden.
Das muss auch bei der Definition der Startbefehle der Anwendungen berücksichtigt werden, weshalb in \cref{lst:benchmarkDefinition} entsprechende Platzhalter vorhanden sind.
Aus diesem Grund muss vor dem Start der Anwendung mithilfe der Methode \texttt{GetStartCmd()} der Startbefehl generiert werden, indem der zu startende Client das in \texttt{Client.ClientDir} gespeicherte Client"=Basisverzeichnis übergibt.
Da einige Anwendungen zudem voraussetzen, dass das genutzte Ausgabe"=Verzeichnis noch nicht im \ac{HDFS} existiert, muss das Verzeichnis vor dem Anwendungsstart gelöscht werden.

Jede Anwendung erhält zudem eine eigene ID, die mit ihrem Index im Array \texttt{BenchmarkController.Benchmarks} übereinstimmt.
Diese wird bei der in \cref{lst:benchmarkChanging} dargestellte Auswahl der nachfolgenden Anwendung benötigt, um innerhalb des gesamten Transitionssystems in \texttt{BenchmarkController.BenchTransitions} die Wahrscheinlichkeiten für die Wechsel von der derzeitigen Anwendung zu anderen Anwendungen auszuwählen.

\begin{lstlisting}[label=lst:benchmarkDefinition,style=cs,
caption={[Definition und Start einer Anwendung]
    Definition und Start einer Anwendung (gekürztes Beispiel).
    Die Generierung des komplettes Startbefehls mit Nutzung des Benchmark"=Scriptes führt der vom Client verwendete Connector durch, weshalb hier nur definiert werden muss, dass das Example"=Programm \acl{wc} gestartet wird.}]
public class Benchmark
{
  public const string BaseDirHolder = "$DIR";
  public const string OutDirHolder = "$OUT";
  public const string InDirHolder = "$IN";
  
  public Benchmark(int id, string name, string startCmd, string outputDir, string inputDir)
  {
    _StartCmd = startCmd;
    _InDir = inputDir;
    HasInputDir = true;
  }
  
  public string GetStartCmd(string clientDir = "")
  {
    var result = _StartCmd.Replace(OutDirHolder, GetOutputDir(clientDir)).Replace(InDirHolder, GetInputDir(clientDir));
    if(result.Contains(BaseDirHolder))
    result = ReplaceClientDir(result, clientDir);
    return result;
  }
}

using static Benchmark;
public class BenchmarkController
{
  
  public static Benchmark[] Benchmarks { get; } // benchmarks
  public static int[][] BenchTransitions { get; } // transitions
  
  static BenchmarkController()
  {
    Benchmarks = new[]
    {
      new Benchmark(04, "wordcount", $"example wordcount {InDirHolder} {OutDirHolder}", $"{BaseDirHolder}/wcout", $"{BaseDirHolder}/rantw"),
    };
  }
}

public class Client : Component
{
    public string StartBenchmark(Benchmark benchmark)
    {
        if(benchmark.HasOutputDir)
        SubmittingConnector.RemoveHdfsDir(benchmark.GetOutputDir(ClientDir));
        var appId = SubmittingConnector.StartApplicationAsync(benchmark.GetStartCmd(ClientDir));
    }
}
\end{lstlisting}

Der \textbf{dynamische Teil} des Controllers ist für die Auswahl der auszuführenden Anwendung zuständig, was auch die Auswahl der initial auszuführenden Anwendung einschließt.
Zur Auswahl der initialen Anwendung wird basierend auf der \acl{sl}"=Anwendung das Transitionssystem genutzt und so eine Anwendung ausgewählt, die keine Eingabedaten benötigt bzw. diese für andere Anwendungen generiert.

Das im vorherigen Abschnitt definierte und im statischen Teil implementierte Transitionssystem kommt auch immer dann zum Einsatz, wenn Entschieden werden muss, welche Anwendung der derzeit ausgeführten Anwendung folgt.
Jeder Client bzw. sein \texttt{BenchmarkController} entscheidet unabhängig von anderen Clients einmal pro \sS-Takt, welche Anwendung ausgeführt wird.

\begin{lstlisting}[label=lst:benchmarkChanging,style=cs,
caption={[Normalisierung und Auswahl der nachfolgenden Anwendung]
    Normalisierung und Auswahl der nachfolgenden Anwendung (gekürzt)}]
// g§§et probabilities f§§rom current benchmark
var transitions = BenchTransitions[CurrentBenchmark.Id];

var ranNumber = RandomGen.NextDouble();
var cumulative = 0D;
for(int i = 0; i < transitions.Length; i++)
{
  cumulative += transitions[i];
  if(ranNumber >= cumulative)
  continue;
  
  // save benchmarks
  PreviousBenchmark = CurrentBenchmark;
  CurrentBenchmark = Benchmarks[i];
}
\end{lstlisting}

Nachdem eine neue Anwendung ausgewählt wurde, muss zunächst sichergestellt werden, dass die bisher ausgeführte Anwendung beendet ist.
Dafür wird der in \cref{lst:hadoopAppKill} dargestellte Befehl von Hadoop zum Abbruch von Anwendungen ausgeführt, wodurch die derzeit ausgeführte Anwendung beendet wird, sollte sie noch nicht abgeschlossen sein.
Im Anschluss kann das von der neuen Anwendung benötigte \ac{HDFS}"=Ausgabeverzeichnis gelöscht werden, bevor die Anwendung selbst gestartet wird.

Eine Anwendung wird wie in \cref{lst:benchmarkDefinition} gezeigt zwar asynchron gestartet, allerdings wird zunächst noch synchron auf die Ausgabe der \texttt{applicationId} gewartet.
Die gesamte Ausgabe einer zu startenden Anwendung ist in \cref{lst:hadoopAppStart} zu finden.
Die ID wird vom Cluster im Rahmen der Übergabe und Initialisierung der Anwendung vergeben.
Erst nachdem diese bekannt ist, wird die restliche Ausführung der Anwendung asynchron durchgeführt.
Benötigt wird die ID damit der zu startende Client die Anwendung im Falle eines Anwendungswechsels in den folgenden Takten beenden kann.
Ohne die direkte Speicherung der ID wäre es sonst nicht möglich, klar entscheiden zu können, welchem Client die Anwendung zugeordnet ist.
Dies ist auch der Grund, weshalb kein HiBench"=Workload in das Transitionssystem aufgenommen wurde, da hier die \texttt{applicationId} gemeinsam mit der gesamten Ausgabe der einzelnen HiBench"=Anwendungen erst nach Abschluss der Ausführung ausgegeben wird.
Gespeichert wird die ID zunächst in einer noch verfügbaren \texttt{YarnApp}"=Instanz, welche anschließend selbst in \texttt{Client.CurrentExecutingApp} gespeichert wird.
