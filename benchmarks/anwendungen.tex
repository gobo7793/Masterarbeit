\section{Übersicht möglicher Anwendungen}
\label{sec:appOverview}

Hadoop-Benchmark enthält bereits die Möglichkeit, unterschiedliche Benchmarks zu starten.
Wie in \cref{sec:hadoopBenchmark} erwähnt, sind folgende Benchmarks in der Plattform bereits integriert:

\begin{itemize}
    \item Hadoop Mapreduce Examples
    \item Intel HiBench
    \item \ac{SWIM}
\end{itemize}

Jeder Benchmark enthält ein spezielles Start"=Script, um die jeweiligen Benchmarks in einem Docker"=Container zu starten.
Dieser Container wird abhängig vom \texttt{HostMode} (vgl. \cref{subsec:hostMode}) auf der VM des \ac{RM} oder direkt auf dem Host gestartet.
Da es in Docker"=Umgebungen \emph{best practice} ist, für jeden Einsatzzweck ein eigenes Image zu erstellen bzw. Container zu starten, werden für jede der drei Benchmark"=Suites eigene Container ausgeführt.
Um mehrere Benchmarks gleichzeitig starten und ausführen zu können, wurden die Startscripte der Benchmarks entsprechend angepasst.

\subsection{Mapreduce Examples}
\label{subsec:mapreduceExamples}

Die Hadoop Mapreduce Examples sind unterschiedliche und meist voneinander unabhängige Anwendungen, die beispielhaft für die meisten Anwendungsfälle in einem produktiv genutzten Cluster sind.
Die Examples sind Teil der Hadoop"=Installation und daher standardmäßig in jedem Hadoop"=Cluster verfügbar.
Einige der Anwendungen der Examples sind:

\begin{itemize}
    \item Generatoren für Text und Binärdaten, \zB \acl{rtw}
    \item Analysieren von Daten, \zB \acl{wc}
    \item Sortieren von Daten, \zB \acl{so}
    \item Ausführen von komplexen Berechnungen, \zB die Ausführung der \emph{Bailey-Borwein-Plouffe-Formel} zur Berechnung einzelner Stellen von $\pi$
\end{itemize}

\subsection{Intel HiBench}
\label{subsec:hibench}

Intel HiBench\footnote{\url{https://github.com/intel-hadoop/HiBench}} ist eine von Intel entwickelte Benchmark"=Suite mit \emph{Workloads} zu verschiedenen Anwendungszwecken mit jeweils unterschiedlichen einzelnen Anwendungen.
Die in \cite{Huang2010} entwickelte Suite enthielt anfangs nur wenige Anwendungen, wurde im Laufe der Zeit jedoch stetig um neue Anwendungen und auch Workloads erweitert.
Das zeigt sich auch darin, dass in Hadoop"=Benchmark die HiBench"=Version \mbox{2.2} integriert ist, die einen noch deutlich geringeren Umfang an Workloads und Anwendungen besitzt, wie \zB die aktuellere Version 7.
Aus diesem Grund wurde vor der Analyse der Anwendungen der HiBench"=Suite das Docker"=Image entsprechend auf Version 7 aktualisiert, um hier entsprechend alle in der Zwischenzeit hinzugefügten Workloads und Anwendungen der Suite nutzen zu können.
HiBench enthält damit folgende Workloads mit einer unterschiedlichen Anzahl an möglichen Anwendungen:

\begin{itemize}
    \item Micro"=Benchmarks (basierend auf den Mapreduce Examples und den Jobclient"=Tests)
    \item Maschinelles Lernen
    \item SQL/Datenbanken
    \item Websuche
    \item Graphen
    \item Streaming
\end{itemize}

\subsection{\acs{SWIM}}
\label{subsec:swim}

\textbf{\ac{SWIM}}\footnote{\url{https://github.com/SWIMProjectUCB/SWIM}} ist eine Benchmark"=Suite, die aus 50 verschiedenen Workloads besteht.
Das besondere an \ac{SWIM} ist, dass die Suite im Rahmen der Studie \cite{Chen2012} entwickelt wurde, und dadurch anhand mehrerer tausend real ausgeführten \ac{MR}"=Jobs entwickelt wurde.
Die dabei enthaltenen Workloads stellen damit eine größere Vielfalt an ausgeführten Anwendungen und damit einen größeren Testumfang dar als vergleichbare Benchmarks\cite{SwimWikiHome}.

Bei der Ausführung auf dem in dieser Fallstudie verwendetem Cluster wurden jedoch nicht alle Workloads fehlerfrei ausgeführt.
Zudem wird in \cite{InriaTutorial} explizit erwähnt, dass die Ausführung auf einem Cluster auf einem Host sehr zeitintensiv ist, sofern die Workloads überhaupt ausgeführt werden können.
\ac{SWIM} ist außerdem für Benchmarks eines Clusters mit mehreren physischen Nodes ausgelegt, weshalb die Ausführung in dieser Fallstudie extrem viel Zeit benötigten würde.
Daher wurde die Nutzung des \ac{SWIM}"=Benchmarks nicht weiter verfolgt.

\subsection{Jobclient"=Tests}
\label{subsec:jobclient}

Ebenfalls im Installationsumfang von Hadoop enthalten sind die hier aufgrund ihres Dateinamens als Jobclient"=Tests bezeichneten Anwendungen.
Hauptbestandteil dieser Tests sind vor allem weitere, den Mapreduce Examples ergänzende, Benchmarks, welche das gesamte Cluster oder einzelne Nodes testen.
Der Fokus der Jobclient"=Tests liegt im Gegensatz zu den Examples nicht auf dem \ac{MR}- bzw. \ac{YARN}"=Framework, sondern beim \ac{HDFS}.
Da die Jobclient-Tests kein Teil von Hadoop"=Benchmark sind, wurde zur Ausführung der Jobclient"=Test zunächst ein eigenes Start"=Script analog zur Ausführung der Mapreduce Examples erstellt, damit diese ebenfalls im Rahmen der Plattform Hadoop"=Benchmark gestartet werden können.
Die Jobclient"=Tests enthalten \uA folgende Arten an Anwendungen:

\begin{itemize}
    \item \ac{HDFS}"=Systemtests, \zB \texttt{SilveTest}
    \item Reine Lastgeneratoren, \zB \texttt{NNloadGenerator}
    \item Eingabe/Ausgabe"=Durchsatz"=Tests, \zB \texttt{TestDFSIO}
    \item Dummy"=Anwendungen \acl{sl} und \acl{fl}
\end{itemize}
