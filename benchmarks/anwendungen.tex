\section{Übersicht möglicher Anwendungen}
\label{sec:appOverview}

Hadoop-Benchmark enthält bereits die Möglichkeit, unterschiedliche Benchmarks zu starten.
Wie in \cref{sec:hadoopBenchmark} erwähnt, sind folgende Benchmarks in der Plattform integriert:

\begin{itemize}
    \item Hadoop Mapreduce Examples
    \item Intel HiBench
    \item \ac{SWIM}
\end{itemize}

Jeder Benchmark enthält zum Starten ein jeweiliges Start"=Script, mit dem ein neuer Docker"=Container auf der Controller"=VM gestartet wird, mit dem die Anwendungen des Benchmarks an das Cluster übergeben werden.
Dass dafür jeweils eigene Docker-Container genutzt werden liegt daran, dass es in Docker"=Umgebungen \emph{best practice} ist, einen Docker"=Container für nur einen Einsatzzweck zu erstellen bzw. zu nutzen.
Die Hauptgründe dafür sind, dass dadurch die Skalierbarkeit erhöht und die Wiederverwendbarkeit gesteigert wird \cite{DockerfileBestPractice}.
Daher wurden im Rahmen dieser Arbeit die bestehenden Startscripte der Plattform für die Benchmarks so angepasst, dass die jeweiligen Benchmarks mehrfach gleichzeitig gestartet werden können.

Die \textbf{Hadoop Mapreduce Examples} sind unterschiedliche und meist voneinander unabhängige Anwendungen, die beispielhaft für die meisten Anwendungsfälle in einem produktiv genutzten Cluster sind.
Die Examples sind Teil von Hadoop und daher bei jeder Hadoop"=Installation enthalten.
Einige der Anwendungen der Examples sind:

\begin{itemize}
    \item Generatoren für Text und Binärdaten, \zB \acl{rtw}
    \item Analysieren von Daten, \zB \acl{wc}
    \item Sortieren von Daten, \zB \acl{so}
    \item Ausführen von komplexen Berechnungen, \zB \emph{Bailey-Borwein-Plouffe-Formel} zur Berechnung einzelner Stellen von $\pi$
\end{itemize}

\textbf{Intel HiBench} ist eine von Intel entwickelte Benchmark-Suite mit \emph{Workloads} zu verschiedenen Anwendungszwecken mit jeweils unterschiedlichen einzelnen Anwendungen.
Der anfangs nur wenige Anwendungen enthaltene Benchmark \cite{Huang2010} wurde stetig mit neuen Anwendungsarten und Workloads erweitert.
Das zeigt sich auch darin, dass in in Hadoop"=Benchmark noch die HiBench"=Version \mbox{2.2} verwendet wird, die einen noch deutlich geringeren Umfang an Workloads und Anwendungen besitzt, als die aktuelle Version 7.
Daher wurde der der Docker"=Container von HiBench zunächst auf die aktuelle Version 7 aktualisiert.
HiBench enthält damit folgende Workloads mit einer unterschiedlichen Anzahl an möglichen Anwendungen:

\begin{itemize}
    \item Micro"=Benchmarks (basierend auf den Mapreduce"=Examples und den Jobclient"=Tests)
    \item Maschinelles Lernen
    \item SQL/Datenbanken
    \item Websuche
    \item Graphen
    \item Streaming
\end{itemize}

\textbf{\ac{SWIM}} ist eine Benchmark-Suite, die aus 50 verschiedenen Workloads besteht.
Das besondere dabei ist, dass die dabei verwendeten Mapreduce"=Jobs anhand mehrerer tausend Jobs erstellt wurden und im Vergleich zu anderen Benchmarks eine größere Vielfalt an Anwendungen und somit ein größerer Testumfang gewährleistet wird \cite{SwimWikiHome}.
Bei der Ausführung auf dem in dieser Fallstudie verwendetem Cluster wurden jedoch nicht alle Workloads fehlerfrei ausgeführt.
Zudem wird in \cite{InriaTutorial} explizit erwähnt, dass es bei der Ausführung auf einem Cluster auf einem einzelnen PC bzw. Laptop Probleme geben kann.
SWIM ist außerdem für Benchmarks eines Clusters mit mehreren physischen Nodes ausgelegt, weshalb die Ausführung in dieser Fallstudie extrem viel Zeit benötigten würde.
Daher wurde die Nutzung des SWIM-Benchmarks nicht weiter verfolgt.

Ebenfalls im Installationsumfang von Hadoop enthalten sind die hier aufgrund ihres Dateinamens als \textbf{Jobclient"=Tests} bezeichneten Anwendungen.
Hauptbestandteil dieser Tests sind vor allem weitere, den Examples ergänzende, Benchmarks, welche das gesamte Cluster oder einzelne Nodes testen.
Der Fokus der Jobclient"=Tests liegt im Gegensatz zu den Examples nicht auf dem MapReduce- bzw. YARN"=Framework, sondern beim \ac{HDFS}.
Da die Jobclient-Tests kein Teil von Hadoop"=Benchmark sind, wurde zur Ausführung der Jobclient"=Test zunächst ein eigenes Start"=Script analog zur Ausführung der Mapreduce"=Examples erstellt, damit hierfür ebenfalls ein eigener Docker"=Container gestartet wird.
Die Jobclient"=Tests enthalten \uA folgende Arten an Anwendungen:

\begin{itemize}
    \item \ac{HDFS}"=Systemtests, \zB \texttt{SilveTest}
    \item Reine Lastgeneratoren, \zB \texttt{NNloadGenerator}
    \item Eingabe/Ausgabe"=Durchsatz"=Tests, \zB \texttt{TestDFSIO}
    \item Dummy"=Anwendungen \acl{sl} (blockiert Ressourcen, führt aber nichts aus) und \acl{fl} (Anwendung schlägt immer fehl)
\end{itemize}
