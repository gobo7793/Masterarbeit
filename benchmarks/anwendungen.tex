\section{Übersicht möglicher Anwendungen}\label{sec:appOverview}

% Hadoop-Benchmark enthält bereits alles zum starten
Hadoop-Benchmark enthält bereits die Möglichkeit, unterschiedliche Benchmarks zu starten. Wie in \autoref{sec:hadoopBenchmark} erwähnt, sind folgende Benchmarks in der Plattform integriert:

\begin{itemize}
    \item Hadoop Mapreduce Examples
    \item Intel HiBench
    \item \ac{SWIM}
\end{itemize}

Jeder Benchmark enthält zum Starten ein jeweiliges Start"=Script, mit dem ein neuer Docker"=Container auf der Controller"=VM gestartet wird, mit dem der Benchmark auf dem Cluster gestartet wird. Dass dafür ein eigener Docker-Container genutzt wird liegt daran, dass es in Docker"=Umgebungen \emph{best practice} ist, einen Docker"=Container für nur einen Einsatzzweck zu erstellen und zu nutzen. Die Hauptgründe dafür sind, dass dadurch die Skalierbarkeit erhöht und die Wiederverwendbarkeit gesteigert werden \cite{DockerBestPractice}. Daher wurden auch die Startscripte für die Benchmarks so angepasst, dass die jeweiligen Benchmarks mehrfach gestartet werden können.

% Examples
Die \textbf{Hadoop Mapreduce Examples} sind unterschiedliche und meist voneinander unabhängige Anwendungen, die beispielhaft für die meisten Anwendungsfällen in einem produktiv genutzten Cluster sind. Die Examples sind Teil von Hadoop und daher bei jeder Hadoop"=Installation enthalten. Einige der Anwendungen der Examples sind:

\begin{itemize}
    \item Generatoren für Text und Binärdaten, \zB \texttt{randomtextwriter}
    \item Analysieren von Daten, \zB \texttt{wordcount}
    \item Sortieren von Daten, \zB \texttt{sort}
    \item Ausführen von komplexen Berechnungen, \zB \emph{Bailey-Borwein-Plouffe-Formel} zur Berechnung einzelner Stellen von $\pi$
\end{itemize}

% HiBench
\textbf{Intel HiBench} ist eine Benchmark-Suite mit \emph{Workloads} zu verschiedenen Anwendungszwecken mit jeweils unterschiedlichen einzelnen Anwendungen. Da in Hadoop"=Benchmark noch die HiBench"=Version \mbox{2.2} verwendet wird, wurde der Docker"=Container von HiBench zunächst auf Version 7 aktualisiert, die einige neuen Workloads und Anwendungen enthält. HiBench enthält damit folgende Workloads mit einer unterschiedlichen Anzahl an möglichen Anwendungen:

\begin{itemize}
    \item Micro"=Benchmarks (basieret auf den Mapreduce"=Examples und den Jobclient"=Tests)
    \item Maschinelles Lernen
    \item SQL/Datenbanken
    \item Websuche
    \item Graphen
    \item Streaming
\end{itemize}

% Kurze infos zum Swim
\textbf{\ac{SWIM}} ist eine Benchmark-Suite, die aus 50 verschiedenen Workloads besteht. Das besondere dabei ist, dass die dabei verwendeten Mapreduce"=Jobs anhand mehrerer tausend Jobs erstellt wurden und im Vergleich zu anderen Benchmarks eine größere Vielfalt an Anwendungen und somit ein größerer Testumfang gewährleistet wird \cite{SwimWikiHome}. Bei der Ausführung auf dem in dieser Arbeit verwendete Cluster gab es jedoch einige Probleme. Zudem wird in \cite{InriaTutorial} explizit erwähnt, dass es bei der Ausführung auf einem Cluster auf einem PC bzw. Laptop Probleme geben kann. Daher wurde die Nutzung des SWIM-Benchmarks nicht weiter verfolgt.

% Hinzufügen von Jobclient
Ebenfalls im Installationsumfang von Hadoop enthalten sind die sog. \textbf{Jobclient"=Tests}. Hauptbestandteil dieser Tests sind vor allem weitere Benchmarks, welche das gesamte Cluster oder einzelne Nodes testen, wobei die meisten Anwendungen das HDFS betreffen. Da die Jobclient-Tests kein Teil von Hadoop"=Benchmark sind, wurde zur Ausführung der Jobclient"=Test zunächst ein eigenes Start"=Script analog zur Ausführung der Mapreduce Examples erstellt, damit hierfür ebenfalls ein eigener Docker"=Container gestartet wird. Die Jobclient"=Tests enthalten \uA folgende Arten an Anwendungen:

\begin{itemize}
    \item HDFS"=Systemtests, \zB \texttt{SilveTest}
    \item Lastgeneratoren, \zB \texttt{NNloadGenerator}
    \item Eingabe/Ausgabe"=Durchsatz"=Tests, \zB \texttt{TestDFSIO}
    \item Dummy"=Anwendungen \texttt{sleep} (blockiert Ressourcen, führt aber nichts aus) und \texttt{fail} (Anwendung schlägt immer fehl)
\end{itemize}