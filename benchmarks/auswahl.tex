\section{Entwicklung des Transitionssystems}
\label{sec:transitionsystem}

Damit die Fallstudie die Realität abbilden kann, wurden von den verfügbaren \glspl{Anwendung} einige ausgewählt und in ein Transitionssystem in Form einer Markow"=Kette überführt.
Diese Kette definiert die Ausführungsreihenfolge zwischen den einzelnen Anwendungen.
Eine zufallsbasierte Markow"=Kette wurde aus dem Grund verwendet, dass auch in der Praxis \glspl{Anwendung} nicht immer in der gleichen Reihenfolge ausgeführt werden und daher auch in der Fallstudie eine unterschiedliche Ausführungsreihenfolge der \glspl{Anwendung} ermöglicht werden soll.
Mithilfe der Festlegung eines bestimmten Seeds für den in der Fallstudie benötigten Pseudo"=Zufallsgenerator besteht bei Bedarf dennoch die Möglichkeit, einen \gls{Test} mit den gleichen \glspl{Anwendung} wiederholen zu können.

\subsection{Auswahl der Benchmarks}
\label{subsec:appSelection}

Einige der in \cref{subsec:mapreduceExamples} erwähnten Mapreduce"=Examples werden häufig als Benchmark verwendet.
Einige Beispiele dafür sind die \glspl{Anwendung} \acrlong{so} und \texttt{grep}, die bereits im Referenzpapier \cite{Dean2004} des \gls{MR}"=Frameworks zum Testen genutzt wurden.
Zum Testen des \gls{HDFS} dient in \cite{Shvachko2010} der DFSIO"=Benchmark, um den Durchsatz beim Lesen und Schreiben einer großen Datenmenge auf dem \gls{HDFS} zu messen.
\acrlong{tsr} und die beiden dazugehörigen Anwendungen \acrlong{tg} und \acrlong{tvl} bilden ebenfalls einen weit verbreiteten Benchmark, der die Hadoop"=Implementierung der standardisierten \emph{Sort Benchmarks}\footnote{\url{https://sortbenchmark.org/}} darstellt \cite{Graves2013}.
Ebenfalls als guter Benchmark dient die \gls{Anwendung} \acrlong{wc}, mit der ein großer Datensatz stark verkleinert bzw. zusammengefasst wird und dient daher als gute Repräsentation für Anwendungsarten, bei denen Daten extrahiert werden \cite{Huang2010,Chen2012}.

Da in dieser Fallstudie ein realistisches Abbild der ausgeführten \glspl{Anwendung} ausgeführt werden soll, ist es nicht sehr hilfreich, die einzelnen Übergangswahrscheinlichkeiten im Transitionssystem anzugleichen oder rein zufällig zu verteilen.
Einen realistischen Einblick, welche Anwendungs- und Datentypen in produktiv genutzten Hadoop"=Clustern genutzt werden, geben \uA die Studien \cite{Chen2012} und \cite{HadoopDataTypes}.
Auffällig ist hierbei, dass die meisten \glspl{Anwendung} in einem Hadoop"=Cluster innerhalb weniger Sekunden oder Minuten abgeschlossen sind und bzw. oder Datensätze im Größenbereich von wenigen Kilobyte bis hin zu wenigen Megabyte verarbeiten.
Zu einem ähnlichen Ergebnis kam auch die Studie \cite{Ren2013}, in der gefolgert wird, dass für kleinere Jobs einfachere Frameworks abseits von Hadoop besser geeignet wären.
Die Autoren der Studie \cite{HadoopDataTypes} bezeichneten Hadoop aufgrund ihrer Ergebnisse als \enquote{potentielle Technologie zum Verarbeiten aller Arten von Daten}, folgerten aber ähnlich wie in \citeauthor{Ren2013} in \cite{Ren2013}, dass mit Hadoop vor allem Daten verarbeitet werden, die auch mit \enquote{traditionellen Plattformen} verarbeitet werden könnten.

Basierend auf den Ergebnissen der Studien in \cite{Huang2010,Chen2012,HadoopDataTypes,Ren2013} und der in den Publikationen \cite{Shvachko2010,Dean2004,Graves2013} verwendeten Benchmarks, wurden folgende \glspl{Anwendung} der Mapreduce"=Examples und Jobclient"=Tests in das Transitionssystem übernommen:

\begin{itemize}
    \item Generieren von Eingabedaten für andere Anwendungen:
    \begin{itemize}
        \item Textdateien:
        \begin{itemize}
            \item \gls{rtw}: Generierung von zufälligen Zeichenfolgen
            \item \gls{dfw}: Schreiben einer großen Datenmenge auf das \gls{HDFS}
        \end{itemize}
        \item Binärdateien:
        \begin{itemize}
            \item \gls{rw}: Generierung von zufälligen Binärdaten
            \item \gls{tg}: Generierung der Eingabedaten für den \acrlong{tsr}"=Benchmark
       \end{itemize}
    \end{itemize}

    \item Verarbeitung von Eingabedaten:
    \begin{itemize}
        \item Auslesen bzw. Zusammenfassen:
        \begin{itemize}
            \item \gls{wc}: Auslesen einer Textdatei und Ermitteln der Anzahl der darin enthaltenen Wörter
            \item \gls{dfr}: Auslesen einer großen Datenmenge auf dem \gls{HDFS}
        \end{itemize}
        \item Transformieren:
        \begin{itemize}
            \item \gls{so}: Sortieren von Daten, wird in dieser Fallstudie zum Sortieren von Textdaten genutzt
            \item \gls{tsr}: Sortieren von großen Binärdatenmengen
        \end{itemize}
        \item Validierung der Transformationen:
        \begin{itemize}
            \item \gls{tms}: Validierung der von \acrlong{so} transformierten Daten
            \item \gls{tvl}: Validierung der vom \acrlong{tsr} sortierten Binärdaten
        \end{itemize}
    \end{itemize}

    \item Ausführen von Berechnungen:
    \begin{itemize}
        \item \acrlong{pi}\glsunset{pi}: Einfache Berechnung von $\pi$ mithilfe der Quasi"=Monte"=Carlo"=Methode.
            Die Monte"=Carlo"=Methode und die darauf basierende Quasi"=Monte"=Carlo"=Methode sind stochastische Verfahren, um komplexe Probleme numerisch lösen zu können.
            Für weitere Informationen hierzu sei auf entsprechende Literatur wie \zB \cite{Korn2010,Lemieux2009} verweisen.
        \item \gls{pt}: Berechnung einer Lösung von Pentomino"=Problemen.
            Hierbei soll eine Fläche aus 64 Quadraten mithilfe von zwölf \emph{Bausteinen} bedeckt werden, wobei jeder Baustein aus fünf Quadraten besteht und nur einmal genutzt werden darf.
            Weitere Informationen hierzu sind in entsprechender Literatur wie \zB in \cite{Golomb1995} zu finden.
    \end{itemize}

    \item Dummy"=Anwendungen:
    \begin{itemize}
        \item  \gls{sl}: Blockieren von Ressourcen
        \item  \gls{fl}: Fehlschlagen einer Anwendung
    \end{itemize}
\end{itemize}

Der Grund für die Berücksichtigung von mehreren gleichen bzw. ähnlichen \glspl{Anwendung} für einige Kategorien liegt darin, dass die unterschiedlichen \glspl{Anwendung} einen unterschiedliche Umfang bzw. Datenrepräsentation (Text- und Binärdaten) repräsentieren.
So stehen die beiden \texttt{TestDFSIO}"=Varianten für eine umfangreichere Datennutzung, während die jeweils anderen \glspl{Anwendung} einen kleineren Umfang repräsentieren.
Ähnlich verhält es sich bei den beiden Berechnungs"=Anwendungen, bei denen die \acrlong{pt}"=Anwendung die deutlich umfangreicheren Berechnungen durchführt.
\texttt{TestDFSIO} enthält zudem die Möglichkeit, Daten zu generieren und zu lesen, weshalb dieser Benchmark in zwei Kategorien als \gls{Anwendung} genutzt wird.

Eine Besonderheit bilden die beiden Dummy"=Anwendnugen.
Beide werden bei der Ausführung dieser Fallstudie dafür genutzt, um zu simulieren, wenn nichts ausgeführt werden soll oder bei der Ausführung der \glspl{Anwendung} ein Fehler auftritt.

Auf die Implementierung einer \gls{Anwendung} der HiBench"=Suite wurde verzichtet.
Da beim Starten einer \gls{Anwendung} durch den Client die vom Cluster zugewiesene Anwendungs"=ID benötigt wird, um die \gls{Anwendung} in späteren Testfällen beenden zu können, kann der HiBench nicht sinnvoll im Testmodell genutzt werden.
Der Grund hierfür ist, dass beim Starten einer \gls{Anwendung} der HiBench"=Suite die ID erst nach Abschluss der gesamten \gls{Anwendung} zurückgegeben wird, womit eine asynchrone Ausführung der \gls{Anwendung} nicht mehr möglich wäre.

\subsection{Entwicklung der Markow"=Kette}
\label{subsec:markovChain}

Basierend auf den ausgewählten \glspl{Anwendung} und der in den Studien genannten Anwendungstypen wurde das Transitionssystem in Form einer Markov"=Kette entwickelt.
Die Markov"=Kette definiert die Wahrscheinlichkeiten, mit denen die ausführenden \glspl{Anwendung} bei der Ausführung eines Testfalls gewechselt werden.
Damit die Übergänge nicht bei jedem \gls{Testfall} stattfinden, sondern \glspl{Anwendung} auch mehrere \glspl{Testfall} lang ausgeführt werden können, wurden Selbst"=Transitionen mit einer Wahrscheinlichkeit von 60 Prozent definiert.

Für die beiden Dummy"=Anwendungen gelten einige Besonderheiten.
Sie können beide unabhängig von der derzeit ausgeführten \gls{Anwendung} mit einer sehr geringen Wahrscheinlichkeit als nachfolgende \gls{Anwendung} ausgewählt werden.
Zudem wurden als den Dummy"=Anwendungen nachfolgende Anwendungen nur solche definiert, die ihrerseits keine Eingabedaten benötigten bzw. diese für andere \glspl{Anwendung} bereitstellen:

\begin{itemize}
    \item \acrlong{dfw}
    \item \acrlong{rtw}
    \item \acrlong{tg}
    \item \acrlong{rw}
    \item \acrlong{pi}
    \item \acrlong{pt}
\end{itemize}

Bei der Entwicklung der Markow"=Kette des Transitionssystems wurde zudem berücksichtigt, welche Anwendungen welche Art von Eingabedaten benötigen.
Dadurch wird sichergestellt, dass benötigte Eingabedaten immer vorhanden sind, da diese ebenfalls im Rahmen der Ausführung der Benchmarks generiert werden können (vgl. \cref{sec:clusterSetup,subsec:testcaseGeneration}).
Anwendungen ohne Eingabedaten können dagegen fast jederzeit ausgeführt werden, wie die entwickelte Markov"=Kette zeigt:

\begin{table}[h]
    \resizebox{\linewidth}{!}{
    \begin{tabu}{l|[1.5pt]c|c|c|c|c|c|c|c|c|c|c|c|c|c}
            & \textit{\acrshort{dfw}} & \textit{\acrshort{rtw}} & \textit{\acrshort{tg}} & \textit{\acrshort{dfr}} & \textit{\acrshort{wc}} & \textit{\acrshort{rw}} & \textit{\acrshort{so}} & \textit{\acrshort{tsr}} & \textit{\acrshort{pi}} & \textit{\acrshort{pt}} & \textit{\acrshort{tms}} & \textit{\acrshort{tvl}} & \textit{\acrshort{sl}} & \textit{\acrshort{fl}} \\ \tabucline[1.5pt]{-}
    	\textit{\acrshort{dfw}} & .600 & .073 &  0   & .145 &  0   &  0   &  0   &  0   & .073 & .073 &  0   &  0   & .018 & .018 \\ \hline
    	\textit{\acrshort{rtw}} & .036 & .600 &  0   &  0   & .145 & .036 & .109 &  0   & .036 &  0   &  0   &  0   & .019 & .019 \\ \hline
    	\textit{\acrshort{tg}}  & 0    & .036 & .600 &  0   &  0   &  0   &  0   & .255 &  0   & .073 &  0   &  0   & .018 & .018 \\ \hline
    	\textit{\acrshort{dfr}} & 0    & .073 &  0   & .600 &  0   & .036 &  0   &  0   & .145 & .109 &  0   &  0   & .018 & .019 \\ \hline
    	\textit{\acrshort{wc}}  & .073 & .109 &  0   &  0   & .600 &  0   & .073 &  0   & .073 & .036 &  0   &  0   & .018 & .018 \\ \hline
    	\textit{\acrshort{rw}}  & 0    & .073 & .073 &  0   &  0   & .600 &  0   &  0   & .109 & .109 &  0   &  0   & .018 & .018 \\ \hline
    	\textit{\acrshort{so}}  & 0    & .073 & .036 &  0   & .073 & .036 & .600 &  0   & .073 &  0   & .073 &  0   & .018 & .018 \\ \hline
    	\textit{\acrshort{tsr}} & 0    &  0   &  0   &  0   &  0   &  0   &  0   & .600 & .109 & .073 &  0   & .182 & .018 & .018 \\ \hline
    	\textit{\acrshort{pi}}  & .145 & .109 &  0   &  0   &  0   &  0   &  0   &  0   & .600 & .109 &  0   &  0   & .018 & .019 \\ \hline
    	\textit{\acrshort{pt}}  & .109 & .109 &  0   &  0   &  0   & .073 &  0   &  0   & .073 & .600 &  0   &  0   & .018 & .018 \\ \hline
    	\textit{\acrshort{tms}} & 0    & .145 &  0   &  0   &  0   & .073 &  0   &  0   & .036 & .109 & .600 &  0   & .018 & .019 \\ \hline
    	\textit{\acrshort{tvl}} & .073 & .109 &  0   &  0   &  0   &  0   &  0   &  0   & .109 & .073 &  0   & .600 & .018 & .018 \\ \hline
    	\textit{\acrshort{sl}}  & .167 & .167 & .167 &  0   &  0   & .167 &  0   &  0   & .167 & .167 &  0   &  0   &  0   &  0   \\ \hline
    	\textit{\acrshort{fl}}  & .167 & .167 & .167 &  0   &  0   & .167 &  0   &  0   & .167 & .167 &  0   &  0   &  0   &  0   
    \end{tabu}}
    \caption{Entwickelte Markov"=Kette für die Anwendungs"=Übergänge in Tabellenform}
    \label{tab:transMatrix}
\end{table}
