\section{Implementierung der Tests}
\label{sec:implTestcases}

Die Implementierung und Ausführung der Testfälle wurde mithilfe von NUnit durchgeführt.
Hierfür wurde eine Testmethode entwickelt und ausgeführt, welche mithilfe der Features von NUnit alle Testfälle ausführen kann:

\lstinputlisting[label=lst:executeTestCases,style=cs,
caption={Methode zur Ausführung der Testfälle}]
{./listings/executeTestCases.cs}

Hierbei werden alle in \autoref{sec:implTestcases} beschriebenen Werte zur Ausführung eines Testfalles festgelegt bevor der Testfall ausgeführt wird.
Damit die bei der Ausführung der Tests generierten Logs einfacher zur Evaluation genutzt werden können, werden die angefallenen Logdateien nach jeder Ausführung in ein entsprechendes Verzeichnis verschoben und gemäß der meisten Testfallparameter umbenannt:

\lstinputlisting[label=lst:moveTestCaseLogs,style=cs,
caption={Verschieben der Logdateien nach der Ausführung eines Testfalls}]
{./listings/moveTestCaseLogs.cs}

Da zunächst alle Testfälle in einem Szenario ohne Mutationen durchgeführt wurden, wurde diese Unterscheidung bei der Verwaltung der Logdateien manuell durchgeführt.
Damit die SSH"=Logs nicht zu groß und damit unübersichtlich werden, wurde das Cluster vor jeder Ausführung komplett neu gestartet.
