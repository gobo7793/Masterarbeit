\section{Durchführung der Tests}
\label{sec:testExecution}

Vor der Durchführung aller Testfälle wurde zunächst eine Systematik entwickelt, nach welcher die Testfälle durchgeführt werden.
Die Grundlage hierzu bilden die in \autoref{sec:testcaseGeneration} und in \autoref{sec:implTestcases} beschriebenen Variablen bzw. Punkte, die einen Testfall definieren.
Ebenfalls eine Rolle spielen die in \autoref{sec:implMutationTests} beschriebenen Mutationsszenarien.

Für die verschiedenen Punkte, die einen Testfall definieren, wurden folgende Werte verwendet:

\begin{table}[h]
    \begin{tabu}{c|[2pt]c|c|c}
    	             & Wert 1 & Wert 2 & Wert 3 \\ \tabucline[2pt]{-}
    	    Seed     & XXX    &  XXX   &  XXX   \\ \hline
    	KF"=Wahrsch. & 0      &  0,3   &  \\ \hline
    	Hosts/Nodes  & 1/4    &  2/6   &  \\ \hline
    	  Clients    & 1      &   2    &
    \end{tabu}
    \caption[Übersicht der zur Testfallgenerierung genutzten Werte]
        {Übersicht der zur Testfallgenerierung genutzten Werte.
        Zur Aktivierung und Deaktivierung von Komponentenfehlern wird die jeweils gleiche generelle Wahrscheinlichkeit genutzt.
        Es wird zudem nur zwischen der Anzahl der Hosts unterschieden, die Anzahl der Nodes pro Host bleibt jeweils gleich.}
    \label{tab:testCaseOverview}
\end{table}

Zur Festlegung der Werte zur generellen Wahrscheinlichkeiten zur Aktivierung bzw. Deaktivierung von Komponentenfehlern wurden zunächst verschiedene Werte simuliert.
\todo{Hinschreiben, dass je über 10.000 mögliche aktivierungen/deaktivierung simuliert wurden?}
Die für die Testfälle ausgewählten Werte stellen hierbei eine ausgewogene Aktivierung bzw. Deaktivierung der Komponentenfehler bei unterschiedlichen Auslastungsgraden der Nodes dar.

Insgesamt ergeben die angegebenen Werte 24 Testfälle, die jeweils mit und ohne Mutationen im realen Cluster ausgeführt werden.
Jeder Testfall wird zudem einmal mit 5 sowie einmal mit 15 Simulations"=Schritten ausgeführt, was für die Evaluation eine Datenbasis von 96 Testausführungen ergibt.
