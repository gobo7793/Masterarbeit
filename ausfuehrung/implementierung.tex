\section{Implementierung der Tests}
\label{sec:implTestcases}

Genauso wie die Simulation wurde zur Implementierung und Ausführung der Testfälle das NUnit"=Framework genutzt.
Alle zur Ausführung der Testfälle der Fallstudie relevanten Methoden wurden zudem in der Klasse \texttt{CaseStudyTests} zusammengefasst, welche auf \texttt{SimulationTests} aufbaut bzw. diese zum Ausführen der Simulation nutzt.
Zur Ausführung der Testfälle wurde folgende Methode entwickelt, bei der mithilfe von NUnit die Testfälle ermittelt werden:

\lstinputlisting[label=lst:executeTestCases,style=cs,
caption={[Methode zur Ausführung der Testfälle]
    Methode zur Ausführung der Testfälle (gekürzt)}]
{./listings/executeTestCases.cs}

Das Starten und Beenden des jeweiligen Cluster dient der automatisierten Ausführung aller Testfälle inkl. denen mit der mutierten Selfbalancing"=Komponente.
Dadurch ist es möglich, das Cluster neben dem normalen Szenario auch im Mutationsszenario zu starten.
Mithilfe der im \texttt{TestCaseSourceAttribute} referenzierten Methode \texttt{GetTestCases()} werden die implementierten Testfälle ermittelt:

\lstinputlisting[label=lst:getTestCases,style=cs,
caption={[Implementierung der Testfälle]
    Implementierung der Testfälle (gekürzt).
    Die hier nicht gezeigten Methoden zur Rückgabe der Implementierten Werte wie \texttt{GetFaultProbabilities()} sind nach dem gleichen Schema aufgebaut wie \texttt{GetSeeds()}.}]
{./listings/getTestCases.cs}

Damit die bei der Ausführung der Tests generierten Logs einfacher zur Evaluation genutzt werden können, werden die angefallenen Logdateien nach jeder Ausführung in ein entsprechendes Verzeichnis verschoben.
Hierbei werden die Logdateien gleichzeitig gemäß der Testfallparameter wie folgt umbenannt:

\lstinputlisting[label=lst:moveTestCaseLogs,style=cs,
caption={Bestimmung des Dateinamens zur Umbenennung der Logdateien}]
{./listings/moveTestCaseLogs.cs}

Da beim Monitoring immer die Daten aller auf dem Cluster ausgeführten Anwendungen übertragen und im SSH"=Log gespeichert werden, hat das Neustarten des Clusters bei jedem Testfall zudem den Nebeneffekt, dass im SSH"=Log keine Daten von ausgeführten Anwendungen eines anderen Testfalls enthalten sind.
