\section{Implementierungen der Mutationstests}
\label{sec:implMutationTests}

Zur Entwicklung der Mutationstest wurde der von \citeauthor{Groce2018} in \cite{Groce2018} vorgestellte \textbf{Universalmutator}\footnote{\url{https://github.com/agroce/universalmutator}} genutzt.
Der Universalmutator ist ein Tool, das den vorhandenen Quellcode eines Programmes so verändert, sodass damit Mutationstests durchgeführt werden können.
Diese werden vor allem in der Forschung eingesetzt, um Testsysteme zu verifizieren, in dem das \ac{SuT} verändert wird.
Ziel hierbei ist es, zu erkennen, dass das \ac{SuT} verändert wurde bzw. nicht korrekt funktioniert.
\todo{Allgemeines zu Mutationstests in Grundlagen?}

Der Universalmutator kann zum Entwickeln von Mutationstests hierbei nicht nur innerhalb einer bestimmten Umgebung bzw. Programmiersprache, sondern prinzipiell für alle Programmiersprachen eingesetzt werden.
Dies wird dadurch ermöglicht, dass die vom Universalmutator generierten Mutationen basierend auf einem oder mehreren Regelsätze durchgeführt werden und somit der Quellcode mutiert wird.
So kann vom Universalmutator Quellcode \uA in den Sprachen Python, Java, C/C++ oder Swift mutiert werden \cite{Groce2018}.

Da bei der Ausführung des Universalmutators auch zahlreiche Mutanten generiert werden, die nicht kompiliert bzw. ausgeführt werden können, nutzt das Tool die Compiler der jeweiligen Sprache zur Validierung der generierten Mutationen.
Ein validierter Mutant zeichnet sich hierbei dadurch aus, dass dieser durch den Original"=Compiler der jeweiligen Sprache kompiliert werden kann und die generierten Objektdateien bzw. Bytecode nicht dem nicht-mutierten Original oder anderen bereits generierten Mutationen entsprechen \cite{Groce2018}.
Diese Validierung kann mithilfe von entsprechenden Startparametern durch ein benutzerdefiniertes Programm durchgeführt werden oder alternativ nicht durchgeführt werden \cite{Groce2018,UniversalmutatorSourceGenmutants}.

Da in dieser Fallstudie nicht nur Hadoop bzw. die Selfbalancing"=Komponente getestet werden soll, sondern vor allem das in den vorherigen Abschnitten und Kapiteln beschriebene Testsystem, wurde auch ein Mutationstest erstellt.
Der hierbei generierten Mutationen wurden im Rahmen eines neuen Szenarios in der Plattform Hadoop"=Benchmark gespeichert (vgl. \autoref{sec:aufbauCluster}).

Zur Entwicklung des Mutationsszenarios wurden mithilfe des Universalmutators insgesamt 431 valide Mutationen aus dem Quellcode der Selfbalancing"=Komponente generiert.
Von allen validen Mutationen wurde für jede der vier Java"=Klassen zufällig eine Mutation ausgewählt, aus der die mutierte Selfbalancing"=Komponente erzeugt wurde.
Die mutierte Komponente wurde anschließend einem speziell hierfür entwickelten Docker"=Image gespeichert, was den Kern des Mutationsszenarios darstellt.

\todo{Auflistung der 4 Mutanten}
