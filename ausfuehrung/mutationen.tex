\section{Generierung der Mutanten}
\label{sec:implMutationTests}

Um das entwickelte Testsystem selbst zu validieren, wurden einige Mutationstests entwickelt.
Mutationstests werden vor allem in der Forschung eingesetzt, um Fehler im \gls{SuT} oder dem Testsystem selbst zu finden, in dem das \gls{SuT} verändert wird.
Ziel ist es, die im \gls{SuT} implementierten Mutanten zu finden, und dabei möglichst weitere Fehlerquellen zu ermitteln \cite{DeMillo1978,Hamlet1977,Jia2011,Groce2018}.

Da sich Mutationstests zu einem großen Forschungszweig entwickelt haben, gibt es hierfür entsprechend viele Tools, um solche Tests zu entwickeln \cite{Jia2011,Groce2018}.
Einige Beispiele hierfür sind PIT\footnote{\url{http://pitest.org/}} und Judy\footnote{\url{http://mutationtesting.org/}} für Java- bzw. Milu\footnote{\url{https://github.com/yuejia/Milu/}} für C"=Programme \cite{Coles2016,Madeyski2010,Jia2008}.

Ein weiteres, sehr einfach zu nutzendes Tool zum Entwickeln von Mutationstests ist der Universalmutator\footnote{\url{https://github.com/agroce/universalmutator/}} \cite{Groce2018}.
Er kann zum Entwickeln von Mutationstests nicht nur innerhalb einer bestimmten Umgebung bzw. Programmiersprache, sondern prinzipiell für alle Programmiersprachen eingesetzt werden.
Dies wird dadurch ermöglicht, da der Universalmutator den Quellcode der Programme verändert, und hierbei einen oder mehrere \gls{Regex}"=basierte Regelsätze anwendet.
So kann vom Universalmutator Quellcode \uA in den Sprachen Python, Java, C/C++ oder Swift mutiert werden \cite{Groce2018}.

Da bei der Ausführung des Universalmutators auch zahlreiche Mutanten erzeugt werden, die nicht kompiliert bzw. ausgeführt werden können, nutzt das Tool die Compiler der jeweiligen Sprache zur Validierung der generierten Mutationen.
Ein validierter Mutant zeichnet sich hierbei dadurch aus, dass dieser durch den Original"=Compiler der jeweiligen Sprache kompiliert werden kann und die generierten Objektdateien bzw. Bytecode nicht dem nicht-mutierten Original oder anderen bereits generierten Mutationen entsprechen \cite{Groce2018}.
Diese Validierung kann mithilfe von entsprechenden Startparametern durch ein benutzerdefiniertes Programm durchgeführt oder alternativ nicht durchgeführt werden \cite{Groce2018,UniversalmutatorSourceGenmutants}.

Da in dieser Fallstudie nicht nur Hadoop bzw. die Selfbalancing"=Komponente getestet werden soll, sondern vor allem das in den vorherigen Abschnitten und Kapiteln beschriebene Testsystem, wurden auch Mutationstests erstellt (vgl. \cref{sec:clusterSetup}).
Hierbei wurden mithilfe des Universalmutators insgesamt 431 valide Mutationen aus dem Quellcode der Selfbalancing"=Komponente generiert.
Von allen validen Mutationen wurden anschließend für jede der vier Klassen der Selfbalancing"=Komponente jeweils ein Mutant zufällig ausgewählt, welche als Basis für die in dieser Fallstudie verwendeten Mutationstests dienen (vgl. \cref{subsec:selfbalancingAnalysis}):

\begin{enumerate}[itemsep=5pt]
    \item
    Zur Ermittlung der Veränderung des \gls{MARP}"=Wertes muss zunächst der jeweils aktuelle Arbeitsspeicher"=Bedarf des Clusters eingelesen werden.
    Dies geschieht im \texttt{Controller} mithilfe einer \texttt{for}"=Schleife, mit deren Hilfe der Speicherverbrauch im Cluster nahezu sekündlich aus der \texttt{memLog}"=Datei eingelesen wird.
    Der Mutant 1 verändert hierbei die Schleifenbedingung, damit die Schleife nicht ausgeführt wird.
    Dadurch wird verhindert, dass der Speicherverbrauch des Clusters vom \texttt{Controller} der Selfbalancing"=Komponente eingelesen und verwendet werden kann.
    Dadurch ist der Speicherverbrauch des Clusters auch nicht für den Algorithmus \cite{Zhang2016} der Selfbalancing"=Komponente verfügbar.
    
    \item 
    Der \texttt{Effectuator} dient dazu, um die Veränderung des \gls{MARP}"=Wertes im Cluster zu speichern.
    Dazu wird das entsprechende Shell"=Script mithilfe der \emph{Bash}"=Shell ausgeführt.
    Der Mutant 2 sorgt dafür, dass anstatt des korrekten Dateipfades der Bash"=Shell (\texttt{/bin/bash}) ein ungültiger Dateipfad (hier \texttt{\%bin/bash}) aufgerufen wird, womit das zur Übertragung des neuen Wertes benötigte Script nicht ausgeführt werden kann.
    
    \item
    Mithilfe des \texttt{ControlNodeMonitor} wird das Schell"=Script zum Ermitteln der Anzahl der aktiven YARN"=Jobs ausgeführt.
    Dies geschieht in einem eigenen Thread, der mithilfe einer \texttt{while}"=Schleife, die solange aktiv ist, solange der Thread aktiv ist.
    Hierbei wird das Script rund einmal pro Sekunde aufgerufen und danach ermittelt, ob bei der Ausführung des Shell"=Scriptes Fehler aufgetreten sind.
    Dazu wird ein entsprechender \texttt{BufferedReader} geöffnet und der Error"=Stream des Scriptes eingelesen und anschließend von der Selfbalancing"=Komponente ausgegeben.
    Umgesetzt wird das mithilfe einer \texttt{while}"=Schleife, die solange den Fehler ausgibt, solange die mithilfe von \texttt{BufferedReader.readLine()} ausgelesene Fehlermeldung nicht \texttt{null} ist, also noch weiteren Text enthält.
    Der Mutant 3 ändert die Schleifenbedingung nun so ab, dass die Schleife durchlaufen wird, solange die Fehlermeldung keinen Text enthält, \texttt{readLine()} also \texttt{null} zurück gibt.
    Dadurch wird der \texttt{ControlNodeMonitor} in einer Dauerschleife gefangen und die Anzahl der aktiven YARN"=Jobs wird einmalig direkt nach dem Start der Selfbalancing"=Komponente ermittelt.
    Dadurch steht die Anzahl der YARN"=Jobs nicht zur Berechnung des \gls{MARP}"=Wertes zur Verfügung.
            
    \item
    Die Klasse \texttt{MemUtilization} dient analog zur \texttt{ControlNodeMonitor}"=Klasse zum Auslesen des Arbeitsspeicher"=Bedarfs des Clusters.
    Der Mutant 4 verhindert hierbei die komplette Ausführung des entsprechenden Threads, indem die Bedingung der Schleife für den gesamten Thread so verändert wurde, dass diese nur dann ausgeführt wird, wenn der Thread nicht aktiv ist.
    Dadurch wird verhindert, dass das entsprechende Shell"=Script überhaupt ausgeführt wird, der Speicherverbrauch somit nicht ausgelesen, und somit auch nicht zur Berechnung des neuen \gls{MARP}"=Wertes zur Verfügung steht.
\end{enumerate}

Als Resultat der Mutanten 1, 3 und 4 kann somit der neue \gls{MARP}"=Wert nicht berechnet werden bzw. im Falle von Mutant 2 der korrekt berechnete \gls{MARP}"=Wert nicht an das Cluster übertragen werden.

Für jede Mutation wurde ein Mutationsszenario im Rahmen der Plattform Hadoop"=Benchmark entwickelt, bei dem keine weitere Mutationen enthalten sind (vgl. \cref{sec:realCluster}).
Zudem wurde ein weiteres Mutationsszenario entwickelt, bei dem alle vier Mutationen enthalten sind.
