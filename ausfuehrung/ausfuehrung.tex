\chapter{Implementierung und Ausführung der Tests}
\label{chap:testExecution}

Um nun Testfälle ausführen zu können, wurde zunächst eine Simulation erstellt, mit der einzelne Testfälle ohne die Aktivierung von Komponentenfehlern ausgeführt werden können.
Die Simulation dient vor allem als Vergleichswert für die Evaluation der gesamten Fallstudie.
Neben der Simulation wurde aber auch ein Analysetest erstellt, bei dem das \sS-Framework die implementierten Komponentenfehler aktiviert und so ermittelt, ob sich das reale Cluster so verhält, wie es erwartet wird.

\section{Implementierung der Simulation}
\label{sec:implSimulation}

Für die Ausführung der Simulation wurden zwei grundlegende Tests implementiert.
Das ist zum einen eine reine Simulation ohne die Aktivierung von Komponentenfehlern, sowie ein weiterer Test, bei dem Komponentenfehler aktiviert werden können.
Ausgeführt werden können die Tests mithilfe des NUnit"=Frameworks.

\subsection{Grundlegender Aufbau}
\label{subsec:simulationBasics}

Da im realen Cluster Hadoop kontinuierlich Anpassungen durchführt und Tests in \ac{ss} mit diskreten Schritten durchgeführt werden, muss beachtet werden, dass die Werte, die beim Test ermittelt werden, immer nur Momentaufnahmen darstellen.
Ebenso muss beachtet werden, dass bei der Deaktivierung von einzelnen Nodes bzw. deren Netzwerkverbindungen diese nicht in Echtzeit, sondern um einige Zeit verzögert erkannt werden und erst nach einer gewissen Zeit aus der Konfiguration des Clusters entfernt werden.
Genauso verhält es sich, wenn ein Node bzw. seine Verbindung wieder aktiviert wird, da dieser zunächst gestartet und die Verbindung mit den YARN"=Controller wiederhergestellt werden muss.
Außerdem werden die für die auf dem Cluster ausgeführten Anwendungen benötigten \ac{AppMstr} und YARN"=Container aufgrund der komplexen internen Prozesse von Hadoop nicht innerhalb weniger Millisekunden allokiert, sondern benötigen ebenfalls eine gewisse Zeit.
Aus diesen Gründen muss ein Simulations"=Schritt um eine gewisse Zeit verzögert werden, sodass alle Aktivitäten innerhalb von Hadoop genügend Zeit zur Ausführung erhalten.

Der grundlegende Ablauf einer Simulation sieht wie folgt aus:

\begin{lstlisting}[label=lst:hadoopSimulation,style=cs,
caption={[Simulation in dieser Fallstudie]
    Simulation in dieser Fallstudie (gekürzt).}]
[Test]
public void SimulateHadoop()
{
  ModelSettings.FaultActivationProbability = 0.0;
  ModelSettings.FaultRepairProbability = 1.0;
  
  var execRes = ExecuteSimulation();
  Assert.IsTrue(execRes, "fatal error occured, see log for details");
}

private bool ExecuteSimulation()
{
  var model = InitModel();
  var isWithFaults = FaultActivationProbability > 0.000001; // prevent inaccuracy
  
  var wasFatalError = false;
  try
  {
    // init simulation
    var simulator = new SafetySharpSimulator(model);
    var simModel = (Model)simulator.Model;
    var faults = CollectYarnNodeFaults(simModel);
    
    SimulateBenchmarks();
    
    // do simuluation
    for(var i = 0; i < StepCount; i++)
    {
      OutputUtilities.PrintStepStart();
      var stepStartTime = DateTime.Now;
      
      if(isWithFaults)
      HandleFaults(faults);
      simulator.SimulateStep();
      
      var stepTime = DateTime.Now - stepStartTime;
      OutputUtilities.PrintDuration(stepTime);
      if(stepTime < ModelSettings.MinStepTime)
      Thread.Sleep(ModelSettings.MinStepTime - stepTime);
      
      OutputUtilities.PrintFullTrace(simModel.Controller);
    }
    
    // collect fault counts and check constraint
  }
  // catch/finally
  
  return !wasFatalError;
}
\end{lstlisting}

Da der Ablauf der Simulation unabhängig von der Aktivierung der Komponentenfehler der gleiche ist, ist hier nur die Variante ohne deren Aktivierung aufgezeigt.
Im Falle einer Aktivierung der Komponentenfehler unterscheiden sich beide Simulationsvarianten nur durch die Angabe der generellen Wahrscheinlichkeiten zum Aktivieren und Deaktivieren der Komponentenfehler.
Da die einzelnen Schritte einer Simulation eine gewisse Mindestdauer haben, wird nach jedem Schritt geprüft, wie viel Zeit für die Ausführung des Schrittes benötigt wurde.
Liegt die Zeit unterhalb der Mindestdauer für einen Schritt, wird die Ausführung des nächsten Schrittes solange hinausgezögert, bis die Mindestdauer des Schrittes erreicht wurde.
Weitere zeitliche Verzögerungen während der Ausführung eines Simulations"=Schrittes sind in \cref{subsec:simulationStep} beschrieben.

Wenn während der Simulation eine im Modell nicht behandelte \texttt{Exception} auftritt, wird diese außerhalb der Simulation abgefangen und entsprechend geloggt.
Dadurch wird zudem die Simulation beim aktuellen Stand abgebrochen.
Nach Abschluss der Simulation werden immer alle zu dem Zeitpunkt mit Komponentenfehlern injizierten Nodes neu gestartet.

\subsection{Initialisierung des Modells}
\label{subsec:simulationModelInit}

Bevor das Modell im Simulator ausgeführt werden kann, muss es initialisiert werden.
Das folgende \cref{lst:hadoopSimulationInit} zeigt die Definition der Felder zur Modellinitialisierung sowie die entsprechenden Methoden, die in \cref{lst:hadoopSimulation} zur Initialisierung aufgerufen werden:

\begin{lstlisting}[label=lst:hadoopSimulationInit,style=cs,
caption={Initialisierung des Modells für die Simulation}]
public TimeSpan MinStepTime { get; set; } = new TimeSpan(0, 0, 0, 25);
public int BenchmarkSeed { get; set; } = Environment.TickCount;
public int StepCount { get; set; } = 3;
public bool PrecreatedInputs { get; set; } = true;
public bool RecreatePreInputs { get; set; } = false;
public double FaultActivationProbability { get; set; } = 0.25;
public double FaultRepairProbability { get; set; } = 0.5;
public int HostsCount { get; set; } = 1;
public int NodeBaseCount { get; set; } = 4;
public int ClientCount { get; set; } = 2;

private Model InitModel()
{
  ModelSettings.HostMode = ModelSettings.EHostMode.Multihost;
  ModelSettings.HostsCount = HostsCount;
  ModelSettings.NodeBaseCount = NodeBaseCount;
  ModelSettings.IsPrecreateBenchInputsRecreate = RecreatePreInputs;
  ModelSettings.IsPrecreateBenchInputs = PrecreatedInputs;
  ModelSettings.RandomBaseSeed = BenchmarkSeed;
  
  var model = Model.Instance;
  model.InitModel(appCount: StepCount, clientCount: ClientCount);
  model.Faults.SuppressActivations();
  
  return model;
}
\end{lstlisting}

Die einzelnen Eigenschaften für die Simulation werden vor dem Initialisieren des Modells in den \texttt{ModelSettings} gespeichert.
Die dort gespeicherten Werte werden wiederum zum Initialisieren der Modell"=Instanz bzw. während der Ausführung der Simulation genutzt.

Einige Eigenschaften haben lediglich einen Zweck, während andere umfangreichere Auswirkungen besitzen.
Die einfachen Eigenschaften sind:

\begin{description}
    \item [MinStepTime] \hfill \\
        Definiert die Mindestdauer eines Schrittes.
        
    \item[BenchmarkSeed] \hfill \\
        Gibt den Seed an, mit dem die Zufallsgeneratoren in den Klassen \texttt{Benchmark""Controller} und \texttt{NodeFaultAttribute} initialisiert werden.
        Dadurch wird es ermöglicht, einzelne Testfälle erneut ausführen zu können.
        
    \item[StepCount] \hfill \\
        Definiert die Anzahl der ausgeführten Schritte.
        
    \item[FaultActivationProbability] \hfill \\
        Definiert die generelle Häufigkeit zum Aktivieren von Komponentenfehlern.
        Ist dieser Wert 0,0, werden grundsätzlich keine Komponentenfehler aktiviert, bei einem Wert von 1,0 werden Komponentenfehler dagegen immer aktiviert.
        
    \item[FaultRepariProbability] \hfill \\
        Definiert die generelle Häufigkeit zum Deaktivieren von Komponentenfehlern.
        Die hier definierte Wahrscheinlichkeit verhält sich analog zu \texttt{\_FaultActivation""Probability}.
        Bei einem Wert von 0,0 werden Komponentenfehler niemals deaktiviert, während sie bei einem Wert von 1,0 im nachfolgenden Schritt immer deaktiviert werden.
        
    \item[HostsCount] \hfill \\
        Definiert die Anzahl der in der Simulation verwendeten Hosts.
        Benötigt wird dieser Wert, damit zu jedem verwendeten Host eine SSH"=Verbindung aufgebaut werden kann.
        
    \item[NodeBaseCount] \hfill \\
        Definiert die Anzahl der Nodes auf Host1.
        Auf Host2 wird die Hälfte der Nodes verwendet.
        Benötigt wird dieser Wert, um mithilfe der REST"=API auf die Hadoop"=Nodes zugreifen zu können, um die Daten der YARN"=Container zu ermitteln.
        
    \item[ClientCount] \hfill \\
        Definiert die Anzahl der zu simulierenden Clients.
        Da jeder Client gleichzeitig nur eine Anwendung startet, wird dadurch gleichzeitig definiert, wie viele Anwendungen gleichzeitig auf dem Cluster ausgeführt werden sollen.
\end{description}

Eine Besonderheit bildet die Eigenschaft \texttt{PrecreatedInputs}.
Es definiert, ob die ausgeführten Anwendungen auf dem Cluster vorab generierte Eingabedaten nutzen oder alle Eingabedaten während der Ausführung selbst generieren.
Der Unterschied zwischen beiden Varianten liegt darin, dass vorab generierte Eingabedaten in einem anderen Verzeichnis im \ac{HDFS} gespeichert sind und während der Simulation die Eingabedaten aus diesem Verzeichnis gelesen werden.
Wenn keine Eingabedaten vorab generiert werden, werden als Eingabeverzeichnisse für die Anwendungen die Ausgabeverzeichnisse der entsprechenden Benchmarks genutzt, die die dafür benötigten Daten generieren.
Die Eigenschaft \texttt{RecreatePreInputs} definiert hierfür, ob bereits bestehende Eingabedaten neu generiert werden, was standardmäßig nicht der Fall ist bzw. dieses Feld auf \texttt{false} gesetzt ist.
Der genaue Ablauf der Bereitstellung der Eingabedaten wird in \todo{Vorabgenerierung der Eingabedaten irgendwo schreiben und hier drauf verweisen} beschrieben.

Die Auswirkungen der in \texttt{InitModel()} definierten Einstellung \texttt{ModelSettings.Host""Mode} wird bereits in \todo{ModelSettings.HostMode beschrieben und hier verweisen} beschrieben.

Die direkt im Anschluss an die Initialisierung des Simulators ausgerufene Methode \texttt{CollectYarnNodeFaults()} ermittelt alle im initialisierten Modell enthaltenen Komponentenfehler, die mit dem \texttt{NodeFaultAttribute} markiert sind:

\begin{lstlisting}[label=lst:hadoopSimulationCollectFaults,style=cs,
caption={[Ermitteln der Komponentenfehler mit dem NodeFaultAttribute]
    Ermitteln der Komponentenfehler mit dem \texttt{NodeFaultAttribute}}]
private FaultTuple[] CollectYarnNodeFaults(Model model)
{
  return (from node in model.Nodes      
    from faultField in node.GetType().GetFields()
    where typeof(Fault).IsAssignableFrom(faultField.FieldType)
    
    let attribute = faultField.GetCustomAttribute<NodeFaultAttribute>()
    where attribute != null
    
    let fault = (Fault)faultField.GetValue(node)
    
    select Tuple.Create(fault, attribute, node, new IntWrapper(0), new IntWrapper(0))
  ).ToArray();
}
\end{lstlisting}

Die gefundenen Komponentenfehler werden als Array aus Tupel, bestehend aus dem Komponentenfehler selbst, dem Attribut sowie dem dazugehörigen Node zurückgegeben.
Zur Speicherung hierfür dient der Typ \texttt{FaultTuple}, welcher ein Alias für das hierfür genutzte \texttt{Tupel<T>} darstellt.
Die jeweiligen Instanzen der Attribute und Nodes werden für die in \cref{subsubsec:simulationFaultActivation} beschriebene Aktivierung der dazugehörigen Komponentenfehler benötigt.
Die beiden im Tupel gespeicherten Instanzen des \texttt{IntWrapper} dienen zur Speicherung der Anzahl der Aktivierungen bzw. Deaktivierungen der Komponentenfehler.
Da der Wert einer Struktur wie \texttt{int} nicht direkt in einem Tupel geändert werden kann, dient die Klasse \texttt{IntWrapper} hierfür als Adapter.

\subsection{Ablauf eines Simulations"=Schrittes}
\label{subsec:simulationStep}

Der Ablauf eines Schrittes lässt sich in die folgenden fünf Abschnitte einteilen.
Während die \nameref{subsubsec:simulationFaultActivation} komplett außerhalb des ausgeführten Modells erfolgt (durch die in \cref{lst:hadoopSimulation} aufgerufene \texttt{HandleFaults()}"=Methode), werden die anderen Abschnitte durch die \texttt{Update()}"=Methode des \texttt{YarnController}s innerhalb des Modells während der Ausführung eines Simulations"=Schrittes ausgeführt.
Die \nameref{subsubsec:simulationStepOutput} werden dagegen gemischt durchgeführt.

\subsubsection{Aktivierung und Deaktivierung der Komponentenfehler}
\label{subsubsec:simulationFaultActivation}

Zur Aktivierung der Komponentenfehler gibt es drei Einzelschritte.
Der erste Schritt ist die Prüfung, ob der Fehler bereits aktiviert wurde.
Bei einem derzeit nicht injizierten Komponentenfehler, wird im zweiten Schritt geprüft, ob der Fehler aktiviert werden soll bevor er im dritten Schritt im realen Cluster injiziert wird.

Zur Entscheidung, ob ein Komponentenfehler aktiviert wird, hängt von folgenden Parametern ab:

\begin{itemize}
    \item Von der Auslastung des Nodes im vorhergehenden Simulationsschritt,
    \item von der in \texttt{ModelSettings.FaultActivationProbability} definierten generellen Wahrscheinlichkeit zur Fehleraktivierung,
    \item sowie von einer Zufallszahl.
\end{itemize}

Ob ein Komponentenfehler aktiviert wird, wird folgendermaßen anhand dieser Parameter berechnet:

\begin{lstlisting}[label=lst:faultActivationCalc,style=cs,
caption={[Berechnung der Aktivierung von Komponentenfehlern]
    Berechnung der Aktivierung von Komponentenfehlern (zusammengefasst).}]
var node = Nodes.First(n => n.Name == nodeName);
var nodeUsage = (node.MemoryUsage + node.CpuUsage) / 2;

if(nodeUsage < 0.1) nodeUsage = 0.1;
else if(nodeUsage > 0.9) nodeUsage = 0.9;

NodeUsageOnActivation = nodeUsage; // for using on repairing

var faultUsage = nodeUsage * ActivationProbability * 2;

var probability = 1 - faultUsage;
var randomValue = RandomGen.NextDouble();
Logger.Info($"Activation probability: {probability} < {randomValue}");
return probability < randomValue;
\end{lstlisting}

Die Entscheidung zur Deaktivierung eines Komponentenfehlers verhält sich analog.
Anstatt der generellen Aktivierungswahrscheinlichkeit in \texttt{ModelSettings.Fault""ActivationProbability} wird hierbei die generelle Wahrscheinlichkeit zur Deaktivierung in \texttt{ModelSettings.FaultRepairProbability} genutzt.
Außerdem spielt bei der Deaktivierung die Auslastung des Nodes zum Zeitpunkt der Aktivierung eine Rolle, welche hierzu in Zeile 7 in \cref{lst:faultActivationCalc} entsprechend gespeichert wird.
Der grundlegende Algorithmus zur Entscheidung ist jedoch gleich.

\subsubsection{Ausführung Benchmarks}
\label{subsubsec:simulationBenchmarkExecution}

Damit die Ausführung der Benchmarks vor dem Monitoring der Anwendungen sowie dem Auswerten der Constraints durch das Oracle stattfindet, wird die Ausführung der Benchmarks ebenfalls durch den \texttt{YarnController} initiiert.
Dazu wird vom \texttt{YarnController} aus für jeden Client die entsprechende Methode aufgerufen, welche ihrerseits den in \cref{sec:appImplementation} erläuterten \texttt{BenchmarkController} nutzt, um den folgenden Benchmark zu bestimmen und im Falle eines Wechsels des Benchmarks diesen zu starten:

\begin{lstlisting}[label=lst:hadoopSimulationStartBenchmark,style=cs,
caption={[Auswahl und Start des nachfolgenden Benchmarks]
    Auswahl und Start des nachfolgenden Benchmarks (gekürzt).
    Die Methode \texttt{BenchmarkController.ChangeBenchmark()} ist bereits in \cref{lst:benchmarkChanging} aufgeführt.}]
public void UpdateBenchmark()
{
    var benchChanged = BenchController.ChangeBenchmark();
    
    if(benchChanged)
    {
        StopCurrentBenchmark();
        StartBenchmark(BenchController.CurrentBenchmark);
    }
}
\end{lstlisting}

Da ein Client auf dem Cluster nur eine Anwendung gleichzeitig ausführt, wird zunächst der zuvor ausgeführte Benchmark abgebrochen.
Bevor der neue Benchmark im Anschluss auf dem Cluster gestartet werden kann, wird zunächst geprüft, ob das Ausgabeverzeichnis der Anwendung im \ac{HDFS} vorhanden ist und gelöscht, da die Anwendung auf dem Cluster andernfalls nicht gestartet werden kann.
Beim Starten der zum Benchmark zugehörigen Anwendung wird zunächst solange gewartet, bis der Anwendung vom \ac{RM} eine \emph{Application ID} zugewiesen wurde, da diese in einer \texttt{YarnApp}"=Instanz sowie in \texttt{Client.CurrentExecutingAppId} gespeichert wird.
Sollte keine \texttt{YarnApp}"=Instanz mehr verfügbar sein, wird stattdessen eine \texttt{OutOfMemoryException} ausgelöst, da während der Simulation keine neuen Instanzen erzeugt werden dürfen (vgl. \cref{sec:ssharp}).

\subsubsection{Monitoring der ausgeführten Anwendungen}
\label{subsubsec:simulationMonitoring}

\todo{besser mit modellkapiutel koordinieren}
Bevor das Monitoring der Anwendungen durchgeführt wird, wird zunächst fünf Sekunden gewartet, bis der \ac{AppMstr} sowie weitere Container der Anwendung allokiert bzw. gestartet wurden.
Diese Wartezeit ist prinzipiell optional, wird hier jedoch genutzt, damit die Auslastung des Clusters besser ermittelt werden kann.
Die Wartezeit vor dem Monitoring ist bereits in der in\cref{subsec:simulationBasics} beschriebenen Mindestdauer eines Schrittes enthalten.

Beim Monitoring werden zunächst die Daten der Nodes, danach die der Anwendungen, ihrer Attempts und zum Abschluss deren Container ermittelt.
Für das Monitoring selbst gib es zwei Ausführungsvarianten.
Die eine Variante liegt darin, dass jede \texttt{IYarnComponent} (also Nodes, Anwendungen, Attempts und Container) jeweils ihre eigenen Daten ermittelt.
Entwickelt wurde diese Variante vor allem für das Monitoring durch die entsprechenden Kommandozeilen"=Befehle.
Die zweite Variante, welche optimal zur Nutzung der REST"=API von Hadoop ist, liegt darin, dass die jeweils übergeordnete Komponente alle Daten für all ihre jeweils untergeordneten Komponenten ermittelt und zur Speicherung übergibt.
Unterschieden werden die beiden Variante durch die Variable \texttt{IYarnComponent.IsSelfMonitoring}:

\begin{lstlisting}[label=lst:hadoopSimulationMonitoring,style=cs,
caption={[Monitoring der Anwendungen]
    Monitoring der Anwendungen (gekürzt).
    Wenn \texttt{IsSelfMonitoring} auf \texttt{false} gesetzt ist, werden die Daten der Anwendung selbst bereits vom \texttt{YarnController} ermittelt und mithilfe von \texttt{YarnApp.SetStatus} gespeichert, analog zu den Attempts, deren Status hier bereits gespeichert wird.}]
public void MonitorStatus()
{
  if(IsSelfMonitoring)
  {
    var parsed = Parser.ParseAppDetails(AppId);
    if(parsed != null)
    SetStatus(parsed);
  }
  
  var parsedAttempts = Parser.ParseAppAttemptList(AppId);
  foreach(var parsed in parsedAttempts)
  {
    var attempt = // get existing or empty attempt instance
    if(attempt == null)
    // throw OutOfMemoryException
    
    attempt.AppId = AppId;
    attempt.IsSelfMonitoring = IsSelfMonitoring;
    if(IsSelfMonitoring)
    attempt.AttemptId = parsed.AttemptId;
    else
    {
      attempt.SetStatus(parsed);
      attempt.MonitorStatus();
    }
  }
}
\end{lstlisting}

Die \texttt{OutOfMemoryException} im vorangegangenen \cref{lst:hadoopSimulationMonitoring} ist analog zur gleichen Ausnahme beim Starten der Anwendung und wird dann ausgelöst, wenn bereits alle \texttt{YarnAppAttempt}"=Instanzen für diese Anwendung belegt sind.

Das Monitoring der Container bietet eine Besonderheit.
Während bei Anwendungen und Attempts auch die Daten von beendeten Anwendungen ermittelt und gespeichert werden, ist dies bei beendeten Containern nicht der Fall.
Das Monitoring für Container wird nur für zum Zeitpunkt des Monitoring aktive bzw. allokierte Container durchgeführt.
Während bei den Anwendungen und Attempts auch solche, deren Daten ausschließlich beim \ac{TLS} gespeichert sind, ermittelt werden, werden die Daten des \ac{TLS} bei Containern nur als Ergänzung der Daten von derzeit ausgeführten Containern vom \ac{RM} genutzt.
Da Container nur während der der Laufzeit von Anwendungen bzw. Attempts zu deren Ausführung existieren, werden die beim vorherigen Schritt ermittelten Container"=Daten gelöscht, bevor die aktuellen Daten der Container eines Attempts ermittelt werden.

\subsubsection{Validierung durch das Oracle}
\label{subsubsec:simulationOracle}

\todo{besser mit modellkapiutel koordinieren}
Im direkten Anschluss an das Monitoring erfolgt die Validierung der Constraints durch das Oracle.
Das Oracle validiert hierbei analog zum Monitoring zunächst die Nodes und danach die Anwendungen, Attempts und Container auf ihre Constraints.
Hierbei wird zunächst überprüft, ob die in \cref{subsec:functionalRequirements} beschrieben funktionalen Anforderungen an Hadoop in Form der in \todo{constraint implementierung} implementierten Constraints für die jeweiligen Komponenten noch erfüllt werden können.
Ist das nicht der Fall, wird dies geloggt und die weiteren Komponenten geprüft.

Das Oracle überprüft auch, ob für das Cluster eine weitere Rekonfiguration möglich ist.
Dies ist dann der Fall, wenn noch mindestens ein Node vorhanden ist, der keine Fehler aufweist und damit den \emph{State} \texttt{Running} hat:

\begin{lstlisting}[label=lst:hadoopSimulationReconf,style=cs,
caption={Prüfung nach der Möglichkeit weiterer Rekonfigurationen}]
public bool IsReconfPossible()
{
  Logger.Debug("Checking if reconfiguration is possible");
  
  var isReconfPossible = ConnectedNodes.Any(n => n.State == ENodeState.RUNNING);
  if(!isReconfPossible)
  {
    Logger.Error("No reconfiguration possible!");
    throw new Exception("No reconfiguration possible!");
  }
  return true;
}
\end{lstlisting}

Ist eine Rekonfiguration nicht mehr möglich, wird durch die hierbei ausgelöste \texttt{Exception} die gesamte Simulation abgebrochen.

Zum Abschluss eines Schrittes werden die in \cref{subsec:testRequirements} beschriebenen Behauptungen an das Testverfahren selbst validiert.
Hierbei können jedoch nicht alle Behauptungen in Form von Constraints durch das Oracle automatisch während der Ausführung validiert werden.
Von den implementierten Constraints können zudem nicht alle direkt innerhalb des Modells während der Ausführung eines Simulations"=Schrittes validiert werden, weshalb außerhalb der Simulation ebenfalls Constraints definiert sind, die zum Abschluss der Simulation geprüft werden (vgl. \todo{constraint implementierung}).

\subsubsection{Ausgaben während eines Schrittes}
\label{subsubsec:simulationStepOutput}

Die wesentlichen Ausgaben während eines Tests wurden bereits in \cref{subsec:dataOrganisation} definiert und beschrieben.
Neben diesen Daten werden im Programmlog weitere Daten gespeichert, damit die Ausführung eines Testfalles besser nachvollzogen werden kann.
Zudem werden alle Ein"= und Ausgabedaten der SSH"=Verbindungen zwischen dem Modell und dem realen Cluster in einer eigenen Log"=Datei gespeichert.
Dieses SSH"=Log dient dazu, die Ursache von unerwarteten Fehlern herauszufinden.

Neben den bereits beschriebenen Daten werden im Programmlog folgende Daten gespeichert:

\begin{itemize}
    \item Verbundene SSH"=Verbindungen mit ihrer ID zur besseren Zuordnung im SSH"=Log
    \item Ausführung der Erstellung von vorab generierten Eingabedaten
    \item Vollständiger Pfad des Setup"=Scriptes (vgl. \cref{sec:realCluster})
    \item URL des Controllers zur Nutzung der REST"=API
    \item Vorschau auf bzw. derzeit vom \texttt{BenchmarkController} ausgewählte Benchmarks
    \item Ausführung von Komponentenfehlern
    \item Diagnostik"=Daten der YARN"=Komponenten
    \item Welche Constraints bei welchen Komponenten verletzt wurden
    \item Die Information, wenn eine Rekonfiguration nicht möglich ist (vgl. \cref{lst:hadoopSimulationReconf})
\end{itemize}

Nach Abschluss der Simulation wird ein erneutes Monitoring des gesamten Clusters durchgeführt und der hierbei ermittelte Status als finaler Clusterstatus ausgegeben.
Zudem werden einige statistische Kenndaten zur Simulation ausgegeben:

\begin{itemize}
    \item Gesamtdauer der Simulation
    \item Anzahl erfolgreicher Schritte
    \item Anzahl der maximal möglichen, aktivierbaren Komponentenfehler
    \item Anzahl aktivierter und deaktivierter Komponentenfehler
    \item Letzter ermittelter \ac{MARP}"=Wert
    \item Anzahl aller ausgeführten, erfolgreicher, nicht erfolgreicher sowie abgebrochener Anwendungen
    \item Anzahl aller ausgeführten Attempts
    \item Anzahl aller während der Ausführung erkannten Container
    \item Anzahl aller validierten Constraints und fehlerhaften Constraints, getrennt nach \ac{SuT}- und Testsystem"=Constraints
\end{itemize}

Ein möglicher Programmlog sowie das exakte Ausgabeformat für eine Ausführung eines Testfalls findet sich in \cref{app:outputFormat}.

\subsection{Weitere mit der Simulation zusammenhängende Methoden}
\label{subsec:simulationUtilities}

Neben der Ausführung der Simulation mit und ohne der Möglichkeit zur Aktivierung der Komponentenfehler gibt es noch einige weitere Methoden, die mit der Simulation zusammenhängen.
Darüber besteht die Möglichkeit, die vorab generierten Eingabedaten für die Simulation, ohne die Simulation selbst auszuführen, zu generieren.
Da die Generierung der Eingabedaten nur dann durchgeführt wird, wenn die Verzeichnisse im \ac{HDFS} noch nicht vorhanden sind (und somit auch die Daten selbst nicht), besteht auch die Möglichkeit, die bestehenden Eingabedaten zu löschen und anschließend neu zu geniereren \todo{vgl. abschnitt benchcontroller damit und dann evtl. neu formulieren}.
Zudem kann die Simulation der durch den \texttt{BenchmarkController} ausgewählten Benchmarks direkt und ohne die Ausführung der gesamten Simulation durchgeführt werden:

\begin{lstlisting}[label=lst:hadoopSimulationBenchmarks,style=cs,
caption={Simulation der auszuführenden Benchmarks}]
public void SimulateBenchmarks()
{
  for(int i = 1; i <= _ClientCount; i++)
  {
    var seed = _BenchmarkSeed + i;
    var benchController = new BenchmarkController(seed);
    Logger.Info($"Simulating Benchmarks for Client {i} with Seed {seed}:");
    for(int j = 0; j < _StepCount; j++)
    {
      benchController.ChangeBenchmark();
      Logger.Info($"Step {j}: {benchController.CurrentBenchmark.Name}");
    }
  }
}
\end{lstlisting}


\section{Generierung der Mutanten}
\label{sec:implMutationTests}

Zur Entwicklung der für die Mutationstests verwendeten Mutanten wurde der von \citeauthor{Groce2018} in \cite{Groce2018} vorgestellte \textbf{Universalmutator}\footnote{\url{https://github.com/agroce/universalmutator}} genutzt.
Der Universalmutator ist ein Tool, das den vorhandenen Quellcode eines Programmes verändert, sodass damit Mutationstests durchgeführt werden können.
Diese werden vor allem in der Forschung eingesetzt, um Testsysteme zu verifizieren, in dem das \ac{SuT} verändert wird.
Ziel hierbei ist es, mithilfe des Testsystems zu erkennen, dass das \ac{SuT} verändert wurde bzw. nicht korrekt funktioniert.
\todo{Allgemeines zu Mutationstests in Grundlagen?}

Der Universalmutator kann zum Entwickeln von Mutationstests hierbei nicht nur innerhalb einer bestimmten Umgebung bzw. Programmiersprache, sondern prinzipiell für alle Programmiersprachen eingesetzt werden.
Dies wird dadurch ermöglicht, dass die vom Universalmutator generierten Mutanten basierend auf einem oder mehreren Regelsätzen durchgeführt werden und damit der Quellcode verändert wird.
So kann vom Universalmutator Quellcode \uA in den Sprachen Python, Java, C/C++ oder Swift mutiert werden \cite{Groce2018}.

Da bei der Ausführung des Universalmutators auch zahlreiche Mutanten erzeugt werden, die nicht kompiliert bzw. ausgeführt werden können, nutzt das Tool die Compiler der jeweiligen Sprache zur Validierung der generierten Mutationen.
Ein validierter Mutant zeichnet sich hierbei dadurch aus, dass dieser durch den Original"=Compiler der jeweiligen Sprache kompiliert werden kann und die generierten Objektdateien bzw. Bytecode nicht dem nicht-mutierten Original oder anderen bereits generierten Mutationen entsprechen \cite{Groce2018}.
Diese Validierung kann mithilfe von entsprechenden Startparametern durch ein benutzerdefiniertes Programm durchgeführt werden oder alternativ nicht durchgeführt werden \cite{Groce2018,UniversalmutatorSourceGenmutants}.

Da in dieser Fallstudie nicht nur Hadoop bzw. die Selfbalancing"=Komponente getestet werden soll, sondern vor allem das in den vorherigen Abschnitten und Kapiteln beschriebene Testsystem, wurden auch Mutationstests erstellt.
Hierbei wurden mithilfe des Universalmutators insgesamt 431 valide Mutationen aus dem Quellcode der Selfbalancing"=Komponente generiert.
Von allen validen Mutationen wurden anschließend für jede der vier Klassen der Selfbalancing"=Komponente jeweils ein Mutant zufällig ausgewählt, welche als Basis für die in dieser Fallstudie verwendeten Mutationstests dienen:

\begin{enumerate}
    \item
    Zur Ermittlung der Veränderung des \ac{MARP}"=Wertes muss zunächst der jeweils aktuelle Arbeitsspeicher"=Verbrauch im Cluster eingelesen werden.
    Dies geschieht im \texttt{Controller} mithilfe einer \texttt{for}"=Schleife, mit der der Speicherverbrauch im Cluster nahezu sekündlich aus der \texttt{memLog}"=Datei eingelesen wird.
    Der Mutant verändert die Schleifenbedingung, damit die Schleife kein einziges mal ausgeführt wird.
    Dadurch wird verhindert, dass der Speicherverbrauch des Cluster vom \texttt{Controller} eingelesen und verwendet werden kann.
    Dadurch ist der Speicherverbrauch des Clusters auch nicht für den in \cite{Zhang2016} vorgestellten Algorithmus der Selfbalancing"=Komponente verfügbar.
    
    \item 
    Der \texttt{Effectuator} dient dazu, um die Veränderung des \ac{MARP}"=Wertes im Cluster zu speichern.
    Dazu wird das entsprechende Shell"=Script mithilfe der \emph{Bash}"=Shell ausgeführt.
    Der Mutant sorgt dafür, dass anstatt des korrekten Dateipfades der Bash (\texttt{/bin/bash}) ein ungültiger Dateipfad (hier \texttt{\%bin/bash}) aufgerufen wird und somit der neue \ac{MARP}"=Wert nicht in das Cluster übertragen werden kann.
    
    \item
    Mithilfe des \texttt{ControlNodeMonitor} wird das Schell"=Script zum Ermitteln der Anzahl der aktiven \ac{YARN}"=Jobs ausgeführt.
    Dies geschieht in einem eigenen Thread, der mithilfe einer \texttt{while}"=Schleife, die solange aktiv ist, solange der Thread aktiv ist.
    Hierbei wird das Script rund einmal pro Sekunde aufgerufen und danach ermittelt, ob bei der Ausführung des Shell"=Scriptes Fehler aufgetreten sind.
    Dazu wird ein entsprechender \texttt{BufferedReader} geöffnet und der Error"=Stream des Scriptes eingelesen und anschließend von der Selfbalancing"=Komponente ausgegeben.
    Umgesetzt wird das mithilfe einer \texttt{while}"=Schleife, die solange den Fehler ausgibt, solange die mithilfe von \texttt{BufferedReader.readLine()} ausgelesene Fehlermeldung nicht \texttt{null} ist, also noch weiteren Text enthält.
    Der Mutant ändert die Schleifenbedingung nun so ab, dass die Schleife durchlaufen wird, solange die Fehlermeldung keinen Text enthält, \texttt{readLine()} also \texttt{null} zurück gibt.
    Dadurch wird der \texttt{ControlNodeMonitor} in einer Dauerschleife gefangen und die Anzahl der aktiven \ac{YARN}"=Jobs wird einmalig direkt nach dem Start der Selfbalancing"=Komponente ermittelt.
            
    \item
    Die Klasse \texttt{MemUtilization} funktioniert analog wie die \texttt{ControlNodeMonitor}"=Klasse, führt jedoch das Script zum Auslesen des Arbeitsspeicher"=Verbrauches aus dem Cluster aus.
    Dieser Mutant verhindert hierbei die komplette Ausführung des entsprechenden Threads, indem die Bedingung der Schleife für den gesamten Thread so verändert wurde, dass diese nur dann ausgeführt wird, wenn der Thread nicht aktiv ist.
    Dadurch wird verhindert, dass das entsprechende Shell"=Script überhaupt ausgeführt wird und der Speicherverbrauch somit nicht ausgelesen wird.
\end{enumerate}

Aufgrund der verwendeten Mutationen erhält die Selfbalancing"=Komponente bei jedem einzelnen Mutanten bereits nicht alle benötigten Informationen zur Anpassung des \ac{MARP}"=Wertes bzw. kann die Änderung des Wertes nicht in der Konfiguration des Clusters speichern.

Für jede Mutation wurde ein Mutationsszenario im Rahmen der Plattform Hadoop"=Benchmark entwickelt, bei dem keine weitere Mutationen enthalten sind (vgl. \cref{sec:realCluster}).
Zudem wurde ein weiteres Mutationsszenario entwickelt, bei dem alle vier Mutationen enthalten sind.


\section{Auswahl der Testkonfigurationen}
\label{sec:selectTestcases}

Zur Definition der konkret für die Testausführung genutzten Testkonfigurationen, aus denen zur Laufzeit die entsprechenden Testfälle dynamisch generiert werden, werden die jeweiligen Parameter der Konfigurationen mithilfe verschiedener Variablen implementiert (vgl. \cref{subsec:testcaseGeneration}).
Relevant sind hierfür folgende, bereits in \cref{lst:hadoopSimulationInit} gezeigte, Eigenschaften:

\begin{lstlisting}[label=lst:hadoopTest,style=cs,
caption={Zur Definition einer Testkonfiguration relevante Felder}]
public int BenchmarkSeed { get; set; }
public double FaultActivationProbability { get; set; }
public double FaultRepairProbability { get; set; };
public int HostsCount { get; set; }
public int NodeBaseCount { get; set; }
public int ClientCount { get; set; }
\end{lstlisting}

Da die jeweiligen Auswirkungen der Eigenschaften bereits in \cref{subsec:simulationModelInit} erläutert wurden, wird an dieser Stelle hierauf verweisen.

Zur Festlegung dieser Variablen und damit der Testkonfigurationen wurde zunächst eine Systematik entwickelt, nach welcher die Testfälle durchgeführt werden.
Hierfür wurden mithilfe des folgenden Programmcodes zunächst zwei Seeds ermittelt:

\begin{lstlisting}[label=lst:generateTestCaseSeeds,style=cs,
caption={Ermittlung der für die Testkonfigurationen genutzten Basisseeds}]
public void GenerateCaseStudyBenchSeeds()
{
  var ticks = Environment.TickCount;
  var ran = new Random(ticks);
  var s1 = ran.Next(0, int.MaxValue);
  var s2 = ran.Next(0, int.MaxValue);
  Console.WriteLine($"Ticks: 0x{ticks:X}");
  Console.WriteLine($"s1: 0x{s1:X} | s2: 0x{s2:X}");
  // Specific output f§§or generating test c§§ase seeds:
  // Ticks: 0x5829F2
  // s1: 0xAB4FEDD | s2: 0x11399D3
}
\end{lstlisting}

Die beiden ermittelten Seeds 0xAB4FEDD und 0x11399D3 stellen somit die erste Variable einer Testkonfiguration dar.

Zur Festlegung der Werte zur generellen Wahrscheinlichkeiten zur Aktivierung bzw. Deaktivierung von Komponentenfehlern wurden über 20.000 mögliche Aktivierungen und Deaktivierungen mit verschiedenen generellen Wahrscheinlichkeiten und Auslastungsgraden der Nodes simuliert.
Der dabei für alle Testkonfigurationen ausgewählte Wert von 0,3 stellte hierbei einen ausgewogenen Wert zur Aktivierung bzw. Deaktivierung der Komponentenfehler bei unterschiedlichen Auslastungsgraden der Nodes dar.

Die Anzahl der Hosts wurde bei einigen Konfigurationen auf 1 festgelegt, bei den meisten liegt diese jedoch bei 2.
Die Node"=Basisanzahl wurde bei allen Konfigurationen auf 4 festgelegt, da hierbei das Cluster eine ausreichende Größe (4 oder 6 Nodes, vgl. \cref{subsec:hostMode,subsec:simulationModelInit}) besitzt und jedem Node ausreichend Ressourcen zur Verfügung stehen, um Anwendungen auszuführen.
Bei einer zu hohen Basisanzahl erhält jeder einzelne Node geringere Ressourcen, was vor allem die Ausführung bei ressourcenintensiven Anwendungen wie \zB \acrlong{pt} behindert, während bei einer zu geringen das Cluster sehr klein ist und daher keine ausreichende Evaluationsbasis bietet.
Die Anzahl der auszuführenden Testfälle wurde variiert, wodurch in einigen Konfigurationen 5 und in anderen 10 Testfälle auszuführende Testfälle definiert sind.
Ebenso variiert wurde die Anzahl der simulierten Clients, die auf 2, 4 oder 6 Clients festgelegt wurde.

Alle Konfigurationen werden mindestens einmal jeweils mit der Selfbalancing"=Komponente ohne Mutationen sowie in einem der in \cref{sec:implMutationTests} erläuterten Mutationsszenarien ausgeführt.
Von den hiermit möglichen 48 Testkonfigurationen werden die möglichen Konfigurationen mit einem Host und sechs simulierten Clients sowie die möglichen Konfigurationen mit zwei simulierten Clients und zehn Testfällen nicht ausgeführt.
Das ergibt für die Evaluation somit eine Datenbasis von 32 grundlegenden Testkonfigurationen.
Eine Übersicht aller genutzten Testkonfigurationen und ausgeführten Tests mit der jeweils benötigten Ausführungszeit und der Anzahl der dabei tatsächlich ausgeführten Testfälle ist in \cref{app:overviewExecutedTestCases} zu finden.

Bei der Ausführung der Tests zur Evaluation wurden Eingabedaten nicht vorab generiert, sondern während der Ausführung von den Anwendungen direkt generiert (vgl. \cref{sec:clusterSetup,subsec:testcaseGeneration,subsec:precreateInputData,subsec:simulationModelInit}).
Dies liegt darin begründet, da durch die Vorabgenerierung der \gls{MARP}"=Wert manipuliert werden würde, wodurch die Tests an Aussagekraft verlieren.

Dazu wurde die Mindestdauer für einen Testfall bei allen Konfigurationen auf 25 Sekunden festgelegt, da in diesem Zeitraum die meisten \emph{kleinen} Anwendungen erfolgreich durchgeführt werden können, wenn sonst keine Anwendung aktiv ist.
Eine ausreichende Mindestdauer ist vor allem für die Generierung der Eingabedaten für nachfolgende Anwendungen wichtig, da nicht vollständig generierte Daten von abgebrochenen Anwendungen nicht von nachfolgenden Anwendungen genutzt werden können.
Zudem stellt dies eine ausreichende Zeitspanne zur Rekonfiguration von Hadoop dar.


\section{Implementierung der Tests}
\label{sec:implTestcases}

Die Implementierung und Ausführung der Testfälle wurde mithilfe von NUnit durchgeführt.
Hierfür wurde eine Testmethode entwickelt und ausgeführt, welche mithilfe der Features von NUnit alle Testfälle ausführen kann:

\lstinputlisting[label=lst:executeTestCases,style=cs,
caption={Methode zur Ausführung der Testfälle}]
{./listings/executeTestCases.cs}

Hierbei werden alle in \autoref{sec:implTestcases} beschriebenen Werte zur Ausführung eines Testfalles festgelegt bevor der Testfall ausgeführt wird.
Damit die bei der Ausführung der Tests generierten Logs einfacher zur Evaluation genutzt werden können, werden die angefallenen Logdateien nach jeder Ausführung in ein entsprechendes Verzeichnis verschoben und gemäß der meisten Testfallparameter umbenannt:

\lstinputlisting[label=lst:moveTestCaseLogs,style=cs,
caption={Verschieben der Logdateien nach der Ausführung eines Testfalls}]
{./listings/moveTestCaseLogs.cs}

Da zunächst alle Testfälle in einem Szenario ohne Mutationen durchgeführt wurden, wurde diese Unterscheidung bei der Verwaltung der Logdateien manuell durchgeführt.
Damit die SSH"=Logs nicht zu groß und damit unübersichtlich werden, wurde das Cluster vor jeder Ausführung komplett neu gestartet.

