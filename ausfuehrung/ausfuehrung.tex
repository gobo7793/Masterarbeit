\chapter{Implementierung und Ausführung der Tests}
\label{chap:testExecution}

Um nun Testfälle ausführen zu können, wurde zunächst eine Simulation erstellt, mit der einzelne Testfälle ohne die Aktivierung von Komponentenfehlern ausgeführt werden können.
Die Simulation dient vor allem als Vergleichswert für die Evaluation der gesamten Fallstudie.
Neben der Simulation wurde aber auch ein Analysetest erstellt, bei dem das \sS-Framework die implementierten Komponentenfehler aktiviert und so ermittelt, ob sich das reale Cluster so verhält, wie es erwartet wird.

\section{Implementierung der Simulation}
\label{sec:implSimulation}

Für die Ausführung der Simulation wurden zwei grundlegende Tests implementiert.
Das ist zum einen eine reine Simulation ohne die Aktivierung von Komponentenfehlern, sowie ein weiterer Test, bei dem Komponentenfehler aktiviert werden können.
Ausgeführt werden können die Tests mithilfe des NUnit"=Frameworks.

\subsection{Grundlegender Aufbau}
\label{subsec:simulationBasics}

Da im realen Cluster Hadoop kontinuierlich Anpassungen durchführt und Tests in \ac{ss} mit diskreten Schritten durchgeführt werden, muss beachtet werden, dass die Werte, die beim Test ermittelt werden, immer nur Momentaufnahmen darstellen.
Ebenso muss beachtet werden, dass bei der Deaktivierung von einzelnen Nodes bzw. deren Netzwerkverbindungen diese nicht in Echtzeit, sondern um einige Zeit verzögert erkannt werden und erst nach einer gewissen Zeit aus der Konfiguration des Clusters entfernt werden.
Genauso verhält es sich, wenn ein Node bzw. seine Verbindung wieder aktiviert wird, da dieser zunächst gestartet und die Verbindung mit den YARN"=Controller wiederhergestellt werden muss.
Außerdem werden die für die auf dem Cluster ausgeführten Anwendungen benötigten \ac{AppMstr} und YARN"=Container aufgrund der komplexen internen Prozesse von Hadoop nicht innerhalb weniger Millisekunden allokiert, sondern benötigen ebenfalls eine gewisse Zeit.
Aus diesen Gründen muss ein Simulations"=Schritt um eine gewisse Zeit verzögert werden, sodass alle Aktivitäten innerhalb von Hadoop genügend Zeit zur Ausführung erhalten.

Der grundlegende Ablauf einer Simulation sieht wie folgt aus:

\begin{lstlisting}[label=lst:hadoopSimulation,style=cs,
caption={[Simulation in dieser Fallstudie]
    Simulation in dieser Fallstudie (gekürzt).}]
private bool ExecuteSimulation()
{
  var model = InitModel();
  var isWithFaults = FaultActivationProbability > 0.000001; // prevent inaccuracy
  
  var wasFatalError = false;
  try
  {
    // init simulation
    var simulator = new SafetySharpSimulator(model);
    var simModel = (Model)simulator.Model;
    var faults = CollectYarnNodeFaults(simModel);
    
    SimulateBenchmarks();
    
    // d§§o simuluation
    for(var i = 0; i < StepCount; i++)
    {
      OutputUtilities.PrintStepStart();
      var stepStartTime = DateTime.Now;
      
      if(isWithFaults)
      HandleFaults(faults);
      simulator.SimulateStep();
      
      var stepTime = DateTime.Now - stepStartTime;
      OutputUtilities.PrintDuration(stepTime);
      if(stepTime < ModelSettings.MinStepTime)
      Thread.Sleep(ModelSettings.MinStepTime - stepTime);
      
      OutputUtilities.PrintFullTrace(simModel.Controller);
    }
    
    // collect fault counts and check constraint
  }
  // cat§§ch/fin§§ally
  
  return !wasFatalError;
}
\end{lstlisting}

Hierbei gibt es zwei Varianten zum Ausführen der Simulation, welche abhängig von der Aktivierung der Komponentenfehler ist.
Sollen keine Komponentenfehler aktiviert bzw. deaktiviert werden, werden die entsprechenden Variablen zur Festlegung der generellen Wahrscheinlichkeiten (vgl. \cref{subsec:simulationModelInit} entsprechend gesetzt und die Simulation ausgeführt.
Da die einzelnen Schritte einer Simulation eine gewisse Mindestdauer haben, wird nach jedem Schritt geprüft, wie viel Zeit für die Ausführung des Schrittes benötigt wurde.
Liegt die Zeit unterhalb der Mindestdauer für einen Schritt, wird die Ausführung des nächsten Schrittes solange hinausgezögert, bis die Mindestdauer des Schrittes erreicht wurde.

Wenn während der Simulation eine im Modell nicht behandelte \texttt{Exception} auftritt, wird diese außerhalb der Simulation abgefangen und entsprechend geloggt.
Dadurch wird zudem die Simulation beim aktuellen Stand abgebrochen.
Nach Abschluss der Simulation werden immer alle noch ausgeführten Anwendungen beendet und defekte Nodes neu gestartet, sofern nötig.

Für die Simulation selbst sind zudem ebenfalls Constraints definiert.
Dies ist dadurch nötig, da die in \cref{subsec:testRequirements} definierte Anforderung, dass Komponentenfehler injiziert bzw. repariert werden, nicht immer innerhalb eines Testfalls validiert werden können.
Aus diesem Grund wird diese Anforderung in Form von Constraints nach Abschluss einer Simulation durch das Oracle geprüft.

\subsection{Initialisierung des Modells}
\label{subsec:simulationModelInit}

Bevor das Modell im Simulator ausgeführt werden kann, muss es initialisiert werden.
Das folgende \cref{lst:hadoopSimulationInit} zeigt die Definition der Felder zur Modellinitialisierung sowie die entsprechenden Methoden, die in \cref{lst:hadoopSimulation} zur Initialisierung aufgerufen werden:

\begin{lstlisting}[label=lst:hadoopSimulationInit,style=cs,
caption={Initialisierung des Modells für die Simulation}]
public TimeSpan MinStepTime { get; set; } = new TimeSpan(0, 0, 0, 25);
public int BenchmarkSeed { get; set; } = Environment.TickCount;
public int StepCount { get; set; } = 3;
public bool PrecreatedInputs { get; set; } = true;
public bool RecreatePreInputs { get; set; } = false;
public double FaultActivationProbability { get; set; } = 0.25;
public double FaultRepairProbability { get; set; } = 0.5;
public int HostsCount { get; set; } = 1;
public int NodeBaseCount { get; set; } = 4;
public int ClientCount { get; set; } = 2;

private Model InitModel()
{
  ModelSettings.HostMode = ModelSettings.EHostMode.Multihost;
  ModelSettings.HostsCount = HostsCount;
  ModelSettings.NodeBaseCount = NodeBaseCount;
  ModelSettings.IsPrecreateBenchInputsRecreate = RecreatePreInputs;
  ModelSettings.IsPrecreateBenchInputs = PrecreatedInputs;
  ModelSettings.RandomBaseSeed = BenchmarkSeed;
  
  var model = Model.Instance;
  model.InitModel(appCount: StepCount, clientCount: ClientCount);
  model.Faults.SuppressActivations();
  
  return model;
}
\end{lstlisting}

Die einzelnen Eigenschaften für die Simulation werden vor dem Initialisieren des Modells in den \texttt{ModelSettings} gespeichert.
Die dort gespeicherten Werte werden wiederum zum Initialisieren der Modell"=Instanz bzw. während der Ausführung der Simulation genutzt.

Einige Eigenschaften haben lediglich einen Zweck, während andere umfangreichere Auswirkungen besitzen.
Die einfachen Eigenschaften sind:

\begin{description}
    \item [MinStepTime] \hfill \\
        Definiert die Mindestdauer eines Schrittes.
        
    \item[BenchmarkSeed] \hfill \\
        Gibt den Seed an, mit dem die Zufallsgeneratoren in den Klassen \texttt{Benchmark""Controller} und \texttt{NodeFaultAttribute} initialisiert werden.
        Dadurch wird es ermöglicht, einzelne Testfälle erneut ausführen zu können.
        
    \item[StepCount] \hfill \\
        Definiert die Anzahl der ausgeführten Schritte.
        
    \item[FaultActivationProbability] \hfill \\
        Definiert die generelle Häufigkeit zum Aktivieren von Komponentenfehlern.
        Ist dieser Wert 0,0, werden grundsätzlich keine Komponentenfehler aktiviert, bei einem Wert von 1,0 werden Komponentenfehler dagegen immer aktiviert.
        
    \item[FaultRepariProbability] \hfill \\
        Definiert die generelle Häufigkeit zum Deaktivieren von Komponentenfehlern.
        Die hier definierte Wahrscheinlichkeit verhält sich analog zu \texttt{\_FaultActivation""Probability}.
        Bei einem Wert von 0,0 werden Komponentenfehler niemals deaktiviert, während sie bei einem Wert von 1,0 im nachfolgenden Schritt immer deaktiviert werden.
        
    \item[HostsCount] \hfill \\
        Definiert die Anzahl der in der Simulation verwendeten Hosts.
        Benötigt wird dieser Wert, damit zu jedem verwendeten Host eine SSH"=Verbindung aufgebaut werden kann.
        
    \item[NodeBaseCount] \hfill \\
        Definiert die Anzahl der Nodes auf Host1.
        Auf Host2 wird die Hälfte der Nodes verwendet.
        Benötigt wird dieser Wert, um mithilfe der REST"=API auf die Hadoop"=Nodes zugreifen zu können, um die Daten der YARN"=Container zu ermitteln.
        
    \item[ClientCount] \hfill \\
        Definiert die Anzahl der zu simulierenden Clients.
        Da jeder Client gleichzeitig nur eine Anwendung startet, wird dadurch gleichzeitig definiert, wie viele Anwendungen gleichzeitig auf dem Cluster ausgeführt werden sollen.
\end{description}

Eine Besonderheit bildet die Eigenschaft \texttt{PrecreatedInputs}.
Es definiert, ob die ausgeführten Anwendungen auf dem Cluster vorab generierte Eingabedaten nutzen oder alle Eingabedaten während der Ausführung selbst generieren.
Der Unterschied zwischen beiden Varianten liegt darin, dass vorab generierte Eingabedaten in einem anderen Verzeichnis im \ac{HDFS} gespeichert sind und während der Simulation die Eingabedaten aus diesem Verzeichnis gelesen werden.
Wenn keine Eingabedaten vorab generiert werden, werden als Eingabeverzeichnisse für die Anwendungen die Ausgabeverzeichnisse der entsprechenden Benchmarks genutzt, die die dafür benötigten Daten generieren (vgl. \cref{subsec:precreateInputData}).
Die Eigenschaft \texttt{RecreatePreInputs} definiert hierfür, ob bereits bestehende Eingabedaten neu generiert werden, was standardmäßig nicht der Fall ist bzw. dieses Feld auf \texttt{false} gesetzt ist.
Die Werte der beiden Eigenschaften werden daher entsprechend in ihren korrespondierenden Eigenschaften in \texttt{ModelSettings} gespeichert.

Weitere Informationen zum \texttt{HostMode} sind in den Abschnitten \labelcref{subsec:hostMode,subsec:implementedConnectors,subsec:modelClass} zu finden.

Die direkt im Anschluss an die Initialisierung des Simulators ausgerufene Methode \texttt{CollectYarnNodeFaults()} ermittelt alle im initialisierten Modell enthaltenen Komponentenfehler, die mit dem \texttt{NodeFaultAttribute} markiert sind (vgl. \cref{subsec:yarnComponentFaults}):

\begin{lstlisting}[label=lst:hadoopSimulationCollectFaults,style=cs,
caption={[Ermitteln der Komponentenfehler mit dem NodeFaultAttribute]
    Ermitteln der Komponentenfehler mit dem \texttt{NodeFaultAttribute}}]
private FaultTuple[] CollectYarnNodeFaults(Model model)
{
  return (from node in model.Nodes      
    from faultField in node.GetType().GetFields()
    where typeof(Fault).IsAssignableFrom(faultField.FieldType)
    
    let attribute = faultField.GetCustomAttribute<NodeFaultAttribute>()
    where attribute != null
    
    let fault = (Fault)faultField.GetValue(node)
    
    select Tuple.Create(fault, attribute, node, new IntWrapper(0), new IntWrapper(0))
  ).ToArray();
}
\end{lstlisting}

Die gefundenen Komponentenfehler werden als Array aus Tupel, bestehend aus dem Komponentenfehler selbst, dem Attribut sowie dem dazugehörigen Node zurückgegeben.
Zur Speicherung hierfür dient der Typ \texttt{FaultTuple}, welcher ein Alias für das hierfür genutzte \texttt{Tupel<T>} darstellt.
Die jeweiligen Instanzen der Attribute und Nodes werden für die in \cref{subsec:faultActivation} beschriebene Aktivierung der dazugehörigen Komponentenfehler benötigt.
Die beiden im Tupel gespeicherten Instanzen des \texttt{IntWrapper} dienen zur Speicherung der Anzahl der Aktivierungen bzw. Deaktivierungen der Komponentenfehler.
Da der Wert einer Struktur wie \texttt{int} nicht direkt in einem Tupel geändert werden kann, dient die Klasse \texttt{IntWrapper} hierfür als Adapter.

\subsection{Weitere mit der Simulation zusammenhängende Methoden}
\label{subsec:simulationUtilities}

Neben der Ausführung der Simulation mit und ohne der Möglichkeit zur Aktivierung der Komponentenfehler gibt es noch einige weitere Methoden, die mit der Simulation zusammenhängen.
So kann \zB die in \cref{subsec:precreateInputData} beschrieben Vorabgenerierung der Eingabedaten durch eine entsprechende Methode durchgeführt werden.
Es ist aber auch möglich, die Simulation der durch den Benchmark"=Controller ausgewählten Benchmarks die Ausführung der gesamten Simulation durchzuführen.
Hierzu kann die bei der Ausführung der Simulation aufgerufene Methode \texttt{SimulateBenchmark()} als eigener Test durchgeführt werden:

\begin{lstlisting}[label=lst:hadoopSimulationBenchmarks,style=cs,
caption={Simulation der auszuführenden Benchmarks}]
[Test]
public void SimulateBenchmarks()
{
  for(int i = 1; i <= _ClientCount; i++)
  {
    var seed = _BenchmarkSeed + i;
    var benchController = new BenchmarkController(seed);
    Logger.Info($"Simulating Benchmarks for Client {i} with Seed {seed}:");
    for(int j = 0; j < _StepCount; j++)
    {
      benchController.ChangeBenchmark();
      Logger.Info($"Step {j}: {benchController.CurrentBenchmark.Name}");
    }
  }
}
\end{lstlisting}

\subsection{Ablauf eines Tests und der Testfälle}
\label{subsec:simulationStep}

Zu Beginn eines Tests werden zunächst die in \cref{subsec:dataOrganisation} definierten, grundlegenden Daten des Tests im Programmlog gespeichert.
Daneben werden aber auch noch weitere, den \texttt{HostMode} (vgl. \cref{subsec:hostMode}) betreffende Daten im Programmlog abgespeichert:

\begin{itemize}
    \item Verbundene SSH"=Verbindungen mit ihrer ID zur besseren Zuordnung im SSH"=Log
    \item Ausführung der Erstellung von vorab generierten Eingabedaten
    \item Vollständiger Pfad des verwendeten Setup"=Scriptes
    \item Adresse des Controllers zur Nutzung der REST"=API
    \item Simulation der auszuführenden Benchmarks (vgl. \cref{subsec:simulationUtilities}
\end{itemize}

Im Anschluss werden das auszuführende \ac{YARN}"=Modell und der zur Ausführung des Tests genutzte Simulator initialisiert, bevor die Testfälle selbst ausgeführt werden.

Der Ablauf eines Testfalls lässt sich mehrere Abschnitte einteilen.
Zunächst wird vom Simulator selbst mithilfe der in \cref{subsec:yarnComponentFaults} den Komponentenfehlern zugewiesenen Attribute geprüft bzw. entschieden, ob ein Komponentenfehler aktiviert und im Cluster injiziert wird.
Anschließend führt der Simulator einen Simulations"=Schritt durch, führt also für alle Komponenten des Modells die \texttt{Update()}"=Methode aus.
Hierdurch werden deaktivierte Komponentenfehler auch im Cluster wieder repariert, da \texttt{YarnNode.Update()} die entsprechenden Start"=Methoden aufruft (vgl. \cref{subsec:yarnComponentFaults}).

Anschließend wird die in \cref{subsec:yarnController} erläuterte Routine des Controllers ausgeführt.
Dabei wird zunächst der \ac{MARP}"=Wert aus dem Cluster ausgelesen, bevor die Benchmark"=Anwendungen des Testfalls gestartet werden.
Jeder Client nutzt dafür die in \cref{subsec:yarnClient} beschriebe Routine, um zunächst durch den Benchmark"=Controller die zu startende Anwendung auszuwählen (vgl. \cref{subsec:selectionNextBenchmark}) und zu starten.
Die dabei vom Cluster zugewiesene ID der Anwendung wird vom Client gespeichert, um die Anwendung in nachfolgenden Testfällen bei Bedarf abbrechen zu können, wenn eine neue Anwendung gestartet werden soll.

Sobald alle Anwendungen gestartet sind, wird vom Controller das Monitoring aller Nodes, Anwendungen und ihrer Attempts durchgeführt.
Dabei wird auch der \ac{MARP}"=Wert ein zweites mal ausgelesen, da er sich durch die zuvor noch aktiven bzw. neu gestarteten Anwendungen unterschiedlich verändert haben könnte.

Den Abschluss eines Testfalls bildet die Validierung der Werte des Clusters, die beim Monitoring im \ac{YARN}"=Modell gespeichert wurden.
Dazu werden für jede Komponente im Modell die jeweiligen auf den Anforderungen in \cref{sec:requirements} basierenden Constraints (vgl. \cref{subsec:yarnComponentConstraints,subsec:yarnController}) durch das Oracle geprüft (vgl. \cref{subsec:oracleImpl}).
Wenn ein Constraint nicht erfolgreich Validiert wurde, wird dies jeweils im Programmlog ausgegeben bzw. die Ausführung der Simulation abgebrochen, wenn sich das Cluster nicht mehr rekonfigurieren kann.

Nach Abschluss eines Testfalls werden durch den Simulator die in \cref{subsec:dataOrganisation} geforderten Daten im Programmlog abgespeichert, die beim Monitoring erkannt wurden.
Daneben werden bei der Ausführung eines Testfalls aber auch weitere Daten im Programmlog gespeichert wie \zB:

\begin{itemize}
    \item Ausführung von Komponentenfehlern
    \item Diagnostik"=Daten der YARN"=Komponenten
    \item Welche Constraints bei welchen Komponenten verletzt wurden
    \item Die Information, wenn eine Rekonfiguration nicht möglich ist
\end{itemize}

Zum Abschluss der Ausführung des Tests werden zudem die für die Simulation als gesamtes betreffenden Constraints validiert, welche nicht im \ac{YARN}"=Modell selbst implementiert wurden (vgl. \cref{subsec:simulationBasics}).

Nach Abschluss des Tests in Form der Simulation wird ein erneutes Monitoring des gesamten Clusters durchgeführt und der hierbei ermittelte Status als finaler Clusterstatus im Programmlog gespeichert.
Zudem werden einige statistische Kenndaten zum Test im Programmlog gespeichert:

\begin{itemize}
    \item Gesamtdauer der Simulation
    \item Anzahl erfolgreicher Schritte
    \item Anzahl der maximal möglichen, aktivierbaren Komponentenfehler
    \item Anzahl aktivierter und deaktivierter Komponentenfehler
    \item Letzter ermittelter \ac{MARP}"=Wert
    \item Anzahl aller ausgeführten, erfolgreicher, nicht erfolgreicher sowie abgebrochener Anwendungen
    \item Anzahl aller ausgeführten Attempts
    \item Anzahl aller während der Ausführung erkannten Container
    \item Anzahl aller validierten Constraints und fehlerhaften Constraints, getrennt nach \ac{SuT}- und Testsystem"=Constraints
\end{itemize}

Der auszugsweise Programmlog eines Tests, sowie das exakte Ausgabeformat für eine Ausführung eines Testfalls findet sich in \cref{app:outputFormat}.

\section{Generierung der Mutanten}
\label{sec:implMutationTests}

Um das entwickelte Testsystem TestingHadoop selbst zu validieren, wurden einige Mutationstests entwickelt.
Mutationstests werden vor allem in der Forschung eingesetzt, um Fehler im \gls{SuT} oder dem Testsystem selbst zu finden, in dem das \gls{SuT} verändert wird.
Ziel ist es, die im \gls{SuT} implementierten Mutanten zu finden, und dabei möglichst weitere Fehlerquellen zu ermitteln \cite{DeMillo1978,Hamlet1977,Jia2011,Groce2018}.

Da sich Mutationstests zu einem großen Forschungszweig entwickelt haben, gibt es hierfür entsprechend viele Tools, um solche Tests zu entwickeln \cite{Jia2011,Groce2018}.
Einige Beispiele hierfür sind PIT\footnote{\url{http://pitest.org/}} und Judy\footnote{\url{http://mutationtesting.org/}} für Java- bzw. Milu\footnote{\url{https://github.com/yuejia/Milu/}} für C"=Programme \cite{Coles2016,Madeyski2010,Jia2008}.

Ein weiteres, sehr einfach zu nutzendes Tool zum Entwickeln von Mutationstests ist der Universalmutator\footnote{\url{https://github.com/agroce/universalmutator/}} \cite{Groce2018}.
Er kann zum Entwickeln von Mutationstests nicht nur innerhalb einer bestimmten Umgebung bzw. Programmiersprache, sondern prinzipiell für alle Programmiersprachen eingesetzt werden.
Dies wird dadurch ermöglicht, da der Universalmutator den Quellcode der Programme verändert, und hierbei einen oder mehrere \gls{Regex}"=basierte Regelsätze anwendet.
So kann vom Universalmutator Quellcode \uA in den Sprachen Python, Java, C/C++ oder Swift mutiert werden \cite{Groce2018}.

Da bei der Ausführung des Universalmutators auch zahlreiche Mutanten erzeugt werden, die nicht kompiliert bzw. ausgeführt werden können, nutzt das Tool die Compiler der jeweiligen Sprache zur Validierung der generierten Mutationen.
Ein validierter Mutant zeichnet sich hierbei dadurch aus, dass dieser durch den Original"=Compiler der jeweiligen Sprache kompiliert werden kann und die generierten Objektdateien bzw. Bytecode nicht dem nicht-mutierten Original oder anderen bereits generierten Mutationen entsprechen \cite{Groce2018}.
Diese Validierung kann mithilfe von entsprechenden Startparametern durch ein benutzerdefiniertes Programm durchgeführt oder alternativ bei der Generierung der Mutanten übersprungen werden \cite{Groce2018,UniversalmutatorSourceGenmutants}.

Da in dieser Fallstudie nicht nur Hadoop bzw. die Selfbalancing"=Komponente getestet werden soll, sondern vor allem das entwickelte Testsystem TestingHadoop, wurden hierfür auch entsprechende Mutationstests erstellt (vgl. \cref{sec:clusterSetup}).
Hierbei wurden mithilfe des Universalmutators insgesamt 431 valide Mutationen aus dem Quellcode der Selfbalancing"=Komponente generiert.
Von allen validen Mutationen wurde anschließend für jede der vier Klassen der Selfbalancing"=Komponente jeweils ein Mutant zufällig ausgewählt, welche als Basis für die in dieser Fallstudie verwendeten Mutationstests dienen (vgl. \cref{subsec:selfbalancingAnalysis}):

\begin{enumerate}[itemsep=5pt]
    \item
    Zur Ermittlung der Veränderung des \gls{MARP}"=Wertes muss zunächst der jeweils aktuelle Arbeitsspeicher"=Bedarf des Clusters eingelesen werden.
    Dies geschieht im \texttt{Controller} mithilfe einer \texttt{for}"=Schleife, mit deren Hilfe der Speicherverbrauch im Cluster etwa sekündlich aus der \texttt{memLog}"=Datei eingelesen wird.
    Der Mutant~1 verändert hierbei die Schleifenbedingung so, damit die Schleife nicht ausgeführt wird.
    Dadurch wird verhindert, dass der Speicherverbrauch des Clusters vom \texttt{Controller} der Selfbalancing"=Komponente eingelesen und verwendet werden kann.
    Dadurch ist der Speicherverbrauch des Clusters auch nicht für den Algorithmus \cite{Zhang2016} der Selfbalancing"=Komponente verfügbar.
    
    \item 
    Der \texttt{Effectuator} dient dazu, um die Veränderung des \gls{MARP}"=Wertes in die Konfiguration des Clusters zu übertragen.
    Dazu wird das entsprechende Shell"=Script mithilfe der Bash"=Shell ausgeführt.
    Der Mutant~2 sorgt dafür, dass anstatt des korrekten Dateipfades der Bash"=Shell (\texttt{/bin/bash}) ein ungültiger Dateipfad (hier \texttt{\%bin/bash}) aufgerufen wird, womit das zur Übertragung des neuen Wertes benötigte Script nicht ausgeführt werden kann.
    
    \item
    Mithilfe des \texttt{ControlNodeMonitor} wird das Schell"=Script zum Ermitteln der Anzahl der aktiven YARN"=Jobs ausgeführt.
    Dies geschieht in einem eigenen Thread, der mithilfe einer \texttt{while}"=Schleife, die solange aktiv ist, solange der Thread aktiv ist.
    Hierbei wird das Script rund einmal pro Sekunde aufgerufen und danach ermittelt, ob bei der Ausführung des Shell"=Scriptes Fehler aufgetreten sind.
    Dazu wird ein entsprechender \texttt{BufferedReader} geöffnet und der Error"=Stream des Scriptes eingelesen und anschließend von der Selfbalancing"=Komponente ausgegeben.
    Umgesetzt wird das mithilfe einer \texttt{while}"=Schleife, die solange den Fehler ausgibt, solange die mithilfe von \texttt{BufferedReader.readLine()} ausgelesene Fehlermeldung nicht \texttt{null} ist, also noch weiteren Text enthält.
    Der Mutant~3 ändert die Schleifenbedingung nun so ab, dass die Schleife durchlaufen wird, solange die Fehlermeldung keinen Text enthält, \texttt{readLine()} also \texttt{null} zurück gibt.
    Dadurch wird der \texttt{ControlNodeMonitor} in so einem Fall in einer Dauerschleife gefangen und die Anzahl der aktiven YARN"=Jobs wird nur einmalig nach dem Start der Selfbalancing"=Komponente ermittelt.
    Dadurch steht die Anzahl der YARN"=Jobs nicht zur Berechnung des \gls{MARP}"=Wertes zur Verfügung.
            
    \item
    Die Klasse \texttt{MemUtilization} dient analog zur \texttt{ControlNodeMonitor}"=Klasse zum Auslesen des Arbeitsspeicher"=Bedarfs des Clusters.
    Der Mutant~4 verhindert hierbei die komplette Ausführung des entsprechenden Threads, indem die Bedingung der Schleife für den gesamten Thread so verändert wird, dass diese nur dann ausgeführt wird, wenn der Thread nicht aktiv ist.
    Dadurch wird verhindert, dass das entsprechende Shell"=Script überhaupt ausgeführt wird, der Speicherverbrauch somit nicht ausgelesen und damit auch nicht zur Berechnung des neuen \gls{MARP}"=Wertes zur Verfügung steht.
\end{enumerate}

Als Resultat der Mutanten 1, 3 und 4 kann somit der neue \gls{MARP}"=Wert nicht berechnet bzw. im Falle von Mutant 2 der korrekt berechnete \gls{MARP}"=Wert nicht an das Cluster übertragen werden.

Für jede Mutation wurde ein Mutationsszenario im Rahmen der Plattform Hadoop"=Benchmark entwickelt, bei dem keine weitere Mutationen enthalten sind.
Zudem wurde ein weiteres Mutationsszenario entwickelt, bei dem alle vier Mutationen enthalten sind (vgl. \cref{sec:realCluster}).


\section{Auswahl der Testfälle}
\label{sec:selectTestcases}

Die im in \autoref{sec:testcaseGeneration} beschriebenen Testfälle werden mithilfe verschiedener Variablen implementiert.
Relevant für die Ausführung eines Testfalles sind folgende, bereits in \autoref{lst:hadoopSimulationInit} gezeigte, Eigenschaften:

\lstinputlisting[label=lst:hadoopTest,style=cs,linerange={2-2,6-10},
caption={Zur Definition eines Testfalls relevante Felder}]
{./listings/hadoopSimulationInit.cs}

Da die jeweiligen Auswirkungen der Eigenschaften bereits in \autoref{sec:simulationModelInit} erläutert wurden, wird an dieser Stelle hierauf verweisen.

Zur Festlegung dieser Variablen und damit der Testfälle wurde zunächst eine Systematik entwickelt, nach welcher die Testfälle durchgeführt werden.
Hierfür wurden mithilfe des folgenden Programmcodes zunächst zwei Seeds ermittelt::

\lstinputlisting[label=lst:generateTestCaseSeeds,style=cs,
caption={Ermittlung der für die Testfälle genutzten Basisseeds}]
{./listings/generateTestCaseSeeds.cs}

Die beiden ermittelten Seeds 0x36159C73 und 0x60E70223 wurden jeweils bei jeder Konfiguration zwischen den anderen Variablen genutzt.

Zur Festlegung der Werte zur generellen Wahrscheinlichkeiten zur Aktivierung bzw. Deaktivierung von Komponentenfehlern wurden zunächst über 20.000 mögliche Aktivierungen und Deaktivierungen mit verschiedenen generellen Wahrscheinlichkeiten und Auslastungsgraden der Nodes simuliert.
Der dabei für alle Testfälle ausgewählte Wert von 0,3 stellt hierbei eine ausgewogene Aktivierung bzw. Deaktivierung der Komponentenfehler bei unterschiedlichen Auslastungsgraden der Nodes dar.

Die Anzahl der Hosts wurde bei den meisten Testfällen auf 2 festgelegt, die Anzahl der Nodes pro Host wurde bei allen Testfällen auf 4 festgelegt, was 4 bzw. 6 Nodes macht.
Die Anzahl der Simulations"=Schritte wurde variiert, wodurch einige Testfälle mit 5, andere mit 10 Simulations"=Schritten ausgeführt werden.
Ebenso variiert wurde die Anzahl der simulierten Clients, die meist bei 4 simulierten Clients liegt, bei einigen Testfällen aber auch bei 2 oder 6.

Alle Testfälle wurden je einmal mit der Selfbalancing"=Komponente ohne Mutationen und im Mutationsszenario ausgeführt.
Insgesamt wurden so 40 Testfälle definiert, die als Datenbasis für die Evaluation dienen.

Die Mindestdauer für einen Simulations"=Schritt wurde in allen Fällen auf 25 Sekunden festgelegt, da hierbei ein Großteil der ausgeführten Anwendungen auf dem Cluster erfolgreich beendet werden können.
Dies ist vor allem für die Generierung der Eingabedaten für nachfolgende Anwendungen wichtig, da die generierten Daten von abgebrochenen Anwendungen nicht von nachfolgenden Anwendungen genutzt werden können.
Zudem stellt dies eine ausreichende Zeitspanne zur Rekonfiguration von Hadoop dar.
Bei der Ausführung der Testfälle zur Evaluation wurden Eingabedaten nicht vorab generiert, sondern während der Ausführung von den Anwendungen direkt generiert.


\section{Durchführung der Tests}
\label{sec:testExecution}

% systematik

% übersicht ausgeführter testfälle

