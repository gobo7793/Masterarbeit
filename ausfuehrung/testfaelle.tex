\section{Auswahl der Testfälle}
\label{sec:selectTestcases}

Die im in \autoref{sec:testcaseGeneration} beschriebenen Testfälle werden mithilfe verschiedener Variablen implementiert.
Relevant für die Ausführung eines Testfalles sind folgende, bereits in \autoref{lst:hadoopSimulationInit} gezeigte, Eigenschaften:

\lstinputlisting[label=lst:hadoopTest,style=cs,linerange={2-2,6-10},
caption={Zur Definition eines Testfalls relevante Felder}]
{./listings/hadoopSimulationInit.cs}

Da die jeweiligen Auswirkungen der Eigenschaften bereits in \autoref{sec:simulationModelInit} erläutert wurden, wird an dieser Stelle hierauf verweisen.

Zur Festlegung dieser Variablen und damit der Testfälle wurde zunächst eine Systematik entwickelt, nach welcher die Testfälle durchgeführt werden.
Hierfür wurden mithilfe des folgenden Programmcodes zunächst zwei Seeds ermittelt::

\lstinputlisting[label=lst:generateTestCaseSeeds,style=cs,
caption={Ermittlung der für die Testfälle genutzten Basisseeds}]
{./listings/generateTestCaseSeeds.cs}

Die beiden ermittelten Seeds 0x36159C73 und 0x60E70223 wurden jeweils bei jeder Konfiguration zwischen den anderen Variablen genutzt.

Zur Festlegung der Werte zur generellen Wahrscheinlichkeiten zur Aktivierung bzw. Deaktivierung von Komponentenfehlern wurden zunächst über 20.000 mögliche Aktivierungen und Deaktivierungen mit verschiedenen generellen Wahrscheinlichkeiten und Auslastungsgraden der Nodes simuliert.
Der dabei für alle Testfälle ausgewählte Wert von 0,3 stellt hierbei eine ausgewogene Aktivierung bzw. Deaktivierung der Komponentenfehler bei unterschiedlichen Auslastungsgraden der Nodes dar.

Die Anzahl der Hosts wurde bei den meisten Testfällen auf 2 festgelegt, die Anzahl der Nodes pro Host wurde bei allen Testfällen auf 4 festgelegt, was 4 bzw. 6 Nodes macht.
Die Anzahl der Simulations"=Schritte wurde variiert, wodurch einige Testfälle mit 5, andere mit 10 Simulations"=Schritten ausgeführt werden.
Ebenso variiert wurde die Anzahl der simulierten Clients, die meist bei 4 simulierten Clients liegt, bei einigen Testfällen aber auch bei 2 oder 6.

Alle Testfälle wurden je einmal mit der Selfbalancing"=Komponente ohne Mutationen und im Mutationsszenario ausgeführt.
Insgesamt wurden so 40 Testfälle definiert, die als Datenbasis für die Evaluation dienen.

Die Mindestdauer für einen Simulations"=Schritt wurde in allen Fällen auf 25 Sekunden festgelegt, da hierbei ein Großteil der ausgeführten Anwendungen auf dem Cluster erfolgreich beendet werden können.
Dies ist vor allem für die Generierung der Eingabedaten für nachfolgende Anwendungen wichtig, da die generierten Daten von abgebrochenen Anwendungen nicht von nachfolgenden Anwendungen genutzt werden können.
Zudem stellt dies eine ausreichende Zeitspanne zur Rekonfiguration von Hadoop dar.
Bei der Ausführung der Testfälle zur Evaluation wurden Eingabedaten nicht vorab generiert, sondern während der Ausführung von den Anwendungen direkt generiert.
