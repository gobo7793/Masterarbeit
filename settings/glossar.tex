% Benchmarks
\newacronym[type=apps]{dfw}{dfw}{\texttt{TestDFSIO -write}}
\newacronym[type=apps]{rtw}{rtw}{\texttt{randomtextwriter}}
\newacronym[type=apps]{tg}{tg}{\texttt{teragen}}
\newacronym[type=apps]{dfr}{dfr}{\texttt{TestDFSIO -read}}
\newacronym[type=apps]{wc}{wc}{\texttt{wordcount}}
\newacronym[type=apps]{rw}{rw}{\texttt{randomwriter}}
\newacronym[type=apps]{so}{so}{\texttt{sort}}
\newacronym[type=apps]{tsr}{tsr}{\texttt{terasort}}
\newacronym[type=apps]{pi}{pi}{\texttt{pi}}
\newacronym[type=apps]{pt}{pt}{\texttt{pentomino}}
\newacronym[type=apps]{tms}{tms}{\texttt{testmapredsort}}
\newacronym[type=apps]{tvl}{tvl}{\texttt{teravalidate}}
\newacronym[type=apps]{sl}{sl}{\texttt{sleep}}
\newacronym[type=apps]{fl}{fl}{\texttt{fail}}

% Abkürzungen/Akronyme
%\newacronym[longplural={plural}]{label}{abbrv}{full}
\newacronym{AM}{AM}{ApplicationManager}
\newacronym{AppMstr}{AppMstr}{ApplicationMaster}
\newacronym{DCCA}{DCCA}{Deductive Cause-Consequence Analysis}
\newacronym{HDFS}{HDFS}{Hadoop Distributed File System}
\newacronym{CLI}{CLI}{Kommandozeile}
\newacronym{MARP}{MARP}{\texttt{maximum-am-resource\-percent}}
\newacronym{MC}{MC}{Model Checking}
\newacronym{MCr}{MC}{Model Checker}
\newacronym{MR}{MR}{MapReduce}
\newacronym{NM}{NM}{NodeManager}
\newacronym[firstplural=Regular Expressions (Regex)]{Regex}{Regex}{Regular Expression}
\newacronym{RM}{RM}{ResourceManager}
\newacronym{ss}{S\#}{Safety Sharp}
\newacronym{SWIM}{SWIM}{Statistical Workload Injector for Mapreduce}
\newacronym{TLS}{TLS}{Timeline Server}
\newacronym{YARN}{YARN}{Yet Another Resource Negotiator}

% Glossar
\longnewglossaryentry{Anwendung}{
    name=Anwendung,
    plural=Anwendungen,}
    {Ein auf dem Hadoop-Cluster ausgeführtes Programm.}
\longnewglossaryentry{Attempt}{
    name=Attempt,
    plural=Attempts,}
    {Ausführungsinstanz einer Anwendung auf dem Hadoop-Cluster.}
\longnewglossaryentry{Container}{
        name=Container,
        plural=Container,}
    {1. Ausführungsinstanz eines Tasks einer YARN-Anwendung auf dem Hadoop-Cluster.
        2. Ausgeführte Instanz eines Docker-Images.}
\longnewglossaryentry{Testkonfiguration}{
    name=Testkonfiguration,
    plural=Testkonfigurationen,}
    {Eine Konfiguration bestehend aus mehreren Parametern, die einen Test definieren.
        Die Nummerierung der in \cref{sec:selectTestcases} definierten Konfigurationen erfolgte fortlaufend.}
\longnewglossaryentry{Testfall}{
    name=Testfall,
    plural=Testfälle,}
    {Ein ausgeführter Schritt der Simulation.
        Ein Testfall wird während der Laufzeit, basierend auf einer zugrundeliegenden Testkonfiguration, sowie den Ereignissen und Ergebnissen zuvor ausgeführter Testfälle der zugrundeliegenden Konfiguration generiert.
        In einem Testfall können daher unterschiedliche Komponentenfehler aktiviert und deaktiviert, sowie unterschiedliche Anwendungen gestartet werden, auch wenn sie durch die gleiche Testkonfiguration generiert wurden (vgl. \cref{subsec:simulationStep}).}
\longnewglossaryentry{Test}{
    name=Test,
    plural=Tests,}
    {Eine Ausführung einer Testkonfiguration mit mehreren Testfällen.
        Um mehrmalige Ausführungen einer Testkonfiguration zu kennzeichnen, wurde der jeweiligen Konfiguration eine weitere Ziffer angehängt.
        Alle ausgeführten Test sind in \cref{app:overviewExecutedTestCases} zu finden.}
\longnewglossaryentry{REST}{
    name=REST,
    plural=REST,}
    {Abkürzung für \emph{Representational State Transfer}.
        Programmierparadigma um maschinenlesbare Schnittstellen, \zB für Webservices, bereitzustellen.}
\longnewglossaryentry{Mutationstest}{
    name=Mutationstest,
    plural=Mutationstests,}
    {Test, bei denen das zu testende Programm verändert wird.
        Ziel hierbei ist es, Fehler im Programm oder Testsystem zu finden (vgl. \cref{sec:implMutationTests}).}

%\newdualentry[glossary options][acronym options]{label}{abbrv}{long}{description}
\newdualentry[plural=Systems under Test][]
    {SuT}{SuT}{System under Test}
    {Das mit einem Test zu testende System selbst.}
