\addchap{Verzeichnisse}
\pagestyle{lists}

\renewcommand{\listfigurename}{Abbildungen}
\listoffigures

\renewcommand{\lstlistlistingname}{Listings}
\lstlistoflistings

\renewcommand{\listtablename}{Tabellen}
\listoftables

\addsec{Abkürzungen}

In dieser Masterarbeit wurden folgende Abkürzungen und Akronyme verwendet:

%    \acro{Kuerzel}[Kurzform]{Langform}
%    \acroplural{Kuerzel}[Kurzform des Plurals]{Langform des Plurals}
\begin{acronym}[AppMstr]
    \acro{AM}{ApplicationManager}
    \acro{AppMstr}{ApplicationMaster}
    \acro{DCCA}{Deductive Cause-Consequence Analysis}
    \acro{HDFS}{Hadoop Distributed File System}
    \acro{MARP}{\texttt{maximum-am-resource-percent}}
    \acro{MC}{Model Checking}
    \acro{MCr}[MC]{Model Checker}
    \acro{MR}{MapReduce}
    \acro{NM}{NodeManager}
    \acro{RM}{ResourceManager}
    \acro{ss}[S\#]{Safety Sharp}
    \acro{SuT}{System under Test}
    \acro{SWIM}{Statistical Workload Injector for Mapreduce}
    \acro{TLS}{Timeline-Server}
    \acro{YARN}{Yet Another Resource Negotiator}
\end{acronym}
Für die genutzten Benchmarks (vgl. \autoref{chap:benchmarks}):
\begin{acronym}[AppMstr]
    \acro{dfw}{\texttt{TestDFSIO -write}}
    \acro{rtw}{\texttt{randomtextwriter}}
    \acro{tg}{\texttt{teragen}}
    \acro{dfr}{\texttt{TestDFSIO -read}}
    \acro{wc}{\texttt{wordcount}}
    \acro{rw}{\texttt{randomwriter}}
    \acro{so}{\texttt{sort}}
    \acro{tsr}{\texttt{terasort}}
    \acro{pi}{\texttt{pi}}
    \acro{pt}{\texttt{pentomino}}
    \acro{tms}{\texttt{testmapredsort}}
    \acro{tvl}{\texttt{teravalidate}}
    \acro{sl}{\texttt{sleep}}
    \acro{fl}{\texttt{fail}}
\end{acronym}