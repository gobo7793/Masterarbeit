\begin{abstract}
    Durch eine Automatisierung von Tests lassen sich im Bereich der Softwareentwicklung hohe Kosten einsparen.
    Daher wurden zahlreiche Test"=Frameworks und Möglichkeiten zum Testen von Systemen und ihrer Software entwickelt.
    Ein solches Framework ist \acrshort{ss} (\acrlong{ss}), mit dem mithilfe eines modellbasierten Ansatzes Systeme getestet werden können.
    Mithilfe des \acrshort{ss}"=Frameworks soll nun ein Testsystem entwickelt werden, um hiermit automatisiert ein verteiltes, adaptives Load"=Balancing"=System zu testen.
    Hierfür wurde Apache Hadoop ausgewählt, welches mit einer selbstadaptiven Komponente ergänzt wird.
    Diese selbstadaptive Komponente verändert dynamisch, und basierend auf den derzeit auf dem Hadoop"=Cluster ausgeführten Anwendungen, einige der sonst statischen Einstellungen von Hadoop, womit die verfügbaren Ressourcen des Clusters optimaler genutzt werden können.
    
    Um Hadoop testen zu können, wurde zunächst mithilfe von \acrshort{ss} ein Modell entwickelt, welches die wesentlichen Komponenten des YARN"=Frameworks von Hadoop abbildet.
    Dieses Modell wurde wiederum mithilfe eines hierfür entwickelten Treibers mit einem realen Hadoop"=Cluster verbunden.
    Dadurch wurde es ermöglicht, durch die Testausführung mit \acrshort{ss} unterschiedliche Anwendungen auf dem realen Cluster auszuführen und die Daten der Anwendungen und des Clusters im Modell zu nutzen.
    Um zu prüfen, ob sich das entwickelte Testsystem \emph{TestingHadoop} zur Testautomatisierung eines verteilten, adaptiven Load"=Balancing"=Systems eignet, wurde hierfür eine Fallstudie durchgeführt.
    
    In dieser Masterarbeit werden der Aufbau und Ablauf der durchgeführten Fallstudie sowie die Entwicklung und Implementierung des hierfür genutzten Testsystems TestingHadoop erläutert.
    Es wird gezeigt, welche Besonderheiten bei der Durchführung und Auswertung der Fallstudie aufgetreten sind, und inwiefern sich das entwickelte, modellbasierte TestingHadoop"=System zur Testautomatisierung eines verteilten, adaptiven Load"=Balancing"=Systems eignet.
\end{abstract}

\clearpage
\begin{otherlanguage}{english}
\begin{abstract}
    By automating tests, high costs can be saved in software development.
    Therefore, numerous test frameworks and ways to test systems and their software have been developed.
    One such framework is \acrshort{ss} (\acrlong{ss}), which uses a model-based approach to test systems.
    By using the \acrshort{ss} framework, a test system was developed to automatically test a distributed, adaptive load-balancing system.
    For this, Apache Hadoop was chosen, which is equipped with an adaptive resource manager.
    The manager detect the current usage of the cluster and modify some of Hadoop's otherwise static settings to make a better use of the available resources.
    
    To test hadoop, a \acrshort{ss} model was developed, which contains the essential components of the Hadoop YARN framework.
    To connect the model to a real Hadoop cluster, a driver was developed for this purpose.
    This allows \acrshort{ss} to run different applications on the real cluster and detect the state of the running applications and the cluster.
    To determine the developed test system \emph{TestingHadoop} is suitable for the test automation of a distributed, adaptive load-balancing system, a case study was performed.
    
    This master thesis explains the structure and processes of the case study, as well as the development and implementation of the test system TestingHadoop used for this purpose.
    It shows the won experiences by performing the case study and shows how the developed, model-based TestingHadoop system is suitable for test automation of a distributed, adaptive load-balancing system.
\end{abstract}
\end{otherlanguage}

