\chapter{Einleitung}
\label{chap:intro}

Im Bereich der Softwaretests wird heutzutage sehr viel mit automatisierten Testverfahren gearbeitet.
Dies ist insofern logisch, als dass diese Testautomatisierung einerseits Aufwand und damit andererseits direkt Kosten einer Software einspart.
Daher gibt es vor allem im Bereich der Komponententests zahlreiche Frameworks, mit denen Tests einfach und automatisiert erstellt bzw. ausgeführt werden können.
Ein Beispiel für ein solches Testframework wäre das \emph{xUnit}"=Framework, zu dem \uA JUnit\footnote{\url{https://junit.org}} für Java und NUnit\footnote{\url{https://nunit.org/}} für .NET zählen.
Dabei werden zunächst einzelne Testfälle erstellt und können im Anschluss mit der jeweils aktuellen Codebasis jederzeit ausgeführt werden.
Automatisierte Tests können auch dazu genutzt werden, um einen einzelnen Test mit verschiedenen Eingaben durchzuführen.
Dadurch können verschiedene Eingabeklassen (wie negative oder positive Ganzzahlen) mit sehr geringem Aufwand in einem Test genutzt werden und somit verschiedene Testfälle direkt ausgeführt werden, wodurch eine massive Kosteneinsparung einhergeht \cite{Polo2013}.

Es gibt aber nicht nur Frameworks für Komponententests, sondern auch für modellbasierte Testverfahren wie \zB dem \ac{MC}.
Beim \ac{MC} wird ein Modell mithilfe eines entsprechenden Frameworks automatisiert auf seine Spezifikation getestet und geprüft, unter welchen Umständen diese verletzt wird \cite{Grumberg1999,Habermaier2015}.

In dieser Masterarbeit soll daher nun ein verteiltes, adaptives Load"=Balancing"=System getestet werden.
Hauptziel ist es, zu ermitteln, wie ein modellbasierter Testansatz auf ein komplexes Beispiel übertragen werden kann.
Dafür wird zunächst ein reales System als vereinfachtes Modell nachgebildet und anschließend mithilfe eines \ac{MC} getestet.
Es soll dabei auch ermittelt werden, wie ein reales System in das Modell eingebunden werden kann und wie bei Problemen mit asynchronen Prozessen innerhalb des verteilten Systems umgegangen werden muss.

\todo{vielleicht Testing MapReduce-Based Systems einbringen?}
