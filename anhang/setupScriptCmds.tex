\chapter{Benötigte Befehle des Setup"=Scriptes}
\label{app:setupScriptCmds}

Das Setupscript dient einerseits zur Trennung des genutzten \texttt{HostMode}s, aber auch zur Vereinfachung der benötigten Befehle zur Steuerung des Clusters (vgl. \cref{sec:realCluster}).
Zur Nutzung der implementierten Connectoren muss das genutzte Setupscript mindestens folgende Befehle beinhalten:

\begin{lstlisting}[label=lst:setupscriptHelp,style=plain,
caption={[Benötigte Befehle eines Setupscriptes]
Benötigte Befehle eines Setupscriptes.
Das Setupscript des \mbox{\texttt{Multihost}}"=Modes bietet zum Teil andere Befehle an, besitzt jedoch entsprechende Befehle zur vollständigen Kompatibilität.}]
hadoop start [node-id]  starting hadoop or the given node
hadoop stop [node-id]   stopping hadoop or the given node
hadoop restart [node]   restarts hadoop or the given node
hadoop destroy          destroys hadoop
hadoop info [id] [form]
                    list running containers or node container details
                          and can use --format string

net start <node-id> enables networking interfaces on the given node
net stop <node-id>  disables networking interfaces on the given node

cmd <cmd>           executes the given command on hadoop controller
hdfs <cmd>          executes the hdfs command and prints the exit code
marp                Gets the current MARP value from hadoop config
\end{lstlisting}

