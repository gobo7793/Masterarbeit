\chapter{REST"=API von Hadoop}
\label{app:hadoopRestApi}

Wie bei der Ausgabe der Daten der YARN"=Komponenten über die Kommandozeile können auch bei der Ausgabe mithilfe der REST"=API die Daten als Liste oder als einzelner Report ausgegeben werden.
Der Unterschied zur Kommandozeile liegt jedoch darin, dass die Listenausgaben einem Array der einzelnen Reports entsprechen.
Neben der hier gezeigten und auch in der Fallstudie genutzten Ausgabe im JSON"=Format unterstützt Hadoop auch eine Ausgabe im XML"=Format.
Im Folgenden sind daher beispielhaft die Ausgaben im JSON"=Format für die Anwendungsliste vom \ac{RM} und für Ausführungen vom \ac{TLS} aufgeführt.
Im Rahmen dieser Masterarbeit sind die Rückgaben für Listen von Anwendungen, Attempts, Container und der Nodes vom \ac{RM} und bzw. \ac{NM} (Container) sowie des \ac{TLS} (Attempts und Container)relevant.
Weitere Informationen zur REST"=API sind in der Dokumentation in \cite{HadoopYarnTlServer271,HadoopRmApi271,HadoopNmApi271} zu finden.

\lstinputlisting[label=lst:hadoopAppListRestRm,style=json,
caption={[REST"=-Ausgabe aller Anwendungen vom \acs{RM}]
    REST"=Ausgabe aller Anwendungen vom \ac{RM}.
    Die Liste kann mithilfe verschiedener Query"=Parameter gefiltert werden.\\
    URL: \url{http://addr:port/ws/v1/cluster/apps}}]
{./listings/hadoopAppListRm.json}

\lstinputlisting[label=lst:hadoopAttemptListRestTls,style=json,
caption={[REST"=Ausgabe aller Ausführungen einer Anwendung vom \acs{TLS}]
    REST"=Ausgabe aller Ausführungen einer Anwendung vom \ac{TLS}.\\
    URL: \url{http://addr:port/ws/v1/applicationhistory/apps/{appid}/appattempts}}]
{./listings/hadoopAttemptListTls.json}
