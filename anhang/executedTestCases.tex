\chapter{Übersicht der ausgeführten Tests}
\label{app:overviewExecutedTestCases}

Die folgende Tabelle gibt eine Übersicht über die für diese Fallstudie ausgeführten Tests.
Die Auswahl der Konfigurationen der Tests ist in \cref{sec:selectTestcases} beschrieben.

Die Spalte \emph{Mutanten} gibt an, welche der in \cref{sec:implMutationTests} definierten Mutanten der Selfbalancing"=Komponente genutzt wurden.
In den Spalten \emph{Ausgeführte Testfälle} und \emph{Dauer} ist angegeben, wie viele Testfälle bzw. Simulations"=Schritte vollständig und erfolgreich ausgeführt wurden bzw. wie viel Zeit die jeweiligen Simulationen in Minuten und Sekunden benötigten.
Wenn nicht alle möglichen Testfälle ausgeführt wurden, war im darauf folgenden Testfall eine Rekonfiguration des Clusters nicht mehr möglich und die Simulation wurde abgebrochen (vgl. \cref{subsec:oracleImpl}).

Die Nummerierung der Konfigurationen bzw. Ausführungen erfolgte basierend auf den grundlegenden Testkonfigurationen, bestehend aus Seed, Anzahl der Hosts, Clients und ausgeführten Testfällen sowie der Angabe, ob ein Mutationsszenario verwendet wurde.
Bei mehrmals ausgeführten Testkonfigurationen ist der Konfiguration eine entsprechende Ziffer angehängt, um die jeweilige Ausführung zu Kennzeichnen.
Eine Besonderheit bildet hierbei die Testkonfiguration 10 mit insgesamt 6 Ausführungen, da diese Konfiguration mit verschiedenen Mutanten durchgeführt wurde.

\begin{table}
    \begin{tabu}{c|[1.5pt]c|c|c|c|c|[1.5pt]c|c}
    	\# & Seed      & Hosts & Clients & Testfälle & Mutanten & \makecell{Ausgeführte\\Testfälle} & Dauer \\ \tabucline[1.5pt]{-}
        \makecell{1.1\\1.2}
           & 0xAB4FEDD &   1   &    2    &    5      &  keine   &
                        \makecell{5\\5} &
                                \makecell{2:44\\2:56}                                \\ \hline
    	2  & 0xAB4FEDD &   1   &    2    &    5      & 1,2,3,4  &     5      & 2:34  \\ \hline
    	3  & 0xAB4FEDD &   1   &    4    &    5      &  keine   &     5      & 5:52  \\ \hline
    	4  & 0xAB4FEDD &   1   &    4    &    5      & 1,2,3,4  &     2      & 3:13  \\ \hline
    	6  & 0xAB4FEDD &   1   &    4    &    10     & 1,2,3,4  &     2      & 3:14  \\ \hline
        \makecell{5.1\\5.2}
           & 0xAB4FEDD &   1   &    4    &    10     &  keine   &
                        \makecell{2\\2} &
                                \makecell{3:35\\3:23}                                 \\ \hline
        \makecell{7.1\\7.2}
           & 0xAB4FEDD &   2   &    2    &    5      &  keine   &
                        \makecell{5\\5} &
                                \makecell{2:49\\2:56}                                \\ \hline
    	8  & 0xAB4FEDD &   2   &    2    &    5      & 1,2,3,4  &     5      & 2:23  \\ \hline
        \makecell{9.1\\9.2}
           & 0xAB4FEDD &   2   &    4    &    5      &  keine   &
                        \makecell{5\\5} &
                                \makecell{07:13\\4:49}                               \\ \hline
        \makecell{10.1\\10.2\\10.3\\10.4\\10.5\\10.6}
           & 0xAB4FEDD &   2   &    4    &    5      &
               \makecell{1,2,3,4\\1\\2\\3\\3\\4} &
                        \makecell{5\\5\\5\\5\\5\\5} &
                                \makecell{7:42\\6:17\\6:04\\6:37\\6:21\\6:26}        \\ \hline
    	11 & 0xAB4FEDD &   2   &    4    &    10     &  keine   &     10     & 12:16 \\ \hline
    	12 & 0xAB4FEDD &   2   &    4    &    10     & 1,2,3,4  &     10     & 11:36 \\ \hline
    	13 & 0xAB4FEDD &   2   &    6    &    5      &  keine   &     5      & 8:02  \\ \hline
    	14 & 0xAB4FEDD &   2   &    6    &    5      & 1,2,3,4  &     5      & 6:24  \\ \hline
    	15 & 0xAB4FEDD &   2   &    6    &    10     &  keine   &     5      & 8:41  \\ \hline
    	16 & 0xAB4FEDD &   2   &    6    &    10     & 1,2,3,4  &     5      & 9:26  \\ \tabucline[1.5pt]{-}
    	17 & 0x11399D3 &   1   &    2    &    5      &  keine   &     5      & 3:07  \\ \hline
    	18 & 0x11399D3 &   1   &    2    &    5      & 1,2,3,4  &     5      & 3:02  \\ \hline
    	19 & 0x11399D3 &   1   &    4    &    5      &  keine   &     5      & 5:25  \\ \hline
    	20 & 0x11399D3 &   1   &    4    &    5      & 1,2,3,4  &     3      & 3:22  \\ \hline
    	21 & 0x11399D3 &   1   &    4    &    10     &  keine   &     3      & 4:17  \\ \hline
    	22 & 0x11399D3 &   1   &    4    &    10     & 1,2,3,4  &     3      & 2:50  \\ \hline
    	23 & 0x11399D3 &   2   &    2    &    5      &  keine   &     5      & 4:25  \\ \hline
    	24 & 0x11399D3 &   2   &    2    &    5      & 1,2,3,4  &     5      & 4:22  \\ \hline
    	25 & 0x11399D3 &   2   &    4    &    5      &  keine   &     5      & 4:53  \\ \hline
    	26 & 0x11399D3 &   2   &    4    &    5      & 1,2,3,4  &     5      & 5:47  \\ \hline
    	27 & 0x11399D3 &   2   &    4    &    10     &  keine   &     10     & 10:30 \\ \hline
        \makecell{28.1\\28.2}
           & 0x11399D3 &   2   &    4    &    10     & 1,2,3,4  &
                        \makecell{7\\7} &
                                \makecell{8:17\\7:37}                                \\ \hline
    	29 & 0x11399D3 &   2   &    6    &    5      &  keine   &     5      & 7:03  \\ \hline
    	30 & 0x11399D3 &   2   &    6    &    5      & 1,2,3,4  &     5      & 6:02  \\ \hline
    	31 & 0x11399D3 &   2   &    6    &    10     &  keine   &     7      & 10:21 \\ \hline
        \makecell{31.1\\31.2}
           & 0x11399D3 &   2   &    6    &    10     &  keine   &
                        \makecell{7\\7} &
                                \makecell{10:41\\10:21}                              \\ \hline
    	32 & 0x11399D3 &   2   &    6    &    10     & 1,2,3,4  &     7      & 11:08 
    \end{tabu}
    \caption{Übersicht der ausgeführten Testkonfigurationen}
    \label{tab:testCaseOverview}
\end{table}
