\chapter{\glsentryshort{CLI}"=Befehle von Hadoop}
\label{app:hadoopCmds}

Für jede der vier relevanten YARN"=Komponenten können die Daten jeweils als Liste oder als ausführlicher Report ausgegeben werden.
Im Folgenden sind beispielhaft die dafür notwendigen Befehle für Anwendungen aufgelistet, für Attempts, Container und Nodes sind analoge Befehle verfügbar.
Neben den Monitoring"=Befehlen sind auch einige weitere für diese Arbeit relevante Befehle mit ihren Ausgaben aufgelistet.
Die Ausgaben zu den Befehlen sind hier zudem auf das Wesentliche gekürzt, \uA da Hadoop bei einigen Befehlen ausgibt, über welche Services (in \cref{lst:hadoopAppListCmd} \zB \gls{TLS}, \gls{RM} und \emph{Application History Server}) die Daten ermittelt werden.
Weiterführende Informationen zu den hier aufgeführten Befehlen sowie die vollständige Befehlsreferenz sind in der Dokumentation von Hadoop \cite{HadoopYarnCmds271} zu finden.

\begin{lstlisting}[label=lst:hadoopAppListCmd,style=plain,
caption={[\glsentryshort{CLI}"=Ausgabe der Anwendungsliste]
    \acrshort{CLI}"=Ausgabe der Anwendungsliste.
    Anwendungen können mithilfe der Optionen \mbox{\texttt{-{}-appTypes}} und \mbox{\texttt{-{}-appStates}} gefiltert werden.}]
$ yarn application --list --appStates ALL
18/02/08 15:37:51 INFO impl.TimelineClientImpl: Timeline service address: http://0.0.0.0:8188/ws/v1/timeline/
18/02/08 15:37:51 INFO client.RMProxy: Connecting to ResourceManager at controller/10.0.0.3:8032
18/02/08 15:37:51 INFO client.AHSProxy: Connecting to Application History server at /0.0.0.0:10200
Total number of applications (application-types: [] and states: [NEW, NEW_SAVING, SUBMITTED, ACCEPTED, RUNNING, FINISHED, FAILED, KILLED]):1
Application-Id	Application-Name	Application-Type	User	Queue	State	Final-State	Progress	Tracking-URL
application_1518100641776_0001	QuasiMonteCarlo	MAPREDUCE	root	default	FINISHED	SUCCEEDED	100%	http://controller:19888/jobhistory/job/job_1518100641776_0001
\end{lstlisting}

\begin{lstlisting}[label=lst:hadoopAppDetailsCmd,style=plain,
caption={[\glsentryshort{CLI}"=Ausgabe des Reports einer Anwendung]
    \acrshort{CLI}"=Ausgabe des Reports einer Anwendung}]
$ yarn application --status application_1518100641776_0001
...
Application Report : 
    Application-Id : application_1518100641776_0001
    Application-Name : QuasiMonteCarlo
    Application-Type : MAPREDUCE
    User : root
    Queue : default
    Start-Time : 1518103712160
    Finish-Time : 1518103799743
    Progress : 100%
    State : FINISHED
    Final-State : SUCCEEDED
    Tracking-URL : http://controller:19888/jobhistory/job/job_1518100641776_0001
    RPC Port : 41309
    AM Host : compute-1
    Aggregate Resource Allocation : 1075936 MB-seconds, 942 vcore-seconds
    Diagnostics :
\end{lstlisting}

\begin{lstlisting}[label=lst:hadoopAppStart,style=plain,
caption={[Starten einer Anwendung in Hadoop"=Benchmark]
    Starten einer Anwendung in Hadoop"=Benchmark.
    Hier mit dem Mapreduce Example \acrlong{pi} und dem Abbruch der Anwendung durch den in \cref{lst:hadoopAppKill} gezeigten Befehl.
    Die Anwendungs"=ID \mbox{\texttt{application\_1520342317799\_0002}} ist hier in Zeile 13 enthalten.}]
$ hadoop-benchmark/benchmarks/hadoop-mapreduce-examples/run.sh pi 20 1000
Number of Maps  = 20
Samples per Map = 1000
Wrote input for Map #0
...
Starting Job
18/03/14 13:06:26 INFO impl.TimelineClientImpl: Timeline service address: http://0.0.0.0:8188/ws/v1/timeline/
18/03/14 13:06:27 INFO client.RMProxy: Connecting to ResourceManager at controller/10.0.0.3:8032
18/03/14 13:06:27 INFO client.AHSProxy: Connecting to Application History server at /0.0.0.0:10200
18/03/14 13:06:27 INFO input.FileInputFormat: Total input paths to process : 20
18/03/14 13:06:27 INFO mapreduce.JobSubmitter: number of splits:20
18/03/14 13:06:27 INFO mapreduce.JobSubmitter: Submitting tokens for job: job_1520342317799_0002
18/03/14 13:06:28 INFO impl.YarnClientImpl: Submitted application application_1520342317799_0002
18/03/14 13:06:28 INFO mapreduce.Job: The url to track the job: http://controller:8088/proxy/application_1520342317799_0002/
18/03/14 13:06:28 INFO mapreduce.Job: Running job: job_1520342317799_0002
18/03/14 13:06:34 INFO mapreduce.Job: Job job_1520342317799_0002 running in uber mode : false
18/03/14 13:06:34 INFO mapreduce.Job:  map 0% reduce 0%
18/03/14 13:06:58 INFO mapreduce.Job:  map 20% reduce 0%
18/03/14 13:06:59 INFO mapreduce.Job:  map 60% reduce 0%
18/03/14 13:07:03 INFO mapreduce.Job:  map 0% reduce 0%
18/03/14 13:07:03 INFO mapreduce.Job: Job job_1520342317799_0002 failed with state KILLED due to: Application killed by user.
18/03/14 13:07:03 INFO mapreduce.Job: Counters: 0
Job Finished in 37.53 seconds
\end{lstlisting}

\begin{lstlisting}[label=lst:hadoopAppKill,style=plain,
caption={[Vorzeitiges Beenden einer Anwendung]
    Vorzeitiges Beenden einer Anwendung.
    Hier wird die in \cref{lst:hadoopAppStart} gestartete Anwendung vorzeitig beendet.}]
$ yarn application -kill application_1520342317799_0002
...
Killing application application_1520342317799_0002
18/03/14 13:07:02 INFO impl.YarnClientImpl: Killed application application_1520342317799_0002
\end{lstlisting}
