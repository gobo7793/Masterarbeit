\section{Model Checking}\label{sec:modelchecking}

\emph{Model Checking} (MC) ist eine Möglichkeit, um Systeme zu testen und zu verifizieren. Dazu werden vom \emph{Model Checker} (MC) alle möglichen Systemzustände in einem \emph{brute-force}-ähnlichem Vorgehen getestet und somit alle möglichen Szenarien getestet. Die Anzahl der Zustände kann sehr schnell $ 10^{120} $ oder mehr betragen \cite{Grumberg1999,Baier2008}.

%TODO: schematisches vorgehen als bild

Ein MC nutzt, wie der Name schon sagt, ein Modell des Systems, um das System zu testen. Wie bei jeder anderen modellbasierten Technik ist daher die Qualität des MC nur so gut wie das darauf zugrunde liegende Modell. Ein Modell kann auch als endlicher Automat angesehen werden, da ein Modell ebenfalls eine endliche Anzahl an möglichen Zuständen und dazugehörige Übergänge besitzt. Für jede Eigenschaft eines Zustandes muss zudem mithilfe einer sog. \emph{temporalen Logik}, also mathematisch bzw. formal, festgelegt werden, was gültige Werte dieser Eigenschaft sind. Die dazu benötigten Informationen werden aus den Anforderungen des Systems ermittelt und dem MC übergeben. So können später verschiedene Eigenschaften des gesamten Systems (\zB die formale Korrektheit, die Ausführbarkeit ohne Deadlocks oder die Einhaltung von Sicherheitsvorgaben) geprüft werden.

Zur Ausführung wird das gesamte Modell zunächst initialisiert und dann automatisch und systematisch vom MC geprüft.

