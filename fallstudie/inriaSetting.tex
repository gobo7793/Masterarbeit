\section{Adaptive Komponente in Hadoop}\label{sec:inriaSetting}

% Aufbau vom INRIA-Setting erklären
% Vorteile der adaptiven Komponente

Ein normales Hadoop besitzt von sich aus keine adaptive Komponente, sondern rein statische Einstellungen. Um damit Hadoop zu optimieren, müssen die Einstellungen immer manuell auf den jeweils benötigten Anwendungstyp angepasst werden. Dazu gibt es auch bereits verschiedene Scheduler, den \emph{Fair Scheduler}, welcher alle Anwendungen ausführt und ihnen gleich viele Ressourcen zuteilt, und den \emph{Capacity Scheduler}. Letzterer sorgt dafür, dass nur eine bestimmte Anzahl an Anwendungen pro Benutzter gleichzeitig ausgeführt wird und teilt ihnen so viele Ressourcen zu, wie benötigt werden bzw. der Benutzter nutzen darf. Entwickelt wurde der Capacity Scheduler vor allem für Cluster, die von mehreren Organisationen gemeinsam verwendet werden und sicherstellen soll, dass jede Organisation eine Mindestmenge an Ressourcen zur Verfügung hat \cite{HadoopCapScheduler271}.

Je nach Bedarf besitzt der Capacity Scheduler entsprechende Einstellungen, um \zB den verfügbaren Speicher pro Container festzulegen, auch \emph{MARP} genannt. Da diese Einstellung direkt beeinflusst, wie viele Container und damit Anwendungen gleichzeitig ausgeführt werden können, ist das gesamte Cluster je nach Wert eher für kleine oder eher große Anwendungen effizient. Da der MARP-Wert jedoch nicht während der Laufzeit dynamisch angepasst werden kann, haben \citeauthor{zhang2016} in \cite{zhang2016} einen Ansatz zur dynamischen Anpassung des MARP-Wertes zur Laufzeit von Hadoop vorgestellt. Dadurch wird der MARP-Wert abhängig von den ausgeführten Anwendungen adaptiv zur Laufzeit angepasst, sodass immer möglichst viele Anwendungen gleichzeitig ausgeführt werden können.



, welche genutzt werden soll. Im Vergleich zur Standard-Einstellung von Hadoop benötigt diese mit einer selbstadaptiven Komponente ausgestatte Modifikation im Schnitt um bis zu 40 Prozent weniger Zeit zur Ausführung eines Tasks. Dazu wird der zur Verfügung stehende Arbeitsspeicher zur Laufzeit so eingeteilt, damit immer die maximal mögliche Anzahl an Tasks ausgeführt werden können.