\section{Grundlegender Versuchsaufbau}
\label{sec:clusterSetup}

Neben den Anforderungen an Hadoop und das gesamte Testsystem muss auch der grundlegende Versuchsaufbau in dieser Fallstudie definiert werden.
Im Grunde wird, wie bereits in \autoref{chap:intro} erwähnt, Hadoop mithilfe des \ac{ss}"=Frameworks nachgebildet und dieses Modell mit einem realen Cluster verbunden.
\todo{dort erwähnen}
In diesem Cluster sollen anhand des Modells unterschiedliche Komponentenfehler injiziert und repariert werden, als auch unterschiedliche Benchmarks gestartet werden.
Hierbei soll nicht nur das Verhalten von Hadoop selbst analysiert werden, sondern auch das der von \citeauthor{zhang2016} entwickelten Selfbalancing"=Komponente.
Anhand dieses Verhaltens und dem des kompletten Testsystems soll schließlich ermittelt werden, ob eine Testautomatisierung in diesem Versuchsaufbau erfolgreich war.

Bei der Entwicklung des Modells liegt der Fokus auf dem grundlegenden Aufbau von \ac{YARN}.
Dazu gehören die Anwendungen und ihre Attempts, sowie zum Teil auch ihre Container.
Daneben muss das Modell auch die Nodes des Clusters und zum Ausführen der Benchmarks auch simulierte Clients enthalten.
Da bei den Tests auch Ausfälle von Nodes eine Rolle spielen, müssen hierfür entsprechende Komponentenfehler implementiert werden, die mithilfe von \ac{ss} aktiviert und deaktiviert werden können.

Da die Auswahl der ausgeführten Benchmarks eines jeden Clients nicht bei jedem Test manuell bestimmt werden soll, wird hierfür ein Transitionssystem verwendet.
Mithilfe dieses Transitionssystems, in dem die Wahrscheinlichkeiten von Wechseln zwischen zwei Anwendungen definiert sind, soll während der Ausführung eines Testfalls zufällig eine nachfolgende Anwendung ausgewählt werden.
%Da zum Testen des Clusters der \ac{ss}"=Simulator eingesetzt wird, hängt die Anzahl der Anwendungen primär von der Anzahl der ausgeführten Simulations"=Schritte ab.
%Ein weiterer Faktor zur Anzahl der Anwendungen ist die Anzahl an simulierten Clients, da auch getestet werden soll, wie sich das Cluster bei der Ausführung von mehreren parallel gestarteten Anwendungen verhält.

Die Verbindung zwischen dem Modell und dem realen Hadoop"=Cluster wird mithilfe eines dafür entwickelten Treibers durchgeführt.
Der Treiber ist dafür verantwortlich, Komponentenfehler und Anwendungen an das reale Cluster zu senden.
Zudem dient er dazu, um den Status des Clusters jederzeit ermitteln und an das Modell zur dortigen Speicherung übergeben zu können.
Er kann daher nicht nur aus der Verbindung zum Cluster selbst bestehen, sondern muss auch die Kommunikation zwischen Modell und Cluster sicherstellen und übermittelte Daten entsprechend umwandeln.

Zur Umsetzung des realen Clusters wird die von \citeauthor{zhang2016} entwickelte Plattform Hadoop"=Benchmark mit für diese Fallstudie entwickelten Szenarios genutzt.
Als Basis dient hierzu das bereits in der Plattform enthaltene Szenario mit der Nutzung der Selfbalancing"=Komponente.
Zudem soll auch mithilfe von Mutationstests, bei denen einer oder mehrere Mutanten in der Selfbalancing"=Komponente implementiert werden, das Testsystem geprüft werden.

Dieser Versuchsaufbau soll zudem mithilfe eines dafür entwickelten \emph{Oracles} geprüft werden.
Das Oracle dient zur Validierung der in \autoref{sec:requirements} definierten Anforderungen an das Cluster und das Testsystem.
Hierfür werden, sofern möglich, die Anforderungen als \emph{Constraints} im Modell implementiert und bei jedem Test automatisch geprüft.

Die Implementierung des eben beschriebenen Modells und Oracles ist im \autoref{chap:modell} beschrieben.
Die Auswahl der verwendeten Benchmarks und deren Implementierung mit dem Transitionssystem findet sich in \autoref{chap:benchmarks}.
