% allgem. Dokumentenformat
\documentclass[a4paper,12pt,headsepline]{report}
%Variablen welche innerhalb der gesamten Arbeit zur Verfügung stehen sollen
\newcommand{\titleDocument}{Masterarbeit}
\newcommand{\subjectDocument}{im Studiengang Informatik}
\newcommand{\zB}{z.\,B. }
\newcommand{\uA}{u.\,A. }
\newcommand{\sS}{S\# }
\newcommand{\cS}{C\# }
\newcommand{\uU}{unter Umständen}

% weitere Pakete
% Grafiken aus PNG Dateien einbinden
\usepackage{graphicx}

% Deutsche Sonderzeichen benutzen 
\usepackage{ngerman}

% deutsche Silbentrennung
\usepackage[ngerman]{babel}

% Eurozeichen einbinden
\usepackage[right]{eurosym}

% Umlaute unter UTF8 nutzen
\usepackage[utf8]{inputenc}

% Zeichenencoding
\usepackage[T1]{fontenc}

\usepackage{lmodern}
\usepackage{fix-cm}

% floatende Bilder ermöglichen
%\usepackage{floatflt}

% mehrseitige Tabellen ermöglichen
\usepackage{longtable}
\usepackage{multirow}
\usepackage{tabularx}
\usepackage{enumitem}

% Unterstützung für Schriftarten
%\newcommand{\changefont}[3]{ 
%\fontfamily{#1} \fontseries{#2} \fontshape{#3} \selectfont}

% Packet für Seitenrandabständex und Einstellung für Seitenränder
\usepackage{geometry}
\geometry{left=3.5cm, right=2cm, top=2.5cm, bottom=2cm}

% Paket für Boxen im Text
\usepackage{fancybox}

% bricht lange URLs "schoen" um
\usepackage[hyphens,obeyspaces,spaces]{url}

% Paket für Textfarben
\usepackage{color}

% Mathematische Symbole importieren
\usepackage{amssymb}

% auf jeder Seite eine Überschrift (alt, zentriert)
%\pagestyle{headings}

% erzeugt Inhaltsverzeichnis mit Querverweisen zu den Kapiteln (PDF Version)
\usepackage[bookmarksnumbered,pdftitle={\titleDocument},hyperfootnotes=false]{hyperref} 
%\hypersetup{colorlinks, citecolor=red, linkcolor=blue, urlcolor=black}
%\hypersetup{colorlinks, citecolor=black, linkcolor= black, urlcolor=black}

% Linkziele oberhalb von Abbildungen und Tabellen
\usepackage{caption}

% neue Kopfzeilen mit fancypaket
\usepackage{fancyhdr} %Paket laden
\pagestyle{fancy} %eigener Seitenstil
\fancyhf{} %alle Kopf- und Fußzeilenfelder bereinigen
\fancyhead[L]{\nouppercase{\leftmark}} %Kopfzeile links
\fancyhead[C]{} %zentrierte Kopfzeile
\fancyhead[R]{\thepage} %Kopfzeile rechts
\renewcommand{\headrulewidth}{0.4pt} %obere Trennlinie
%\fancyfoot[C]{\thepage} %Seitennummer
%\renewcommand{\footrulewidth}{0.4pt} %untere Trennlinie

% für Tabellen
\usepackage{array}

% Runde Klammern für Zitate
\usepackage[numbers,round]{natbib}

% Festlegung Art der Zitierung
\bibliographystyle{abbrvdin}

% Schaltet den zusätzlichen Zwischenraum ab, den LaTeX normalerweise nach einem Satzzeichen einfügt.
\frenchspacing

% Paket für Zeilenabstand
\usepackage{setspace}

% für Bildbezeichner
\usepackage{capt-of}

% für Stichwortverzeichnis
\usepackage{makeidx}

% für Listings
\usepackage{listings}
\lstset{numbers=left, numberstyle=\tiny, numbersep=5pt, keywordstyle=\color{black}\bfseries, stringstyle=\ttfamily,showstringspaces=false,basicstyle=\footnotesize,captionpos=b}
\lstset{language= [sharp]c}

% Indexerstellung
\makeindex

% Abkürzungsverzeichnis
%\usepackage[german]{nomencl}
%\let\abbrev\nomenclature

% Abkürzungsverzeichnis LiveTex Version
%\renewcommand{\nomname}{Abkürzungsverzeichnis}
%\setlength{\nomlabelwidth}{.25\hsize}
%\renewcommand{\nomlabel}[1]{#1 \dotfill}
%\setlength{\nomitemsep}{-\parsep}
%\makenomenclature
%\makeglossary

% Abkürzungsverzeichnis TeTEX Version
% \usepackage[german]{nomencl}
% \makenomenclature
% %\makeglossary
% \renewcommand{\nomname}{Abkürzungsverzeichnis}
% \setlength{\nomlabelwidth}{.25\hsize}
% \renewcommand{\nomlabel}[1]{#1 \dotfill}
% \setlength{\nomitemsep}{-\parsep}

% Disable single lines at the start of a paragraph (Schusterjungen)
\clubpenalty = 10000
% Disable single lines at the end of a paragraph (Hurenkinder)
\widowpenalty = 10000
\displaywidowpenalty = 10000

\begin{document}
% hier werden die Trennvorschläge inkludiert
%hier müssen alle Wörter rein, welche Latex von sich auch nicht korrekt trennt bzw. bei denen man die genaue Trennung vorgeben möchte
\hyphenation
{
    Film-pro-du-zen-ten
    Lux-em-burg
    Soft-ware-bau-steins
    zeit-in-ten-siv
}
\renewcommand{\texttt}[1]{%
    \begingroup
    \ttfamily
    \begingroup\lccode`~=`/\lowercase{\endgroup\def~}{/\discretionary{}{}{}}%
    \begingroup\lccode`~=`[\lowercase{\endgroup\def~}{[\discretionary{}{}{}}%
    \begingroup\lccode`~=`.\lowercase{\endgroup\def~}{.\discretionary{}{}{}}%
    \catcode`/=\active\catcode`[=\active\catcode`.=\active
    \scantokens{#1\noexpand}%
    \endgroup
}

%Schriftart Helvetica
%\changefont{phv}{m}{n}

% Titelseite %
\begin{center}
    {\Large{Universität Augsburg\\Fakultät für Angewandte Informatik}}
    \vspace{4\baselineskip}
    
    \begin{onehalfspace}
        \textbf{\large{Modellbasierte Testautomatisierung eines\\verteilten, adaptiven Load-Balancing-Systems}}
    \end{onehalfspace}
    \vspace{3\baselineskip}
    
    \textbf{{\Large{Masterarbeit}}}
    \vspace{1\baselineskip}
    
    \textbf{im Studiengang Informatik}
    \vspace{1\baselineskip}
    
    \textbf{zur Erlangung des akademischen Grades\\Master of Science}
    \vspace{1\baselineskip}
    
    \textbf{von\\Gerald Siegert}
    \vspace{\fill}
    
    \begin{singlespace}
        \begin{tabular}{llll}
            \textbf{Mat.-Nr.:}  &  & 1450117                      &  \\
            &  &  \\
            \textbf{Datum:}     &  & \today                       &  \\
            &  &  \\
            \textbf{Betreuer:}  &  & M.Sc. Benedikt Eberhardinger &  \\
            \textbf{1. Prüfer:} &  & Prof. Dr. X                  &  \\
            \textbf{2. Prüfer:} &  & Prof. Dr. Y                  &
        \end{tabular}
    \end{singlespace}
\end{center}


% Eidesstattliche Erklärung
%\addcontentsline{toc}{section}{Eidesstattliche Erklärung}
\thispagestyle{empty}

\begin{verbatim}

\end{verbatim}

\chapter*{Eidesstattliche Erklärung}

\begin{verbatim}

\end{verbatim}

Ich versichere, die von mir vorgelegte Arbeit selbstständig verfasst zu haben. Alle Stellen, die wörtlich oder sinngemäß aus veröffentlichten oder nicht veröffentlichten Arbeiten anderer entnommen sind, habe ich als entnommen kenntlich gemacht. Sämtliche Quellen und Hilfsmittel, die ich für die Arbeit benutzt habe, sind angegeben. Die Arbeit hat mit gleichem Inhalt bzw. in wesentlichen Teilen noch keiner anderen Prüfungsbehörde vorgelegen.

\begin{verbatim}

\end{verbatim}

Ort, Datum:~~~~~~~~~~~~~~~~~~~~~~~~~~~~~~~~~~~~~~~~~~
Unterschrift:~~~~~~~~~~~~~~~~~~~~~~~~~~~~~~~~~~~~~~~~~~


% römische Numerierung
\pagenumbering{Roman}

% 1.5 facher Zeilenabstand
\onehalfspacing

% Sperrvermerk
%\input{sperrvermerk}

% Einleitung / Abstract
\begin{abstract}
    Durch eine Automatisierung von Tests lassen sich im Bereich der Softwareentwicklung hohe Kosten einsparen.
    Daher wurden zahlreiche Test"=Frameworks und Möglichkeiten zum Testen von Systemen und ihrer Software entwickelt.
    Ein solches Framework ist \acrshort{ss} (\acrlong{ss}), mit dem mithilfe eines modellbasierten Ansatzes Systeme getestet werden können.
    Mithilfe des \acrshort{ss}"=Frameworks soll nun ein Testsystem entwickelt werden, um hiermit automatisiert ein verteiltes, adaptives Load"=Balancing"=System zu testen.
    Hierfür wurde Apache Hadoop ausgewählt, welches mit einer selbstadaptiven Komponente ergänzt wird.
    Diese selbstadaptive Komponente verändert dynamisch und basierend auf den derzeit auf dem Hadoop"=Cluster ausgeführten Anwendungen einige der sonst statischen Einstellungen von Hadoop, womit die verfügbaren Ressourcen des Clusters optimaler genutzt werden können.
    
    Um Hadoop testen zu können, wurde zunächst mithilfe von \acrshort{ss} ein Modell entwickelt, welches die wesentlichen Komponenten des YARN=Frameworks von Hadoop abbildet.
    Dieses Modell wurde wiederum mithilfe eines hierfür entwickelten Treibers mit einem realen Hadoop"=Cluster verbunden.
    Dadurch wurde es ermöglicht, durch die Testausführung mit \acrshort{ss} unterschiedliche Anwendungen auf dem realen Cluster auszuführen und die Daten der Anwendungen und des Clusters im Modell zu nutzen.
    Um zu testen, ob sich das entwickelte Testsystem zur Testautomatisierung eines verteilten, adaptiven Load"=Balancing"=Systems eignet, wurde hierfür eine Fallstudie durchgeführt.
    
    In dieser Masterarbeit werden der Aufbau und Ablauf der durchgeführten Fallstudie, sowie die Entwicklung und Implementierung des hierfür genutzten Testsystems erläutert.
    Es wird gezeigt, welche Besonderheiten bei der Durchführung und Auswertung der Fallstudie aufgetreten sind, und inwiefern sich das entwickelte, modellbasierte Testsystem zur Testautomatisierung eines verteilten, adaptiven Load"=Balancing"=Systems eignet.
\end{abstract}

\clearpage
\begin{otherlanguage}{english}
\begin{abstract}
    By automating tests, high costs can be saved in software development.
    Therefore, numerous test frameworks and ways to test systems and their software have been developed.
    One such framework is \acrshort{ss} (\acrlong{ss}), which uses a model-based approach to test systems.
    By using the \acrshort{ss} framework, a test system will be developed to automatically test a distributed, adaptive load-balancing system.
    For this, Apache Hadoop was chosen, which is equipped with an adaptive resource manager.
    The manager detect the current usage of the cluster and modify some of Hadoop's otherwise static settings to make a better use of the available resources.
    
    To test hadoop, a \acrshort{ss} model was developed, which contains the essential componentens of the Hadoop YARN framework.
    To connect the model to a real Hadoop cluster a driver was developed for this purpose.
    This allows \acrshort{ss} to run different applications on the real cluster and detect the state of the running applications and the cluster.
    To determine the developed test system is suitable for the test automation of a distributed, adaptive load-balancing system, a case study was performed.
    
    This master thesis explains the structure and processes of the case study, as well as the development and implementation of the test system used for this purpose.
    It shows the won experiences by performing the case study and shows how the developed, model-based test system is suitable for test automation of a distributed, adaptive load-balancing system.
\end{abstract}
\end{otherlanguage}



% einfacher Zeilenabstand
\singlespacing

% Inhaltsverzeichnis anzeigen
\newpage
\tableofcontents

% das Abbildungsverzeichnis
\newpage
% Abbildungsverzeichnis soll im Inhaltsverzeichnis auftauchen
\addcontentsline{toc}{chapter}{Abbildungsverzeichnis}
% Abbildungsverzeichnis endgueltig anzeigen
\listoffigures

% das Tabellenverzeichnis
%\newpage
% Abbildungsverzeichnis soll im Inhaltsverzeichnis auftauchen
%\addcontentsline{toc}{section}{Tabellenverzeichnis}
%\fancyhead[L]{Abbildungsverzeichnis / Abkürzungsverzeichnis} %Kopfzeile links
% Abbildungsverzeichnis endgueltig anzeigen
%\listoftables

%% WORKAROUND für Listings
%\makeatletter% --> De-TeX-FAQ
%\renewcommand*{\lstlistoflistings}{%
%  \begingroup
%    \if@twocolumn
%      \@restonecoltrue\onecolumn
%    \else
%      \@restonecolfalse
%    \fi
%    \lol@heading
%    \setlength{\parskip}{\z@}%
%    \setlength{\parindent}{\z@}%
%    \setlength{\parfillskip}{\z@ \@plus 1fil}%
%    \@starttoc{lol}%
%    \if@restonecol\twocolumn\fi
%  \endgroup
%}
%\makeatother% --> \makeatletter
% das Listingverzeichnis
\newpage
% Listingverzeichnis soll im Inhaltsverzeichnis auftauchen
\addcontentsline{toc}{chapter}{Listingverzeichnis}
\fancyhead[L]{Listingverzeichnis} %Kopfzeile links
\renewcommand{\lstlistlistingname}{Listingverzeichnis}
\lstlistoflistings
%%%%

% das Abkürzungsverzeichnis
%\newpage
% Abkürzungsverzeichnis soll im Inhaltsverzeichnis auftauchen
%\addcontentsline{toc}{section}{Abkürzungsverzeichnis}
% das Abkürzungsverzeichnis entgültige Ausgeben
%\fancyhead[L]{Abkürzungsverzeichnis} %Kopfzeile links
%\chapter*{Abkürzungsverzeichnis}

\begin{acronym}
%    \acro{Kuerzel}[Kurzform]{Langform}
%    \acroplural{Kuerzel}[Kurzform des Plurals]{Langform des Plurals}
    \acro{MC}{Model Checker}
    \acro{MCr}[MC]{Model Checker}
    \acro{DCCA}{Deductive Cause-Consequence Analysis}
    \acro{RM}{ResourceManager}
    \acro{AM}{ApplicationManager}
    \acro{AMstr}[AppMstr]{ApplicationMaster}
\end{acronym}
%\printnomenclature

% Definiert Stegbreite bei zweispaltigem Layout
\setlength{\columnsep}{25pt}

%%%%%%% EINLEITUNG %%%%%%%%%%%%
%\twocolumn
\newpage
\fancyhead[L]{\nouppercase{\leftmark}} %Kopfzeile links
\pagenumbering{arabic}

% 1,5 facher Zeilenabstand
\onehalfspacing

% einzelne Kapitel
\chapter{Einleitung}
\label{chap:intro}

Im Bereich der Softwaretests wird heutzutage sehr viel mit automatisierten Testverfahren gearbeitet.
Dies ist insofern logisch, als dass diese Testautomatisierung einerseits Aufwand und damit andererseits direkt Kosten einer Software einspart.
Daher gibt es vor allem im Bereich der Komponententests zahlreiche Frameworks, mit denen Tests einfach und automatisiert erstellt bzw. ausgeführt werden können.
Ein Beispiel für ein solches Testframework wäre das \emph{xUnit}"=Framework, zu dem \uA JUnit\footnote{\url{https://junit.org}} für Java und NUnit\footnote{\url{https://nunit.org/}} für .NET zählen.
Dabei werden zunächst einzelne Testfälle erstellt und können im Anschluss mit der jeweils aktuellen Codebasis jederzeit ausgeführt werden.
Automatisierte Tests können auch dazu genutzt werden, um einen einzelnen Test mit verschiedenen Eingaben durchzuführen.
Dadurch können verschiedene Eingabeklassen (wie negative oder positive Ganzzahlen) mit sehr geringem Aufwand in einem Test genutzt werden und somit verschiedene Testfälle direkt ausgeführt werden, wodurch eine massive Kosteneinsparung einhergeht \cite{Polo2013}.

Es gibt aber nicht nur Frameworks für Komponententests, sondern auch für modellbasierte Testverfahren wie \zB dem \ac{MC}.
Beim \ac{MC} wird ein Modell mithilfe eines entsprechenden Frameworks automatisiert auf seine Spezifikation getestet und geprüft, unter welchen Umständen diese verletzt wird \cite{Grumberg1999,Habermaier2015}.

In dieser Masterarbeit soll daher nun ein verteiltes, adaptives Load"=Balancing"=System getestet werden.
Hauptziel ist es, zu ermitteln, wie ein modellbasierter Testansatz auf ein komplexes Beispiel übertragen werden kann.
Dafür wird zunächst ein reales System als vereinfachtes Modell nachgebildet und anschließend mithilfe eines \ac{MC} getestet.
Es soll dabei auch ermittelt werden, wie ein reales System in das Modell eingebunden werden kann und wie bei Problemen mit asynchronen Prozessen innerhalb des verteilten Systems umgegangen werden muss.


\onecolumn
% einfacher Zeilenabstand
\singlespacing
% Literaturliste soll im Inhaltsverzeichnis auftauchen
\newpage
\addcontentsline{toc}{chapter}{Literaturverzeichnis}
% Literaturverzeichnis anzeigen
\renewcommand\refname{Literaturverzeichnis}
\bibliography{literatur, abbildungen}

%% Index soll Stichwortverzeichnis heissen
% \newpage
% % Stichwortverzeichnis soll im Inhaltsverzeichnis auftauchen
% \addcontentsline{toc}{section}{Stichwortverzeichnis}
% \renewcommand{\indexname}{Stichwortverzeichnis}
% % Stichwortverzeichnis endgueltig anzeigen
% \printindex

\onehalfspacing
% evtl. Anhang
%\newpage
%\addcontentsline{toc}{section}{Anhang}
%\fancyhead[L]{Anhang} %Kopfzeile links
%%\chapter*{Anhang}\label{chap:anhang}
%\addcontentsline{toc}{chapter}{Anhang}
%\fancyhead[L]{Anhang}
%\renewcommand{\thesection}{\Alph{section}}

\chapter{\glsentryshort{CLI}"=Befehle von Hadoop}
\label{app:hadoopCmds}

Für jede der vier relevanten YARN"=Komponenten können die Daten jeweils als Liste oder als ausführlicher Report ausgegeben werden.
Im Folgenden sind beispielhaft die dafür notwendigen Befehle für Anwendungen aufgelistet, für Attempts, Container und Nodes sind analoge Befehle verfügbar.
Neben den Monitoring"=Befehlen sind auch einige weitere für diese Arbeit relevante Befehle mit ihren Ausgaben aufgelistet.
Die Ausgaben zu den Befehlen sind hier zudem auf das wesentliche gekürzt, \uA da Hadoop bei einigen Befehlen ausgibt, über welche Services (in \cref{lst:hadoopAppListCmd} \zB \gls{TLS}, \gls{RM} und \emph{Application History Server}) die Daten ermittelt werden.
Weiterführende Informationen zu den hier aufgeführten Befehlen sowie die vollständige Befehlsreferenz sind in der Dokumentation von Hadoop \cite{HadoopYarnCmds271} zu finden.

\begin{lstlisting}[label=lst:hadoopAppListCmd,style=plain,
caption={[\glsentryshort{CLI}"=Ausgabe der Anwendungsliste]
    \acrshort{CLI}"=Ausgabe der Anwendungsliste.
    Anwendungen können mithilfe der Optionen \mbox{\texttt{-{}-appTypes}} und \mbox{\texttt{-{}-appStates}} gefiltert werden.}]
$ yarn application --list --appStates ALL
18/02/08 15:37:51 INFO impl.TimelineClientImpl: Timeline service address: http://0.0.0.0:8188/ws/v1/timeline/
18/02/08 15:37:51 INFO client.RMProxy: Connecting to ResourceManager at controller/10.0.0.3:8032
18/02/08 15:37:51 INFO client.AHSProxy: Connecting to Application History server at /0.0.0.0:10200
Total number of applications (application-types: [] and states: [NEW, NEW_SAVING, SUBMITTED, ACCEPTED, RUNNING, FINISHED, FAILED, KILLED]):1
Application-Id	Application-Name	Application-Type	User	Queue	State	Final-State	Progress	Tracking-URL
application_1518100641776_0001	QuasiMonteCarlo	MAPREDUCE	root	default	FINISHED	SUCCEEDED	100%	http://controller:19888/jobhistory/job/job_1518100641776_0001
\end{lstlisting}

\begin{lstlisting}[label=lst:hadoopAppDetailsCmd,style=plain,
caption={[\glsentryshort{CLI}"=Ausgabe des Reports einer Anwendung]
    \acrshort{CLI}"=Ausgabe des Reports einer Anwendung}]
$ yarn application --status application_1518100641776_0001
...
Application Report : 
    Application-Id : application_1518100641776_0001
    Application-Name : QuasiMonteCarlo
    Application-Type : MAPREDUCE
    User : root
    Queue : default
    Start-Time : 1518103712160
    Finish-Time : 1518103799743
    Progress : 100%
    State : FINISHED
    Final-State : SUCCEEDED
    Tracking-URL : http://controller:19888/jobhistory/job/job_1518100641776_0001
    RPC Port : 41309
    AM Host : compute-1
    Aggregate Resource Allocation : 1075936 MB-seconds, 942 vcore-seconds
    Diagnostics :
\end{lstlisting}

\begin{lstlisting}[label=lst:hadoopAppStart,style=plain,
caption={[Starten einer Anwendung in Hadoop"=Benchmark]
    Starten einer Anwendung in Hadoop"=Benchmark.
    Hier mit dem Mapreduce Example \acrlong{pi} und dem Abbruch der Anwendung durch den in \cref{lst:hadoopAppKill} gezeigten Befehl.
    Die Anwendungs"=ID \mbox{\texttt{application\_1520342317799\_0002}} ist hier in Zeile 13 enthalten.}]
$ hadoop-benchmark/benchmarks/hadoop-mapreduce-examples/run.sh pi 20 1000
Number of Maps  = 20
Samples per Map = 1000
Wrote input for Map #0
...
Starting Job
18/03/14 13:06:26 INFO impl.TimelineClientImpl: Timeline service address: http://0.0.0.0:8188/ws/v1/timeline/
18/03/14 13:06:27 INFO client.RMProxy: Connecting to ResourceManager at controller/10.0.0.3:8032
18/03/14 13:06:27 INFO client.AHSProxy: Connecting to Application History server at /0.0.0.0:10200
18/03/14 13:06:27 INFO input.FileInputFormat: Total input paths to process : 20
18/03/14 13:06:27 INFO mapreduce.JobSubmitter: number of splits:20
18/03/14 13:06:27 INFO mapreduce.JobSubmitter: Submitting tokens for job: job_1520342317799_0002
18/03/14 13:06:28 INFO impl.YarnClientImpl: Submitted application application_1520342317799_0002
18/03/14 13:06:28 INFO mapreduce.Job: The url to track the job: http://controller:8088/proxy/application_1520342317799_0002/
18/03/14 13:06:28 INFO mapreduce.Job: Running job: job_1520342317799_0002
18/03/14 13:06:34 INFO mapreduce.Job: Job job_1520342317799_0002 running in uber mode : false
18/03/14 13:06:34 INFO mapreduce.Job:  map 0% reduce 0%
18/03/14 13:06:58 INFO mapreduce.Job:  map 20% reduce 0%
18/03/14 13:06:59 INFO mapreduce.Job:  map 60% reduce 0%
18/03/14 13:07:03 INFO mapreduce.Job:  map 0% reduce 0%
18/03/14 13:07:03 INFO mapreduce.Job: Job job_1520342317799_0002 failed with state KILLED due to: Application killed by user.
18/03/14 13:07:03 INFO mapreduce.Job: Counters: 0
Job Finished in 37.53 seconds
\end{lstlisting}

\begin{lstlisting}[label=lst:hadoopAppKill,style=plain,
caption={[Vorzeitiges Beenden einer Anwendung]
    Vorzeitiges Beenden einer Anwendung.
    Hier wird die in \cref{lst:hadoopAppStart} gestartete Anwendung vorzeitig beendet.}]
$ yarn application -kill application_1520342317799_0002
...
Killing application application_1520342317799_0002
18/03/14 13:07:02 INFO impl.YarnClientImpl: Killed application application_1520342317799_0002
\end{lstlisting}


\chapter{REST"=API von Hadoop}
\label{app:hadoopRestApi}

Wie bei der Ausgabe der Daten der YARN"=Komponenten mithilfe der \gls{CLI} können auch bei der Ausgabe mithilfe der \gls{REST}"=API die Daten als Liste oder als einzelner Report ausgegeben werden.
Der Unterschied zur \gls{CLI} liegt jedoch darin, dass in Listenform und als einzelner Report immer die vollständigen Objekte der Komponenten zurückgegeben werden.
Neben der hier gezeigten und auch in der Fallstudie genutzten Ausgabe im JSON"=Format unterstützt Hadoop auch eine Ausgabe im XML"=Format.
Im Folgenden sind daher beispielhaft die Ausgaben im JSON"=Format für die Anwendungsliste vom \gls{RM} und für Ausführungen vom \gls{TLS} aufgeführt.
Im Rahmen dieser Masterarbeit sind die Rückgaben für Listen von Anwendungen, Attempts, \gls{Container} und der Nodes vom \gls{RM} und bzw. \gls{NM} (Container) sowie des \gls{TLS} (Attempts und Container)relevant.
Weitere Informationen zur \gls{REST}"=API sowie hier nicht gezeigte Pfade für die \gls{YARN}"=Komponenten sind in der Dokumentation in \cite{HadoopYarnTlServer271,HadoopRmApi271,HadoopNmApi271} zu finden.

\begin{lstlisting}[label=lst:hadoopAppListRestRm,style=json,
caption={[REST"=-Ausgabe aller \glspl{Anwendung} vom \acrshort{RM}]
    \gls{REST}"=Ausgabe aller \glspl{Anwendung} vom \acrshort{RM}.
    Die Liste kann mithilfe verschiedener Query"=Parameter gefiltert werden.\\
    URL: \url{http://addr:port/ws/v1/cluster/apps}}]
{
  "apps": {
    "app": [
      {
        "id": "application_1518429920717_0001",
        "user": "root",
        "name": "QuasiMonteCarlo",
        "queue": "default",
        "state": "FINISHED",
        "finalStatus": "SUCCEEDED",
        "progress": 100,
        "trackingUI": "History",
        "trackingUrl": "http://controller:8088/proxy/application_1518429920717_0001/",
        "diagnostics": "",
        "clusterId": 1518429920717,
        "applicationType": "MAPREDUCE",
        "applicationTags": "",
        "startedTime": 1518430260179,
        "finishedTime": 1518430404123,
        "elapsedTime": 143944,
        "amContainerLogs": "http://compute-2:8042/node/containerlogs/container_1518429920717_0001_01_000001/root",
        "amHostHttpAddress": "compute-2:8042",
        "allocatedMB": -1,
        "allocatedVCores": -1,
        "runningContainers": -1,
        "memorySeconds": 1756786,
        "vcoreSeconds": 1546,
        "preemptedResourceMB": 0,
        "preemptedResourceVCores": 0,
        "numNonAMContainerPreempted": 0,
        "numAMContainerPreempted": 0
      }
    ]
  }
}
\end{lstlisting}

\begin{lstlisting}[label=lst:hadoopAttemptListRestTls,style=json,
caption={[REST"=Ausgabe aller Ausführungen einer \gls{Anwendung} vom \acrshort{TLS}]
    \gls{REST}"=Ausgabe aller Ausführungen einer \gls{Anwendung} vom \acrshort{TLS}.\\
    URL: \url{http://addr:port/ws/v1/applicationhistory/apps/{appid}/appattempts}}]
{
  "appAttempt": [
    {
      "appAttemptId": "appattempt_1518429920717_0001_000001",
      "host": "compute-2",
      "rpcPort": 46481,
      "trackingUrl": "http://controller:8088/proxy/application_1518429920717_0001/",
      "originalTrackingUrl": "http://controller:19888/jobhistory/job/job_1518429920717_0001",
      "diagnosticsInfo": "",
      "appAttemptState": "FINISHED",
      "amContainerId": "container_1518429920717_0001_01_000001"
    }
  ]
}
\end{lstlisting}



\end{document}