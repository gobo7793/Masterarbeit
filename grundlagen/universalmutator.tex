\section{Universalmutator}
\label{sec:universalmutator}

\citeauthor{Groce2018} haben in \cite{Groce2018} den \textbf{Universalmutator} vorgestellt.\footnote{\url{https://github.com/agroce/universalmutator}} 
Der Universalmutator ist ein Tool, mit dessen Hilfe sich automatisiert Mutationstests erstellen lassen.
Mutationstests werden vor allem in der Forschung eingesetzt, um Testsysteme zu verifizieren, indem das \ac{SuT} verändert wird (mutiert).
Der Universalmutator kann hierbei nicht nur innerhalb einer bestimmten Umgebung bzw. Programmiersprache, sondern prinzipiell für alle Programmiersprachen eingesetzt werden.
Dies wird dadurch ermöglicht, dass die Mutationen basierend auf einem oder mehreren Regelsätze durchgeführt werden und somit der Quellcode mutiert wird.
So kann vom Universalmutator Quellcode \uA in den Sprachen Python, Java, C/C++ oder Swift mutiert werden \cite{Groce2018}.

Da bei der Ausführung des Universalmutators auch zahlreiche Mutanten generiert werden, die nicht kompiliert bzw. ausgeführt werden können, nutzt das Tool die Compiler der jeweiligen Sprache zur Validierung der generierten Mutationen.
Ein validierter Mutant zeichnet sich hierbei dadurch aus, dass dieser durch den Original"=Compiler der jeweiligen Sprache kompiliert werden kann und nicht redundant ist, also durch eine andere Mutation bereits erzeugt wurde \cite{Groce2018}.
Der Universalmutator generiert zwar für alle Mutanten einen entsprechenden Quellcode, speichert standardmäßig jedoch nur für validierte Mutanten.
Mithilfe eines entsprechenden Startparameters kann die Validierung jedoch durch ein benutzerdefiniertes Programm durchgeführt werden oder nicht durchgeführt werden \cite{Groce2018,UniversalmutatorSourceGenmutants}.
