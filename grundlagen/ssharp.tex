\section{\acl{ss}}
\label{sec:sSharp}

\todo{Testen unter SS allgemein genauer erklären}
\textbf{\acf{ss}} ist ein am \isse der Universität Augsburg entwickeltes Framework zum Testen und Verifizieren von Systemen und Modellen.
Da es in \cS entwickelt wurde und \cS auch zum Entwickeln von Modellen und dazugehörigen Testszenarien genutzt wird, können zahlreiche Features des .NET"=Frameworks bzw. der Sprache \cS im Speziellen genutzt werden.
\ac{ss} vereint dabei die Simulation, die Visualisierung, modellbasierte Tests sowie die Verifizierung der Modelle durch einen \ac{MCr} \cite{Habermaier2015,Habermaier2016}.
Dadurch können alle Schritte einer vollständigen Analyse inkl. Modellierung direkt im Visual Studio ausgeführt werden und somit auch alle Features der IDE und .NET, wie \zB die Debugging"=Werkzeuge, genutzt werden.
Um entsprechende Analysen zu gewährleisten, hat das Framework jedoch auch einige Einschränkungen, wodurch \zB Schleifen und Rekursionen nur eingeschränkt bzw. nicht möglich sind.
Eine der größten Einschränkungen ist allerdings, dass während der Laufzeit keine neuen Objektinstanzen innerhalb des zu testenden Modells erzeugt werden können, sodass alle benötigten Instanzen bereits während der Initialisierung des Modells erzeugt werden müssen \cite{Habermaier2015}.

Um nun ein System testen zu können, muss dieses zunächst mithilfe von \cS-Klassen und -Instanzen modelliert werden.
Die dafür verwendeten Modelle sind meist stark vereinfacht und bilden nur die wesentlichen Aspekte der realen Systeme ab.
Für einen korrekten Test ist es jedoch wichtig, dass das Modell des Systems vergleichbar mit dem echten System ist.

Folgendes Beispiel zeigt den typischen, grundlegenden Aufbau einer \ac{ss}-Komponente:

\lstinputlisting[label=lst:ssExample,style=cs,
caption={Grundlegender Aufbau einer \acs{ss}-Komponente}]
{./listings/ssExample.cs}

Jede Komponente des Modells muss von \texttt{Component} erben, um als \ac{ss}-Komponente definiert zu sein.
Jede Komponente kann nun temporäre (\texttt{TransientFault}) oder dauerhafte (\texttt{PermanentFault}) Komponentenfehler enthalten, welche zunächst innerhalb der Komponente als Felder definiert werden. 
Der Effekt eines Komponentenfehlers wird anschließend in der entsprechenden Effekt"=Klasse definiert, welche von der Hauptklasse (hier \texttt{YarnNode}) erbt und mithilfe des Attributs \texttt{FaultEffectAttribute} dem dazugehörigen Komponentenfehler zugeordnet wird \cite{Habermaier2016}.

Um die Modelle zu testen, kommen in \ac{ss} verschiedene Werkzeuge zum Einsatz.
Eines davon ist eine reine Simulation, bei der das Framework nur einen Ausführungspfad ausführt und dabei keine Komponentenfehler aktiviert bzw. die Aktivierung \textit{manuell} gesteuert werden kann.
Ein weiterer Nutzen liegt in der Möglichkeit, im ausgeführten Ausführungspfad zeitliche Abläufe zu berücksichtigen, da hier das Modell Schritt für Schritt ausgeführt wird.
Hierbei wird für jede im Modell genutzte Komponente pro Schritt einmal die jeweilige Methode \texttt{Update()} aufgerufen, in der die jeweiligen Komponenten ihre Aktivitäten durchführen \cite{Habermaier2016}.

Ein anderes wichtiges Werkzeug von \ac{ss} ist die \ac{DCCA}, welche eine vollautomatische und \ac{MC}-basierte Sicherheitsanalyse ermöglicht.
Dabei wird selbstständig die Menge der aktivierten Komponentenfehler ermittelt, mit denen das Gesamtsystem nicht mehr rekonfiguriert werden kann und somit ausfällt.
Je nach Konfiguration können dazu auch Heuristiken genutzt werden, welche die Analyse beschleunigen und genauer machen können \cite{Eberhardinger2016}.
Dabei werden die verschiedenen aktivierten Komponentenfehler während der Analyse in tolerierbare und nicht-tolerierbare Fehler unterschieden.
Tolerierbare Komponentenfehler werden dazu genutzt, die Grenzen der Selbstkonfiguration des Systems zu ermitteln.
Dabei wird für jeden Systemzustand nach einer Rekonfiguration durch die \ac{DCCA} eine neue Fehlermenge ermittelt, mit der das System gerade noch lauffähig ist.
Das Auftreten eines tolerierbaren Komponentenfehler ist also gleichbedeutend mit einem einfachen Fehler im System, welcher die gesamte Funktionsweise des Systems nicht massiv einschränkt und eine Rekonfiguration noch ermöglicht.
Sobald jedoch ein Fehler auftritt, durch den eine Rekonfiguration des Systems nicht mehr möglich ist, wurde ein nicht-tolerierbarer Fehler gefunden, durch den das System nicht mehr funktionsfähig ist \cite{Habermaier2015}.

Das dritte Werkzeug zur Ausführung von Modellen in \ac{ss} ist der \ac{MCr} selbst.
Hierbei kann der in \ac{ss} bereits enthaltene, oder alternativ \emph{LTSmin}\footnote{\url{http://ltsmin.utwente.nl/}} genutzt werden \cite{SSWikiModelChecking}.
Beim \ac{MC} werden in einem \emph{brute-force}-ähnlichem Verfahren alle möglichen Zustände und Ausführungspfade in einem Modell mit einer endlichen Anzahl an Zuständen getestet.
Dadurch wird es ermöglicht, verschiedene Eigenschaften eines System zu testen und Fehler (\zB Deadlocks) zu erkennen \cite{Grumberg1999}.
