\section{Generelle Erkenntnisse}
\label{sec:generalResults}

\todo{Zusammenfassung}

\subsection{Betrachtung der \ac{MARP}"=Werte}
\label{sec:marpValueResults}

% Ergebnisse MARP-Werte
Bei der Betrachtung der \ac{MARP}"=Werte lässt sich generell sagen, dass die Selfbalancing"=Komponente den Wert selbstständig anpasst.
Während bei allen Testfällen mit dem Mutationsszenario der \ac{MARP}"=Wert unverändert blieb, wurde er bei 15 von 19 Ausführungen der 16 Testfälle ohne Mutationsszenario verändert:

\begin{table}[h]
    \begin{tabular}{l|c|c|c|c|c|c|c|c|c|c}
    	Testfall &  1.1  &  1.2  &   3   &  5.1  &  5.2  &  7.1  &  7.2  &   9   &  11   &  13   \\ \hline
    	Wert     & 0,100 & 0,100 & 0,474 & 0,242 & 0,100 & 0,100 & 1,000 & 0,269 & 0,539 & 0,356 \\
        \multicolumn{11}{c}{} \\
    	Testfall &  15   &  17   &  19   &  21   &  23   &  25   &  27   &  29   &  31   &  \\ \hline
    	Wert     & 0,368 & 0,731 & 0,430 & 0,335 & 0,498 & 0,521 & ,0819 & 0,273 & 0,333 &
    \end{tabular}
    \caption{Finale \ac{MARP}"=Werte der Ausführungen ohne Mutationsszenario}
    \label{tab:finalMarpValues}
\end{table}

% Warum einige MARP-Werte nicht geändert?
Da er in den Testfällen \#1 und \#7 bei der ersten Ausführung nicht verändert wurde, wurden beide Testfälle erneut ausgeführt, wobei der \ac{MARP}"=Wert bei letzterem mehrmals geändert wurde, bevor er im finalen Clusterstatus den Wert 1 erhielt.
Im Testfall \#1 wurde der Wert dagegen bei keiner der beiden Ausführungen verändert.

\subsection{Aktivierung und Deaktivierung der Komponentenfehler}
\label{sec:faultInjectionEval}

% Verhalten der Faults?

\subsection{Nicht erfolgreich abgeschlossene Anwendungen}
\label{sec:failedAppsEval}

% Was ist bei der Anzahl der gefailten Apps aufgefallen?

% Warum sind Apps gefailt?

% Kuriose Steps: AppMstr auf zb Node1 -> Stop -> AppMstr zb Node 5 -> Stop -> App Failed

% DFSIO Read Mapfail

