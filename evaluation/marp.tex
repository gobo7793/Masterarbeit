\section{Betrachtung der \ac{MARP}"=Werte}
\label{sec:marpValueResults}

Bei der Betrachtung der \ac{MARP}"=Werte lässt sich generell sagen, dass die Selfbalancing"=Komponente den \ac{MARP}"=Wert entsprechend der Auslastung des Clusters anpasst.
Während bei allen Testkonfigurationen, bei denen alle 4 Mutationen aktiv waren, der \ac{MARP}"=Wert unverändert blieb, wurde er bei 16 von 20 Ausführungen der 16 Konfigurationen ohne Mutationen verändert:

\begin{table}[h]
    \begin{tabular}{l|c|c|c|c|c|c|c|c|c|c}
    	Konf. &  1.1  &  1.2  &   3   &  5.1  &  5.2  &  7.1  &  7.2  &  9.1  &  9.2  &  11   \\ \hline
    	Wert  & 0,100 & 0,100 & 0,474 & 0,242 & 0,100 & 0,100 & 1,000 & 0,269 & 0,175 & 0,539 \\
    	\multicolumn{11}{c}{} \\
    	Konf. &  13   &  15   &  17   &  19   &  21   &  23   &  25   &  27   &  29   &  31   \\ \hline
    	Wert  & 0,356 & 0,368 & 0,731 & 0,430 & 0,335 & 0,498 & 0,521 & ,0819 & 0,273 & 0,333
    \end{tabular}
    \caption{Finale \ac{MARP}"=Werte der Testkonfigurationen ohne Mutanten}
    \label{tab:finalMarpValues}
\end{table}

Da er in den Konfigurationen 1 und 7 bei der jeweils ersten Ausführung nicht verändert wurde, wurden beide Konfigurationen erneut ausgeführt, wobei der \ac{MARP}"=Wert bei letzterem mehrmals erhöht wurde, bevor er im finalen Clusterstatus auf 1 gesetzt wurde.
Bei der Konfiguration 1 wurde der Wert dagegen bei keiner der beiden Ausführungen verändert.

Die nicht durchgeführte Änderung des \ac{MARP}"=Wertes in Konfiguration 1 liegt sehr wahrscheinlich daran, dass in hier nur im ersten der fünf Testfälle zwei Anwendungen gleichzeitig gestartet werden.
Dadurch wurden in allen 10 Testfällen zusammen nur 8 Anwendungen gestartet, die Hälfte davon jeweils beim ersten Testfall.
Da zudem 4 der 8 Anwendungen nur kleine Anwendungen (\acl{rtw} und \acl{pi}) sind und diese entsprechend schnell beendet werden können, steht den wesentlich ressourcenintensiveren Anwendungen \acl{dfw} und \acl{dfr} das gesamte Cluster nahezu exklusiv zur Verfügung.
Daher stehen den Anwendungen ausreichend Ressourcen zur Verfügung, was eine Anpassung des \ac{MARP}"=Wertes unnötig erscheinen lässt und daher durch die Selfbalancing"=Komponente nicht durchgeführt wird.

Bei der Testkonfiguration 7 ist dies ähnlich, wobei die gesamte Last auf mehr Nodes verteilt werden kann.
Der Test 7.2 und der hier deutlich veränderte \ac{MARP}"=Wert im Vergleich zu Test 7.1 ohne Anpassung zeigt jedoch auch, dass es stark abhängig davon ist, wie die Last im Cluster verteilt wird.
Bestätigt wird dies durch die Tests 9.1 und 9.2, da bei letzterem weniger Komponentenfehler injiziert wurden und sich die Last entsprechend besser verteilen konnte.
Dadurch war im Test 9.2 ein um rund 0,1 niedrigerer \ac{MARP}"=Wert als im Test 9.1 nötig.

Auffällig war zudem, dass der \ac{MARP}"=Wert in den Testausführungen 7.2, 9.1 und 23 nicht direkt im ersten Testfall verändert wurde, sondern erst bei der Ausführung von Testfällen im späteren Verlauf der jeweiligen Tests.
Als Resultat wurde daher in 9 der 15 Testfälle zunächst das hierfür genutzte Constraint als ungültig validiert.
