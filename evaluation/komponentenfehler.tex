\section{Betrachtung der Komponentenfehler}
\label{sec:faultEval}

\todo{Umformulieren}
Die Aktivierung und Deaktivierung der Komponentenfehler in einem Testfall hängt neben dem zur Berechnung benötigtem Seed vor allem von den zuvor ausgeführten Testfällen einer Testkonfiguration ab (vgl. \autoref{sec:simulationFaultActivation}).
Daher werden abhängig von der Lastverteilung im Cluster auch bei einer mehrmaligen Ausführung der gleichen Konfiguration \uU unterschiedliche Komponentenfehler aktiviert.
Unterschieden werden muss hierbei zudem zwischen aktivierten und injizierten Komponentenfehlern.
Während beide implementierten Komponentenfehler für einen Node in einem Testfall auch gleichzeitig aktiviert werden konnten, wurde gemäß \todo{Abschnitt mit Komponentenfehler im Node} in so einem Fall jedoch nur der \texttt{NodeDead}"=Fehler im Cluster injiziert.
Die Deaktivierung bzw. das Reparieren der Komponentenfehler verhält sich analog hierzu.

\subsection{Aktivierte und deaktivierte Komponentenfehler}
\label{sec:actDeactFaults}

Im Vergleich zwischen korrespondierenden Konfigurationen, die sich nur in der Nutzung des Mutationsszenarios unterschieden, wurde nur 5 mal die gleiche Anzahl an Komponentenfehler aktiviert, bei der Deaktivierung der Komponentenfehler besitzen nur 3 korrespondierende Konfigurationen die gleiche Anzahl.
Die Anzahl der aktivierten und deaktivierten Komponentenfehler unterschied sich dagegen in 8 bzw. 7 korrespondierenden Testkonfigurationen um jeweils einen Komponentenfehler.
In allen anderen Ausführungen von korrespondierenden Konfigurationen unterschied sich die Anzahl um jeweils mehr als einen Komponentenfehler.
Mit 20 aktivierten Komponentenfehlern wurden bei der Ausführung der Testkonfiguration 32 die meisten aktiviert, die meisten Komponentenfehler deaktiviert wurden bei den Konfigurationen 11 und 12 mit jeweils 15 Stück.
Nur im Test zur Konfiguration 2 wurden mit 3 Stück alle aktivierten Komponentenfehler während der Simulation auch wieder deaktiviert.
In den Tests 4, 5.1, 5.2 und 6 wurden jeweils 6 oder 7 Komponentenfehler aktiviert, jedoch keine deaktiviert, weshalb diese Tests bereits beim 3. ausgeführten Testfall gemäß \autoref{sec:simulationOracle} abgebrochen wurden.

Im Vergleich zwischen den Tests von korrespondierenden Testkonfigurationen sind die Tests der Konfigurationen 1 und 2 auffällig.
Während beim Test 1.1 mit 5 Komponentenfehlern bzw. beim Test 1.2 mit 7 Komponentenfehlern jeweils rund jeder achte mögliche Komponentenfehler aktiviert wurde, wurden beim Test 2 lediglich 3 Komponentenfehler für 4 Nodes in 5 Testfällen (insgesamt also 40 mögliche Komponentenfehler) aktiviert.
Eine geringere Quote weist lediglich Test 9.2 auf, bei dem mit 4 von 60 möglichen Komponentenfehler nur 7 \% aktiviert wurden.
Die Testkonfiguration 9 ist auch deshalb auffällig, da im Test 9.1 fast dreimal so viele Komponentenfehler, also 11 Stück, aktiviert wurden.
Auch in den korrespondierenden Tests der Konfiguration 10 liegt die Anzahl der aktivierten Komponentenfehler mit 7 bis 11 jeweils mehr als doppelt so hoch wie in Test 9.2.

Auffällig ist zudem, dass bei korrespondierenden Testkonfigurationen mit unterschiedlicher Anzahl an aktivierten Komponentenfehlern die niedrigere Anzahl meist die Mutationstests aufweisen.
Nur bei den Tests 27 und 28.1 sowie bei den Tests 31 und 32 weisen die Tests ohne Mutationsszenario die niedrigere Anzahl auf.
\todo{auf diskussion der selfbalancing komponenten verweisen als grund}

\subsection{Nicht erkannte, injizierte bzw. reparierte Komponentenfehler}
\label{sec:notDetectedFaults}

% SuT/Test-Constraint StartNode 4 bei #17-28
