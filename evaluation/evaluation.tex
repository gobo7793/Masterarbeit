\chapter{Evaluation der Ergebnisse}
\label{chap:evaluationResults}

In \autoref{sec:selectTestcases} wurden 32 Testkonfigurationen ermittelt, die mithilfe des in dieser Fallstudie entwickelten Testansatzes insgesamt 43 mal ausgeführt wurden.
Die meisten Konfigurationen wurden hierbei jeweils einmal ausgeführt, 7 Konfigurationen wurden auch mehrmals ausgeführt.
Die Gründe für eine mehrfache Ausführung einzelner Konfigurationen sind in diesem Kapitel im Rahmen der entsprechenden Auffälligkeiten bzw. Fehler beschrieben.
Eine Übersicht aller genutzten Testkonfigurationen und deren Ausführungen findet sich in \autoref{app:overviewExecutedTestCases}.

Bei der Auswertung der Programmlogs der einzelnen Tests musste zudem beachtet werden, dass die jeweiligen Monitoring"=Informationen nur Momentaufnahmen bilden.
Vor allem bei längeren Schritten werden vom \ac{RM} sehr viele Anpassungen vorgenommen, die aufgrund der in \autoref{sec:simulationStep} implementierten Struktur eines Simulations"=Schrittes bzw. Testfalls nicht erkannt werden können.

Zur Auswertung der Evaluation dienten vor allem die in \todo{Constraintabschnitt} implementierten Constraints, die sich aus den in \autoref{sec:requirements} definierten Anforderungen ergeben.

In diesem Kapitel wurden folgende Begriffe mit folgenden Bedeutungen genutzt:

\begin{description}
    \item [Gefailte Anwendung] \hfill \\
        Nicht vollständig abgeschlossene Anwendung, die aufgrund eines Fehler vorzeitig beendet wurde.
    \item [Testkonfiguration] \hfill \\
        Eine Konfiguration bestehend aus mehreren Parametern, die einen Test definieren.
        Die Nummerierung der in \autoref{sec:selectTestcases} definierten Konfigurationen erfolgte fortlaufend.
    \item [Testfall] \hfill \\
        Ein ausgeführter Simulations"=Schritt.
        Ein Testfall wird während der Laufzeit basierend auf einer zugrundeliegenden Testkonfiguration sowie den Ereignissen und Ergebnissen zuvor ausgeführter Testfälle der zugrundeliegenden Konfiguration generiert.
        In einem Testfall können daher unterschiedliche Komponentenfehler aktiviert und deaktiviert sowie unterschiedliche Anwendungen gestartet werden, auch wenn sie durch die gleiche Testkonfiguration generiert wurden.
    \item [Test] \hfill \\
        Eine Ausführung einer Testkonfiguration.
        Um mehrmalige Ausführungen einer Testkonfiguration zu kennzeichnen, wurde der jeweiligen Konfiguration eine weitere Ziffer angehängt.
\end{description}

\section{Statistische Kenndaten}
\label{sec:evaluationStats}

Die Dauer aller Simulationen betrug 4:06:03 Stunden, die gesamte Ausführungsdauer inkl. Starten und Beenden des Clusters bei jeder Konfiguration betrug 5:07:47 Stunden.
Von den 280 Testfällen, die ausgeführt hätten sollen, wurden nur 215 Testfälle (77 \%) ausgeführt.
Der Grund für den Abbruch von 13 Tests liegt im in \autoref{sec:simulationOracle} beschriebenen Abbruch der Simulation, wenn keine Rekonfiguration des Clusters mehr möglich ist, also bei allen Nodes ein Komponentenfehler injiziert und dies beim Monitoring erkannt wurde.

Insgesamt wurde bei allen Tests 419 Komponentenfehler aktiviert (14 \% von 2980 möglichen), von denen jedoch nicht alle injiziert wurden, da bei einigen Testfällen beide Komponentenfehler der Nodes gleichzeitig aktiviert wurden.
In diesen Fällen überwog die Aktivierung es Komponentenfehlers, der den Node komplett beendet.
Von allen aktivierten Komponentenfehlern wurden während der Simulationen 251 Komponentenfehler deaktiviert bzw. repariert, was eine Quote von 60 \% ergibt.
In 4 der ausgeführten Testfällen wurde jedoch kein einziger Komponentenfehler deaktiviert, weshalb die Tests der Konfigurationen 4, 5.1, 5.2 und 6 entsprechend frühzeitig abgebrochen wurden.

Bei den 42 Tests wurden 393 Anwendungen im Cluster gestartet, von denen 200 (51 \%) erfolgreich beendet wurden, nicht erfolgreich waren 102 (26 \%).
Vorzeitig abgebrochen wurden 49 Anwendungen (12 \%), was 42 Anwendungen macht, die am Ende der Simulationen noch ausgeführt wurden.
Nicht eingerechnet sind hier 25 nicht gestartete Anwendungen, die Gründe hierfür sind in \autoref{sec:notStartedApps} erläutert.
Für die gestarteten Anwendungen wurden 532 Attempts mit einem \ac{AppMstr} allokiert, was 1,35 Attempts pro Anwendung ergibt.
Auffällig ist hierbei, dass mit 200 Attempts ein Attempt mehr aufgrund eines \ac{AppMstr}"=Timeouts abgebrochen wurde, als erfolgreich beendet wurden.
30 weitere Attempts wurden aufgrund eines Fehlers im Map"=Task abgebrochen, 12 weitere terminierten mit dem Exitcode -100, was ebenfalls auf Fehler hindeutet.
Das macht dadurch eine Quote von 45,5 \% aller Attempts, die nicht erfolgreich abgeschlossen werden konnten.
Beim Monitoring wurden 3039 Anwendungs"=Container erkannt, was im Schnitt 7,73 Container pro Anwendung bzw. 5,71 pro Attempt ergibt.
Da bei den zu startenden Anwendungen einige kleine und einige sehr ressourcenintensive Anwendungen enthalten sind (vgl. \autoref{sec:appSelection}), kann sich die Anzahl der Container zwischen den einzelnen Anwendungen sehr unterscheiden.

Vom Oracle wurden bei allen Tests zusammengezählt 74.051 Constraints validiert, von denen 541 falsifiziert wurden (0,73 \%).
Die meisten ungültigen Constraints hatten hierbei die Konfigurationen 31 und 32 mit 40 bzw. 42 Constraints (von jeweils 5140 geprüften), die höchste Quote Konfiguration 8 mit 1,97 \% (13 von 661) der Constraints.
Der Hauptgrund für die teilweise sehr hohe Anzahl an ungültiger Constraints vor allem liegt darin, dass die Constraints für fehlerhaften Anwendungen auch in nachfolgenden Testfällen innerhalb einer Ausführung einer Testkonfiguration als ungültig erkannt werden.
Dies resultiert in bis zu 34 ungültigen Constraints für fehlerhafte Anwendungen bei den einzelnen Tests.


\section{Zusammenfassung der Ergebnisse}
\label{sec:evaluationResults}

Zusammenfassend lässt sich, basierend auf der vorhandenen Datenbasis der 43 ausgeführten Tests, sagen, dass sich das entwickelte Testsystem und das Hadoop"=Cluster im Großen und Ganzen so verhält, wie es erwartet werden konnte.
Dennoch wurden nicht alle der in \cref{subsec:functionalRequirements} definierten funktionalen Anforderungen an das Cluster selbst und der in \cref{subsec:testRequirements} definierten Anforderungen an das gesamte Testsystem vollständig erfüllt.
Die meisten dieser Anforderungen wurden mithilfe der definierten Constraints implementiert (vgl. \cref{subsec:yarnComponentConstraints,subsec:yarnController,subsec:simulationBasics}), wodurch diese Anforderungen bei jedem Testfall automatisch validiert werden konnten.

Vor allem die funktionale Anforderung, wonach alle Tasks vollständig ausgeführt, sofern sie nicht abgebrochen werden, wurde bei den 110 nicht erfolgreichen Anwendungen nicht erfüllt.
Aber auch bei einigen vollständig ausgeführten Anwendungen wurde diese nicht komplett erfüllt, was die mit dem Exitcode -100 beendeten Attempts zeigen.
Bei Attempts mit dem Exitcode -100 wird zudem die Anforderung, dass kein Task an defekte Nodes gesendet wird, verletzt (vgl. \cref{subsec:failedApps}).

Bei der Validierung der Constraints ist es zudem vorgekommen, dass die Constraints der Anforderungen, dass die Konfiguration aktualisiert sowie der Status des Clusters vom Testsystem erkannt und im Testmodell gespeichert wird, als ungültig validiert wurden.
Dies waren nach einer genaueren Betrachtung der Gründe hierfür zum Teil jedoch falscher Alarm, wodurch diese Anforderungen zu großen Teilen als erfüllt angesehen werden können (vgl. \cref{subsec:notDetectedHost2}).
Genauso verhält es sich bei den nicht erkannten, injizierten und reparierten Komponentenfehlern, wonach die Anforderungen, dass defekte Nodes und Verbindungsabbrüche erkannt werden, bei der Betrachtung eines einzelnen Testfalls in 19 Fällen zwar nicht erfüllt, bei der Betrachtung der gesamten Tests jedoch als erfüllt angesehen werden können (vgl. \cref{subsec:notDetectedFaults}).

Auch die Anforderung, dass sich der MARP"=Wert anhand der ausgeführten Anwendungen verändert, wurde nicht immer erfüllt.
Die Betrachtung der Werte bei den einzelnen Tests ergab, dass es durchaus möglich ist, dass die bei einem Test ausgeführten Anwendungen nicht ausreichen, damit sich der Wert verändert (vgl. \cref{sec:marpValueResults}).
Dennoch war es anhand dieser Anforderung möglich, die in \cref{sec:implMutationTests} implementierten Mutanten der Selfbalancing"=Komponente, bis auf einige Besonderheiten, in den ausgeführten Tests zu erkennen (vgl. \cref{sec:killingMutants}).
Auch die Anforderung, dass die in \cref{subsec:yarnComponentFaults} implementierten Komponentenfehler im realen Cluster injiziert und repariert werden, konnte nicht immer erfüllt werden (vgl. \cref{subsec:noReconf1922}).
In diesem Kontext zeigte sich aber, dass immer erkannt werden konnte, wenn keine weitere Rekonfiguration des Clusters möglich ist, womit diese Anforderung vollständig erfüllt wird (vgl. \cref{sec:noReconfig}).
Zudem konnte in einigen Fällen die Anforderung, dass mehrere Benchmark"=Anwendungen gleichzeitig gestartet und ausgeführt werden können, nicht erfüllt werden (vgl. \cref{subsec:notStartedApps}).

Zu einem großen Teil erfüllt werden konnte jedoch die Anforderung, dass ein Test vollautomatisch ausgeführt werden kann.
Lediglich bei der Ausführung von mehreren Testfällen direkt hintereinander mithilfe der in \cref{sec:implTestcases} implementierten \texttt{CaseStudyTests}"=Klasse kam es vor, dass die vom Connector bereitgestellten Submitter zum Starten von Anwendungen nicht ausgereicht haben (vgl. \cref{subsec:notEnoughSubmitter}).

Die genauen Gründe für die verletzten Anforderungen und Constraints sind in den bereits verwiesenen, nachfolgenden Abschnitten erläutert.

Die 43 ausgeführten Tests haben aber auch gezeigt, dass das Cluster ohne Auswirkung auf seine Funktionsweise auf einem oder mehreren Hosts ausgeführt werden kann.
Auch zeigte sich bei den 7 Testkonfigurationen mit mehrmaligen Ausführungen, dass die Tests und seine Testfälle im Grunde mehrmals ausgeführt werden können.
Die einzigen Unterschiede bei den jeweiligen Ausführungen waren ausschließlich durch die Verteilung der Last innerhalb des Clusters bedingt, was sich vor allem in direkten Vergleichen zwischen korrespondierenden Tests zeigt.


\section{Betrachtung der MARP"=Werte}
\label{sec:marpValueResults}

Bei der Betrachtung der \gls{MARP}"=Werte lässt sich generell sagen, dass die Selfbalancing"=Komponente den \gls{MARP}"=Wert entsprechend der Auslastung des Clusters anpasst.
Während bei allen Testkonfigurationen, bei denen Mutationen aktiv waren, der \gls{MARP}"=Wert unverändert blieb (vgl. \cref{sec:killingMutants}), wurde er bei 17 von 21 Ausführungen der 16 Konfigurationen ohne Mutationen verändert:

\begin{table}[h]
    \begin{tabular}{l|c|c|c|c|c|c|c}
    	Konf. &  1.1  &  1.2  &   3   &  5.1  &  5.2  &  7.1  &  7.2  \\ \hline
    	Wert  & 0,100 & 0,100 & 0,474 & 0,242 & 0,100 & 0,100 & 1,000 \\
    	\multicolumn{8}{c}{} \\
    	Konf. &  9.1  &  9.2  &  11   &  13   &  15   &  17   &  19   \\ \hline
    	Wert  & 0,269 & 0,175 & 0,539 & 0,356 & 0,368 & 0,731 & 0,430 \\
    	\multicolumn{8}{c}{} \\
    	Konf. &  21   &  23   &  25   &  27   &  29   & 31.1  & 31.2  \\ \hline
    	Wert  & 0,335 & 0,498 & 0,521 & ,0819 & 0,273 & 0,488 & 0,333
    \end{tabular}
    \caption[Finale \glsentryshort{MARP}"=Werte der Testkonfigurationen ohne Mutanten.]
    {Finale \acrshort{MARP}"=Werte der Testkonfigurationen ohne Mutanten.
    Eine Übersicht aller Tests findet sich in \cref{app:overviewExecutedTestCases}.}
    \label{tab:finalMarpValues}
\end{table}

Da er in den Konfigurationen 1 und 7 bei der jeweils ersten Ausführung nicht verändert wurde, wurden beide Konfigurationen erneut ausgeführt, wobei der \gls{MARP}"=Wert bei letzterem mehrmals erhöht wurde, bevor er im finalen Clusterstatus auf 1 gesetzt wurde.
Bei der Konfiguration 1 wurde der Wert dagegen bei keiner der beiden Ausführungen verändert.

Die nicht durchgeführte Änderung des \gls{MARP}"=Wertes in Konfiguration 1 liegt sehr wahrscheinlich daran, dass in hier nur im ersten der fünf Testfälle zwei Anwendungen gleichzeitig gestartet werden.
Dadurch wurden in allen zehn ausgeführten Testfällen zusammen nur acht Anwendungen gestartet, die Hälfte davon jeweils beim ersten Testfall.
Da zudem vier der acht Anwendungen nur kleine Anwendungen (\acrlong{rtw} und \acrlong{pi}) sind, und diese entsprechend schnell abgeschlossen werden können, steht den wesentlich umfangreicheren Anwendungen \acrlong{dfw} und \acrlong{dfr} das gesamte Cluster nahezu exklusiv zur Verfügung.
Daher stehen in diesen Tests allen Anwendungen ausreichend Ressourcen zur Verfügung, was eine Anpassung des \gls{MARP}"=Wertes unnötig erscheinen lässt und daher durch die Selfbalancing"=Komponente nicht durchgeführt wird.

Bei der Testkonfiguration 7 ist dies ähnlich, wobei die gesamte Last auf mehr Nodes verteilt werden kann.
Der beim Test 7.2 deutlich veränderte \gls{MARP}"=Wert im Vergleich zum Test 7.1 ohne Anpassung zeigt jedoch auch, dass es stark abhängig davon ist, wie die Last im Cluster verteilt wird.
Bestätigt wird dies durch die Tests 9.1 und 9.2, da bei letzterem weniger Komponentenfehler injiziert wurden und sich die Last entsprechend auf mehr aktive Nodes verteilen konnte.
Dadurch war im Test 9.2 ein um rund 0,1 niedrigerer \gls{MARP}"=Wert als im Test 9.1 nötig.

Auffällig war zudem, dass der \gls{MARP}"=Wert in den Testausführungen 7.2, 9.1 und 23 nicht direkt im ersten Testfall verändert wurde, sondern erst bei der Ausführung von Testfällen im späteren Verlauf der jeweiligen Tests.
Als Resultat wurde daher in 9 der 15 Testfälle der drei Testausführungen das entsprechende Constraint verletzt.


\section{Erkennung der Mutanten}
\label{sec:killingMutants}

Da die Selfbalancing"=Komponente den \ac{MARP}"=Wert basierend auf der aktuellen Auslastung des Clusters anpasst, konnte anhand der Betrachtung der \ac{MARP}"=Werte auch geprüft werden, ob die implementierten Mutationen vom Testsystem erkannt wurden.
Um das zu bewerkstelligen wurden von den 16 Testkonfigurationen mit einem Mutationsszenario 22 Testausführungen durchgeführt (vgl. \cref{app:overviewExecutedTestCases}).

Zunächst wurde das Mutationsszenario genutzt, in dem alle vier Mutationen enthalten sind (vgl. \cref{sec:implMutationTests}).
Bei jeder der 17 Testausführungen mit allen Mutanten wurden diese basierend auf dem Constraint zur Erkennung des \ac{MARP}"=Wertes erkannt.
Eine Besonderheit bildet hier jedoch der Test 2, der den korrespondierenden Mutationstest zur Konfiguration 1 darstellt, bei der bei beiden Ausführungen der \ac{MARP}"=Wert nicht verändert wurde.
Bei Test 2 kann daher nicht eindeutig festgestellt werden, ob der Mutant erkannt wurde, oder ob aufgrund der gestarteten Anwendungen der \ac{MARP}"=Wert nicht verändert wurde (vgl. Vermutungen zu Testkonfiguration 1 in \cref{sec:marpValueResults}).

Anders ist dies im Vergleich der Ausführungen der Testkonfigurationen 7 und 8.
Durch die massive Veränderung des \ac{MARP}"=Wertes im Test 7.2 auf den finalen Wert von 1 kann davon ausgegangen werden, dass die Mutanten der Konfiguration 8 erkannt wurden.
Dies wird dadurch gestützt, dass bei Konfiguration 8 im Gegensatz zur korrespondieren Testkonfiguration 3 der 4 gestarteten Anwendungen gefailt sind.
Zudem stellt das ein Indiz dafür dar, dass der Mutant in Konfiguration 2 erkannt worden sein könnte.

Während bei jeder Konfiguration ein Mutationsszenario mit jeweils allen vier Mutanten genutzt wurde, wurde die Testkonfiguration 10 zusätzlich mit jeweils einem Mutanten ausgeführt.
Ziel hierbei war es zu validieren, ob einzelne Mutanten ebenfalls vom Testsystem erkannt werden oder zur Erkennung der Mutanten vom Testsystem eine Kombination aus mehreren Mutaten nötig ist.
Hierzu wurde die Testkonfiguration 10 mit unterschiedlichen Mutationsszenarien der Plattform Hadoop"=Benchmark ausgeführt, bei denen jeweils einer der in \cref{sec:implMutationTests} definierten Mutanten aktiv ist.
Die Auswahl dieser Testkonfiguration hierfür liegt darin begründet, dass hier das Cluster auf beiden Hosts mit zusammen sechs Nodes gestartet wird, auf denen bis zu vier Anwendungen gleichzeitig gestartet werden.
Zudem wurde bei den Tests 9.1 und 9.2 festgestellt, dass sich der \ac{MARP}"=Wert nicht direkt im ersten, sondern auch in später ausgeführten Testfällen ändern kann, er aber während der Ausführung wirklich geändert wird (vgl. \cref{sec:marpValueResults}).

Einige Ergebnisse der hierfür 5 ausgeführten Tests sind sehr unterschiedlich.
So variiert die Anzahl der aktivierten und deaktivierten Komponentenfehler zwischen 7 und 11 bzw. 5 und 9, sowie die Anzahl der gefailten Anwendungen zwischen 1 und 3.
Gemein haben alle Tests jedoch, dass der \ac{MARP}"=Wert bei allen 5 Tests nicht verändert wurde, womit alle Mutationen erkannt wurden.
Damit kann festgestellt werden, dass jeder der vier in \cref{sec:implMutationTests} beschriebenen Mutanten durch das entwickelte Testsystem erkannt wird.


\section{Betrachtung der Komponentenfehler}
\label{sec:faultEval}

Die Aktivierung und Deaktivierung der Komponentenfehler in einem \gls{Testfall} hängt neben dem zur Berechnung benötigtem Basisseed vor allem von den zuvor ausgeführten Testfällen bzw. der Lastverteilung bei den zuvor ausgeführten Testfällen eines \glspl{Test} ab (vgl. \cref{subsec:faultActivation}).
Daher wurden abhängig von der Lastverteilung im Cluster auch bei einer mehrmaligen Ausführung der gleichen Konfiguration bei einigen Testausführungen unterschiedliche Komponentenfehler aktiviert.

Unterschieden werden muss hierbei zudem zwischen aktivierten und injizierten Komponentenfehlern.
Während beide implementierten Komponentenfehler für einen Node in einem \gls{Testfall} auch gleichzeitig aktiviert werden konnten, wurde in so einem Fall jedoch nur der \texttt{NodeDead}"=Fehler im Cluster injiziert (vgl. \cref{subsec:yarnComponentFaults}).
Die Deaktivierung bzw. das Reparieren der Komponentenfehler verhält sich analog hierzu.

Im Folgenden wird nun ein Überblick über die bei den \glspl{Test} aktivierten bzw. deaktivierten und nicht injizierten Komponentenfehler bzw. erkannten Injektionen und Reperaturen der Komponentenfehler gegeben.

\subsection{Aktivierte und deaktivierte Komponentenfehler}
\label{subsec:actDeactFaults}

Die Aktivierung und Deaktivierung der Komponentenfehler hing manchmal stark von der ausgeführten \gls{Testkonfiguration} ab (eine Übersicht aller Testkonfigurationen und Tests findet sich in \cref{app:overviewExecutedTestCases}).
Im Vergleich zwischen korrespondierenden Konfigurationen, die sich nur in der Nutzung des Mutationsszenarios unterschieden, wurde nur bei 5 korrespondierenden Testkonfigurationen die gleiche Anzahl an Komponentenfehler aktiviert, bei der Deaktivierung der Komponentenfehler besitzen nur 4 korrespondierende Konfigurationen die gleiche Anzahl bei allen Tests.
Die Anzahl der aktivierten und deaktivierten Komponentenfehler unterschied sich dagegen in 8 bzw. 7 korrespondierenden \glspl{Testkonfiguration} um einen Komponentenfehler in allen Testausführungen.
Bei den anderen korrespondierenden Konfigurationen unterschied sich die Anzahl bei allen jeweiligen \glspl{Test} um mehr als einen Komponentenfehler.
Mit jeweils 20 aktivierten Komponentenfehlern wurden bei den \glspl{Test} 28.1, 31.1 und 32 die meisten aktiviert, die meisten Komponentenfehler deaktiviert wurden bei den \glspl{Test} der Konfigurationen 11 und 12 mit jeweils 15 Stück.
Nur im \gls{Test} zur Konfiguration 2 wurden mit 3 Fehlern alle aktivierten Komponentenfehler während der Simulation auch wieder deaktiviert.
In den \glspl{Test} 4, 5.1, 5.2 und 6 wurden jeweils 6 oder 7 Komponentenfehler aktiviert, jedoch keine deaktiviert, weshalb diese \glspl{Test} bereits beim dritten ausgeführten \gls{Testfall} abgebrochen wurden (vgl. \cref{subsec:oracleImpl,subsec:noReconf36}).

Im Vergleich zwischen den \glspl{Test} von korrespondierenden \glspl{Testkonfiguration} sind die \glspl{Test} der Konfigurationen 1 und 2 auffällig.
Während beim \gls{Test} 1.1 mit 5 Komponentenfehlern bzw. beim \gls{Test} 1.2 mit 7 Komponentenfehlern jeweils rund jeder achte mögliche Komponentenfehler aktiviert wurde, wurden beim \gls{Test} 2 lediglich 3 Komponentenfehler für 4 Nodes in 5 Testfällen (insgesamt also 40 mögliche Komponentenfehler) aktiviert.
Eine geringere Quote weist lediglich \gls{Test} 9.2 auf, bei dem mit 4 von 60 möglichen Komponentenfehler nur 7 Prozent aktiviert wurden.
Die \gls{Testkonfiguration} 9 ist darüber hinaus auch deshalb auffällig, da im \gls{Test} 9.1 fast dreimal so viele Komponentenfehler, also 11 Stück, aktiviert wurden.
Auch in den korrespondierenden \glspl{Test} der Konfiguration 10 liegt die Anzahl der aktivierten Komponentenfehler mit 7 bis 11 jeweils mehr als doppelt so hoch wie in \gls{Test} 9.2.

Auffällig ist zudem, dass bei korrespondierenden \glspl{Testkonfiguration} mit unterschiedlicher Anzahl an aktivierten Komponentenfehlern die niedrigere Anzahl meist diejenigen mit Mutationen aufweisen.
Nur bei den Konfigurationen 9 und 10, 13 und 14, 27 und 28 und 31 und 32 weisen einige \glspl{Test} ohne Mutationen eine geringere Anzahl an aktivierten Komponentenfehler auf als \glspl{Test} mit Mutationen.
Dies liegt wohl auch darin begründet, dass durch den veränderten \gls{MARP}"=Wert die verfügbaren Ressourcen besser an die \glspl{Anwendung} verteilt werden konnten und bestätigt damit die Funktionalität der von \citeauthor{Zhang2016} entwickelten Komponente.

Weitere Auffälligkeiten ergeben sich zudem beim Vergleich der Ausführungszeiten der Simulationen.
Die \glspl{Test} 9.2, 15, 31.1 sowie 31.2 stellen die einzigen \gls{Test} ohne Mutationen dar, bei denen die Simulation schneller abgeschlossen wurde als in den korrespondierenden \glspl{Test} mit Mutationsszenario.
Da sich das mit der generellen Aussage beim Vergleich der aktivierten Komponentenfehler deckt, kann davon ausgegangen werden, dass die geringere Anzahl an Komponentenfehler zudem die Auswirkung hat, dass \glspl{Anwendung} schneller gestartet werden können.
Der Grund hierfür könnte darin liegen, dass bei weniger injizierten Komponentenfehler auch entsprechend weniger Verwaltungsaufwand für bereits ausgeführte \glspl{Anwendung} nötig ist, wodurch neu gestartete \glspl{Anwendung} ebenfalls schneller durch das Cluster verarbeitet werden können.
Um hier jedoch eine fundierte Aussage treffen zu können, wären weitere vergleichende \glspl{Test} nötig (vgl. \cref{sec:discussionResults})

\subsection{Nicht erkannte, injizierte bzw. reparierte Komponentenfehler}
\label{subsec:notDetectedFaults}

Bei 18 aller ausgeführten \glspl{Test} ist aufgetreten, dass ein injizierter bzw. reparierter Komponentenfehler zunächst nicht vom Testsystem erkannt wurde.
Das betraf konkret drei mal das Injizieren eines Komponentenfehlers sowie 16 mal das Reparieren eines Komponentenfehlers:

\begin{table}[h]
    \begin{tabular}{c|ccc}
    	 \gls{Test}   & \gls{Testfall} &      Art       & Node \\ \hline
    	  1.1   &    5     &  Node beenden  &  4   \\
    	   2    &    5     &  Node starten  &  2   \\
    	  7.1   &    2     &  Node beenden  &  5   \\
    	  7.1   &    5     &  Node starten  &  5   \\
    	  7.2   &    5     &  Node starten  &  5   \\
    	  11    &    6     &  Node trennen  &  6   \\
    	17-28.2 &    2     & Node verbinden &  4
    \end{tabular} 
    \caption[Übersicht der nicht erkannten, injizierten/reparierten Komponentenfehler]
    {Übersicht der nicht erkannten, injizierten bzw. reparierten Komponentenfehler.
    Eine Übersicht aller Tests findet sich in \cref{app:overviewExecutedTestCases}.}
    \label{tab:notDetectedFaults}
\end{table}

Bei den aufgetretenen, verletzten Constraints fällt auf, dass die betroffenen Nodes im jeweils nachfolgenden \gls{Testfall} mit ihrem jeweils korrekten Status erkannt wurden.
Die 19 als ungültig markierten Constraints zu den Anforderungen, dass defekte Nodes und Verbindungsabbrüche erkannt werden (vgl. \cref{sec:requirements}), wurden somit korrekt, als auch inkorrekt, als ungültig validiert.
Dies liegt daran, dass das Cluster bei defekten Nodes erst einige Zeit benötigt, um den Ausfall eines Nodes zu erkennen.
Auch wenn ein Node nicht mehr defekt ist, benötigt dieser bzw. der \gls{RM} erst einige Zeit, bis erkannt wird, dass der Node wieder aktiv ist.
Dies liegt einerseits daran, dass Hadoop bzw. der \gls{RM} nicht kontinuierlich, sondern periodisch nach einer bestimmten Zeitspanne den Status der Nodes prüft und bei nicht erreichbaren Nodes zunächst solange wartet, bis die Abfrage durch einen \emph{Timeout} beendet wird (vgl. \cref{sec:hadoop}).
Zwar wurden beide Zeitspannen in den genutzten Szenarien der Plattform Hadoop"=Benchmark auf jeweils 10 Sekunden festgelegt (vgl. \cref{subsec:clusterBasics}), jedoch reichte diese Zeitspanne wohl nicht immer aus, um den Status rechtzeitig zu erkennen.
Beim Starten bzw. Wiederverbinden eines Nodes verhält es sich analog dazu, wobei Hadoop auf dem jeweiligen Node hier zunächst gestartet werden muss, bevor es sich dann selbstständig mit dem \gls{RM} verbindet, was ebenfalls eine gewisse Zeit benötigt.
Dies wird auch dadurch bestätigt, dass für die betroffenen Nodes in den jeweils nachfolgenden Testfällen bzw. dem finalen Clusterstatus oder in korrespondierenden \glspl{Test} der Status korrekt erkannt wurde, den die Nodes gemäß aufgrund der Komponentenfehler besitzen sollten.

Eine Besonderheit bilden hierbei zunächst die beiden \glspl{Test} zur Konfiguration 7.
Im \gls{Test} 7.1 wurde der gleiche Node zunächst nicht als defekt erkannt bevor er im letzten \gls{Testfall} noch als defekt erkannt wurde, obwohl er bereits wieder gestartet wurde.
Dazwischen wurde der betroffene Node 5 zunächst im \gls{Testfall} 3 wieder gestartet, bevor er im \gls{Testfall} 4 wieder beendet wurde, von denen beide Aktionen korrekt erkannt wurden.
Im Unterschied zum ersten \gls{Test} der Konfiguration wurde im \gls{Test} 7.2 nur das Starten des Nodes nicht korrekt erkannt, während das beenden des Nodes 5 im 2. \gls{Testfall} erkannt wurde.
Im finalen Clusterstatus ist der Node jedoch wie im \gls{Test} 7.1 korrekt als aktiv markiert.

Die weitere Besonderheit bilden die \glspl{Test} der Konfigurationen 17 bis 28.
Hier wurde in allen \glspl{Test} der Node 4 im jeweils ersten \gls{Testfall} direkt vom Cluster getrennt, was noch korrekt erkannt wurde.
In den \glspl{Testkonfiguration} 25 bis 28 wurde im nachfolgenden dritten \gls{Testfall} jedoch der Node direkt wieder vom Netzwerk getrennt, weshalb hier nur vermutet werden kann, dass der Node im zweiten \gls{Testfall} korrekt mit dem Cluster verbunden wurde.
Davon kann jedoch ausgegangen werden, da in den sechs \glspl{Test} auf einem Host (Tests 17 bis 22) der Node im nachfolgenden, dritten \gls{Testfall} nicht verändert wird und auch als aktiv erkannt wurde.
Auch in den \glspl{Test} 29 bis 32 wurde alles korrekt erkannt, da hier der Node im ersten \gls{Testfall} ebenfalls getrennt, im zweiten wieder verbunden, und im dritten \gls{Testfall} erneut vom Cluster getrennt wird.
In den \glspl{Test} 23 bis 28.2 wird der Node ebenfalls im dritten \gls{Testfall} wieder vom Cluster getrennt, was hier auch korrekt erkannt wird.


\section{Rekonfiguration des Clusters nicht möglich}
\label{sec:noReconfig}

Insgesamt 13 der 42 ausgeführten \glspl{Test} wurden vorzeitig abgebrochen, da eine Rekonfiguration des Clusters nicht möglich war.
Dies entspricht dem in \cref{subsec:testRequirements} gefordertem und in \cref{subsec:oracleImpl} implementierten Verhalten, wenn alle Node im Cluster defekt sind.
Im Folgenden wird für die betroffenen \glspl{Testkonfiguration} und \glspl{Test} (vgl. \cref{app:overviewExecutedTestCases}) betrachtet, weshalb es dazu kam.

\subsection{Testkonfigurationen 3 bis 6}
\label{subsec:noReconf36}

Erstmalig ist ein Abbruch im \gls{Test} 4 aufgetreten, auch die weiteren korrespondierende \glspl{Test} der Konfigurationen 5 und 6 wurden abgebrochen.
Hier waren bereits beim 3. \gls{Testfall} alle verfügbaren Nodes beendet, was auffällig ist, vor allem da damit die Hälfte der \glspl{Test} mit dem ersten Seed und dem Cluster auf einem Host vorzeitig abgebrochen wurden.
Das liegt einerseits daran, dass im Gegensatz zu den beiden Konfigurationen mit nur zwei Clients hier bis zu vier \glspl{Anwendung} gleichzeitig gestartet werden, was die Last auf den Nodes deutlich erhöht.
In \gls{Test} 3, welcher somit theoretisch ebenfalls abgebrochen werden hätte müssen, wurden 11 \glspl{Anwendung} im Cluster gestartet.
Dies liegt an der geringeren Auslastung eines einzelnen Nodes im Gegensatz zu den anderen Tests.
In den abgebrochenen \glspl{Test} hatte Node 4 im ersten \gls{Testfall} eine hohe bzw. sehr hohe Auslastung, im \gls{Test} 3 jedoch nur eine mittlere.
Diese mittlere Auslastung reichte jedoch aus, um den Node im ausgeführten 3. \gls{Testfall} gemäß \cref{subsec:faultActivation} wieder zu aktivieren, während die anderen noch aktiven Nodes spätestens in diesem \gls{Testfall} aufgrund der hohen Last einen Komponentenfehler injiziert bekamen.
Durch diesen einen nun weiterhin ausgeführten Node ist es dem Cluster daher möglich gewesen, sich im \gls{Test} 3 zu rekonfigurieren.

\subsection{Testkonfigurationen 15 und 16}
\label{subsec:noReconf1516}

Die Ausführung der \glspl{Test} 13 bzw. 14 und 15 bzw. 16 unterscheidet sich nur in der Anzahl der \glspl{Testfall} der jeweiligen Testkonfiguration.
Dementsprechend wurden die äquivalenten \glspl{Test} 13 und 14 im Gegensatz zu den beiden anderen vollständig ausgeführt, da der Abbruch der \glspl{Test} 15 und 16 im sechsten ausgeführten \gls{Testfall} stattfand.
Die Nodes hatten im fünften \gls{Testfall} der vier \glspl{Test} folgende Auslastung:

\begin{table}[h]
    \begin{tabular}{c|cccc}
    	        \gls{Test}          & 13 & 14 & 15 & 16 \\ \hline
    	  Fehlerhafte Nodes   & 2  & 2  & 3  & 1  \\
    	Auslastung in Prozent & 47 & 97 & 96 & 98
    \end{tabular}
    \caption[Status der Nodes im fünften \gls{Testfall} der \glspl{Test} 13 bis 16]
        {Status der Nodes im fünften \gls{Testfall} der \glspl{Test} 13 bis 16.
        Der Wert der Auslastung ist die kumulierte Auslastung aller noch aktiven Nodes.}
    \label{tab:loadTests1316}
\end{table}

Bei den beiden betroffenen \glspl{Test} 15 und 16 sehr hohe Auslastung der noch aktven Nodes im fünften \gls{Testfall} führte im darauf folgenden \gls{Testfall} dazu, dass bei allen noch aktiven Nodes ein Komponentenfehler aktiviert wurde.
Daher wurden die beiden \glspl{Test} im jeweils sechsten ausgeführten \gls{Testfall} abgebrochen.
Es ist auch davon auszugehen, dass der \gls{Test} 14 aufgrund der ebenfalls sehr hohen Auslastung im sechsten \gls{Testfall} wahrscheinlich ebenfalls abgebrochen worden wäre.

\subsection{Testkonfigurationen 19 bis 22}
\label{subsec:noReconf1922}

Bei den Konfigurationen 19 bis 22 verhält es sich ähnlich wie bei den Konfigurationen 3 bis 6.
Analog dazu wurde auch \gls{Test} 19 nicht vorzeitig abgebrochen, die \glspl{Test} 20 bis 22 im vierten \gls{Testfall} dagegen schon.

Alle vier \glspl{Test} haben gemeinsam, dass im jeweils dritten \gls{Testfall} lediglich Node 1 inaktiv ist.
Bei den beiden \glspl{Test} ohne Mutationsszenario wurde hierbei jeweils die Verbindung zum Node im \gls{Testfall} zuvor getrennt, bei den Mutationstests wurde der Node durch einen Komponentenfehler beendet.
Dies liegt in der Historie des Nodes innerhalb des \glspl{Test} begründet:

\begin{table}[h]
    \begin{tabu}{c|[1.5pt]cccc}
    	   \gls{Test}    &                       19                       &                  20                  &                     21                      &                  22                  \\ \tabucline[1.5pt]{-}
    	Testfall 1 &                  Ausl.: 93 \%                  &             Ausl.: 0 \%              &                Ausl.: 100 \%                &             Ausl.: 0 \%              \\ \hline
    	Testfall 2 &  \makecell{Injiziert:\\Verbindung\\getrennt}   &            Ausl.: 100  \%            & \makecell{Injiziert:\\Verbindung\\getrennt} &            Ausl.: 93  \%             \\ \hline
    	Testfall 3 &                       -                        & \makecell{Injiziert:\\Node\\beendet} &                      -                      & \makecell{Injiziert:\\Node\\beendet} \\ \hline
    	Testfall 4 & \makecell{Repariert:\\Verbunden\\Ausl.: 93 \%} &                  -                   &   \emph{\makecell{Repariert:\\Verbunden}}   &                  -
    \end{tabu} 
    \caption{Auslastungen und Komponentenfehler in Node 1 der \glspl{Test} 19 bis 22}
    \label{tab:loadNode1Tests1922}
\end{table}

Die Aktivierung und Deaktivierung der Komponentenfehler in den anderen Nodes ist in allen \glspl{Test} gleich und daher zur Ermittlung der Gründe des Abbruchs der Testausführung nicht relevant.
Durch die unterschiedliche Auslastungen im ersten \gls{Testfall} der \glspl{Test} zwischen \glspl{Test} ohne Mutationen (19 und 21) und mit Mutationen (20 und 22) wurden unterschiedliche Komponentenfehler aktiviert.
Dies führte dazu, dass der relevante Node 1 bei den Mutationsstests nicht gestartet wurde, während die anderen Nodes beendet wurden wie in den \glspl{Test} 19 und 21.

Eine Besonderheit bildet hier zudem \gls{Test} 21, bei dem der Komponentenfehler vom Testsystem deaktiviert wurde, jedoch nicht repariert werden konnte.
Dies liegt darin, dass der Docker"=Container nicht mit dem Docker"=Netzwerk verbunden werden konnte.
Aus diesem Grund wurde vom Oracle bei der Prüfung der Rekonfigurierbarkeit des Clusters der \gls{Test} entsprechend beendet, da der Node nicht verbunden war.
Zwar wurde der Fehler von Docker nicht absichtlich oder durch das Testsystem herbeigeführt, hat jedoch eine positive, als auch eine negative Seite.
So wurde auch ein externer, nicht direkt durch die Anforderungen in \cref{subsec:testRequirements} abgedeckter Fehler erkannt, jedoch auf Kosten der Anforderung, dass im Modell implementierte Komponentenfehler im realen Cluster repariert werden.

\subsection{Testkonfigurationen 27 und 28}
\label{subsec:noReconf2728}

In den beiden \glspl{Test} 28.1 und 28.2 wird der \gls{Test} im 8. \gls{Testfall} abgebrochen, während \gls{Test} 27 nach allen 10 Testfällen regulär beendet wird.
Das liegt daran, dass im 8. \gls{Testfall} bei den beiden Mutationstests in fünf der sechs Nodes ein Komponentenfehler injiziert wird, von Node 1 wird die Verbindung getrennt, die Nodes 3 bis 6 komplett beendet.
Im \gls{Test} 27 ohne Mutationsszenario wird dagegen zwar auch die Verbindung von Node 1 getrennt, aber zusätzlich nur Node 3 beendet, sodass die Nodes 4 bis 6 weiterhin aktiv sind.
Node 2 wird in allen drei \glspl{Test} bereits im dritten \gls{Testfall} beendet, da die Auslastung des Nodes im zweiten \gls{Testfall} bei jeweils über 90 Prozent liegt.
Die übrigen der 19 bzw. 20 aktivierten und zwischen 10 und 13 wieder deaktivierten Komponentenfehlern unterschieden sich in den drei \glspl{Test} bis auf einzelne, hier nicht relevante, Ausnahmen nicht.

\todo{kürzen und die genaue erklärung auf entsprechende abschnitte verweisen}
Der Grund für die Injizierung von Komponentenfehlern bei noch allen aktiven Nodes im achten \gls{Testfall} liegt in der Auslastung der Nodes im siebten Testfall.
Diese beträgt im \gls{Test} 27 ohne Mutationen bei den beiden betroffenen Nodes jeweils 100 Prozent, bei den übrigen Nodes ist jedoch keine bzw. eine geringe Auslastung vorhanden.
In den beiden \glspl{Test} der Konfiguration 28 ist das Cluster jeweils vollständig ausgelastet, wodurch die Wahrscheinlichkeit zur Aktivierung der Komponentenfehler im folgenden \gls{Testfall} stark ansteigt.
Dadurch war es möglich, dass alle noch aktiven Nodes vom Cluster getrennt bzw. beendet wurden und der \gls{Test} aufgrund fehlender Rekonfigurationsmöglichkeiten abgebrochen wurde.

\subsection{Testkonfigurationen 31 und 32}
\label{subsec:noReconf3132}

Die \glspl{Test} der Konfigurationen 31 und 32 verliefen ähnlich zueinander.
Die hohe Anzahl der 19 bzw. 20 aktivierten Komponentenfehler reichten bei jeweils 11 wieder deaktivierten Fehlern aus, um die \glspl{Test} 31.2 und 32 im achten \gls{Testfall} abzubrechen.
Der \gls{Test} 31.1 verlief zwar ebenfalls ähnlich zu den beiden anderen Tests, wurde jedoch aufgrund fehlender, verfügbaren Submitter des Connectors beendet.
Die Gründe dafür sind in \cref{subsec:notStartedApps} erläutert, weshalb der \gls{Test} 31.1 hier nicht genauer betrachtet wird.

Bei den beiden \glspl{Test} 31.2 und 32 fällt auf, dass es bei jeweils mehreren Testfällen vorgekommen ist, dass mehr als 3 Komponentenfehler aktiviert bzw. deaktiviert wurden.
So kam es vor, dass \zB in dritten ausgeführten \gls{Testfall} bereits eine Rekonfiguration nur deshalb möglich war, weil der zuvor vom Cluster getrennte Node 1 wieder mit dem Cluster verbunden wurde, während die Nodes 2 und 4 bis 6 anderen Nodes getrennt oder beendet wurden, während Node 3 bereits im \gls{Testfall} zuvor beendet wurde.
Ebenso verlief der dritte \gls{Testfall} auch in den beiden \glspl{Test} der Konfigurationen 29 und 30, bei denen nur fünf \glspl{Testfall} ausgeführt wurden.

Bis auf den beendeten Node 2 wurden spätestens im sechsten ausgeführten \gls{Testfall} die im dritten \gls{Testfall} injizierten Komponentenfehler wieder repariert.
Zwar wurde im siebten \gls{Testfall} je ein Komponentenfehler repariert, jedoch im \gls{Test} ohne Mutationen auch ein weiterer injiziert.
In Kombination mit den drei bzw. vier aktivierten Komponentenfehlern im achten \gls{Testfall} führte das daher dazu, dass kein aktiver Node im Cluster mehr vorhanden war und der \gls{Test} entsprechend abgebrochen wurde.


\section{Betrachtung der Anwendungen}
\label{sec:appEval}

Bei der Betrachtung der \glspl{Anwendung} sind vor allem zwei Punkte aufgefallen:
Viele \glspl{Anwendung} wurden aufgrund von Fehlern beendet und einige Anwendungen, vor allem beim zweiten Seed, konnten nicht gestartet werden.
Dies widerspricht zum Teil den in \cref{sec:requirements} definierten Anforderungen, hat aber mehrere Gründe, die im Folgenden erläutert werden.

\subsection{Aufgrund von Fehlern abgebrochene Anwendungen}
\label{subsec:failedApps}

Wie bereits erwähnt, sind etwas mehr als ein viertel aller gestarteten \glspl{Anwendung} gefailt, was im Schnitt 2,6 gefailte \glspl{Anwendung} pro ausgeführten \gls{Test} ergibt.
Die meisten gefailten \glspl{Anwendung} sind hierbei mit 9 bzw. 8 bei den \glspl{Test} der Konfigurationen 31 und 32 zu finden.
Auffällig ist zudem der Vergleich zwischen den \glspl{Test} 19 und 20.
Während bei der Ausführung der \gls{Testkonfiguration} 19 ganze 5 \glspl{Anwendung} gefailt sind, ist bei der Ausführung des \glspl{Test} 20 keine einzige gefailt.
Ebenfalls auffallend ist, dass bei Konfigurationen mit Mutationsszenario fast immer weniger oder gleich viele \glspl{Anwendung} gefailt sind als bei den korrespondierenden Konfigurationen ohne Mutationsszenario.
Eine Ausnahme bildet der \gls{Test} 8, bei dem 3 \glspl{Anwendung} gefailt sind, während bei den \glspl{Test} 7.1 und 7.2 jeweils keine \gls{Anwendung} gefailt ist.
Eine weitere Ausnahme bildet der \gls{Test} 9.2, bei dem eine \gls{Anwendung} mehr gefailt ist als im \gls{Test} 10.3, die restlichen \glspl{Test} der Konfigurationen 9 und 10 verhalten sich jedoch wie andere korrespondierende Testkonfigurationen.

Bei der Betrachtung der Constraints, welche die in \cref{subsec:functionalRequirements} definierte Anforderung umsetzen, dass \glspl{Anwendung} vollständig ausgeführt werden, solange sie nicht manuell bzw. durch das Testsystem vorzeitig abgebrochen werden, fällt auf, dass die Anzahl der ungültigen Validierungen durch das Oracle mit kumuliert 343 ungültigen Constraints mehr als die Hälfte aller ungültigen Constraints ausmacht (59,9 \% ).
Im Schnitt ergibt das somit rund 8 ungültige Constraints pro \gls{Test} bzw. ca. 1,8 ungültige Constraints pro durch das Oracle überprüften Testfall.
Die auf den ersten Blick sehr hohe Anzahl an ungültigen Constraints resultiert daraus, dass eine gefailte \gls{Anwendung} bei jedem nachfolgenden \gls{Testfall} bei einer Testausführung erneut durch das Oracle entsprechend validiert wurde.
Dadurch sind ein Großteil der als ungültig validierten Constraints ein falscher Alarm, da die entsprechende Anforderung pro \gls{Anwendung} nur einmal nicht erfüllt werden kann.
Aussagekräftiger ist daher die Anzahl von 110 nicht vollständig abgeschlossenen bzw. aufgrund eines Fehlers abgebrochenen Anwendungen.

Anhand der Datenbasis lassen sich vier Ursachen für nicht vollständig ausgeführte \glspl{Anwendung} ausmachen:

\begin{itemize}
    \item
        Der \gls{AppMstr} ist nicht mehr erreichbar, da der auszuführende Node aufgrund eines Komponentenfehlers nicht mehr erreichbar ist.
        Dadurch wird der \gls{AppMstr} nach einiger Zeit mit dem Fehler \emph{\gls{AppMstr}"=Timeout} als abgebrochen markiert.
    \item
        Die den \gls{AppMstr} zugewiesenen Nodes sind vollständig ausgelastet, wodurch dem \gls{AppMstr} selbst die benötigten Ressourcen nicht allokiert bzw. der \gls{AppMstr} nicht ausgeführt werden kann.
        Nach einiger Zeit wird der \gls{AppMstr} daher mit dem Fehler \emph{\gls{AppMstr}"=Timeout} abgebrochen.
        Das beinhaltet auch Timeouts, wenn einem \gls{AppMstr} nicht einmal ein ausführender Node zugewiesen werden kann.
    \item
        Während der Ausführung einer \gls{MR}\gls{Anwendung} wird ein Fehler im Map"=Task festgestellt, der dazu führt, dass der Task abgebrochen wird.
        Dieser Fehler kam bei den hier ausgeführten \glspl{Test} bei der \gls{Anwendung} \acrlong{dfr} vor, wenn die zuvor generierten Eingabedaten für diesen Benchmark aufgrund aktivierter Komponentenfehler nicht mehr im Cluster vorhanden waren.
        Zwar werden Dateien im \gls{HDFS} immer auf mehr als einem Node gespeichert (vgl. \cref{sec:hadoop}), jedoch ist es möglich, dass die für die \gls{Anwendung} benötigten Daten auf Nodes repliziert wurden, die alle beendet wurden.
        Dies führte dazu, dass die benötigten Daten nicht gefunden werden können bzw. bereits im \gls{HDFS} als fehlerhaft markiert sind.
        Dadurch wird im Map"=Task ein Fehler ausgelöst, der die gesamte \gls{Anwendung} vorzeitig beendet.
        Aufgrund eines Fehlers im Map"=Task wird auch die \acrlong{fl}\gls{Anwendung} beendet, jedoch ist das in diesem Fall das gewünschte Verhalten der \gls{Anwendung} und zählt daher nicht als Fehler.
    \item
        Der \gls{AppMstr} eines \glspl{Attempt} wird mit dem Exitcode -100 beendet.
        Dieser Fehler kommt dann vor, wenn versucht wird, einen Task eines Anwendungs"=Containers der jeweiligen \gls{Anwendung} bzw. \gls{Attempt} auf einem defekten Node auszuführen und widerspricht somit zusätzlich der in \cref{subsec:functionalRequirements} Anforderung, dass kein Task oder \gls{Anwendung} an defekte Nodes gesendet wird.
        Dieser Fehler trat nur dann auf, wenn im ausführende Node des betroffenen \gls{AppMstr} im gleichen \gls{Testfall} ein Komponentenfehler injiziert wurde und der Node dadurch ausfiel.
        Aufgrund der mit dem Fehler verbundene Fehlermeldung \textit{\enquote{Container released on a *lost* node}} liegt die Vermutung nahe, dass Anwendungs"=Container, hier wahrscheinlich der \gls{AppMstr}, zum Zeitpunkt der Fehlerinjizierung bereits abgeschlossen waren und das Cluster die benötigten Ressourcen zu dem Zeitpunkt freigegeben hat.
        Da dies jedoch nicht möglich war, wurde der \gls{AppMstr} mit dem entsprechenden Fehler beendet.
\end{itemize}

Hierbei werden aufgrund eines \gls{AppMstr}"=Timeouts zunächst nur die \glspl{Attempt} mit dem entsprechenden Fehler abgebrochen, nicht jedoch die \gls{Anwendung} selbst.
Die \gls{Anwendung} selbst wird in so einem Fall erst dann als gefailt abgebrochen, sobald zwei \glspl{Attempt} aufgrund eines Timeouts abgebrochen werden mussten.
Wenn ein \gls{Attempt} mit dem Exitcode -100 terminiert, wird unabhängig von zuvor ausgeführten \glspl{Attempt} ein erneuter \gls{Attempt} mit entsprechendem \gls{AppMstr} gestartet, wodurch hier die Anforderung, dass ein Task vollständig ausgeführt werden muss, teilweise erfüllt werden kann.

Bei einigen der \gls{AppMstr}"=Timeouts aufgrund der Aktivierung von Komponentenfehler lässt sich zudem ein spezielles Muster erkennen.
Hierbei wurde in einem zuvor ausgeführten \gls{Testfall} auf einem Node ein \gls{AppMstr} einer \gls{Anwendung} ohne Fehler allokiert.
Nun kann es passieren, dass für diesen Node ein Komponentenfehler injiziert wird, was dazu führt, dass der Node nicht mehr erreichbar ist und der \gls{AppMstr} aufgrund eines Timeouts als beendet markiert wird.
Hierbei wird direkt im Anschluss ein neuer \gls{AppMstr} allokiert, was auch dazu führt, dass die \gls{Anwendung} nun einen zweiten \gls{Attempt} besitzt, nachdem der erste aufgrund des Timeouts abgebrochen wurde.
Dabei ist es nun möglich, dass dies noch während der Aktivierung von Komponentenfehlern innerhalb des Testfalls geschieht (vgl. \cref{subsec:simulationStep}), wodurch es möglich ist, dass der auszuführende Node des zweiten \gls{AppMstr} ebenfalls aufgrund eines im gleichen \gls{Testfall} injizierten Komponentenfehlers nicht mehr erreichbar ist.
Dadurch wird der zweite \gls{AppMstr} bzw. \gls{Attempt} aufgrund des Timeouts vorzeitig als abgebrochen markiert und die gesamte \gls{Anwendung} dadurch abgebrochen.

\subsection{Nicht gestartete Anwendungen}
\label{subsec:notStartedApps}

Bei den \glspl{Test} 19, 25 und 27 bis 32 kam es vor, dass insgesamt 29 \glspl{Anwendung} nicht gestartet werden konnten.
Meistens war die \gls{Anwendung} \acrlong{tsr} davon betroffen, einige male die \gls{Anwendung} \acrlong{tvl}.
Ursächlich dafür ist die jeweils hohe Auslastung des Clusters in den Testfällen zuvor, bei denen den benötigten \gls{AppMstr} der \acrlong{tg}\gls{Anwendung}en keine Ressourcen auf den ausführenden Nodes allokiert werden konnte und diese daher mit einem \gls{AppMstr}"=Timeout beendet wurden (vgl. \cref{subsec:failedApps}).
Da in \cref{sec:selectTestcases} definiert wurde, dass benötigte Eingabedaten für \glspl{Anwendung} während der Ausführung der \glspl{Test} generiert werden, konnten so die benötigten Eingabedaten für die \gls{Anwendung} \acrlong{tsr} nicht generiert werden (vgl. \todo{details zu tsort}).
Aufgrund der fehlenden Daten wurde daher die \gls{Anwendung} direkt wieder abgebrochen, wodurch in 42 Testfällen nicht jeder Client eine \gls{Anwendung} ausgeführt hat.
In diesen Fällen wurde als Resultat zudem das Constraint der Anforderung aus \cref{subsec:testRequirements}, wonach mehrere Benchmark\gls{Anwendung}en gleichzeitig gestartet und ausgeführt werden können, aufgrund der Implementation aus \todo{Constraint-Impl} durch das Oracle als ungültig validiert.
Analog dazu verhält es sich bei der \gls{Anwendung} \acrlong{tvl}, welche wiederum die \acrlong{tsr}"=Ausgabedaten als Eingabedaten benötigt (vgl. \todo{details zu tvalidate}).

\subsection{Nicht ausreichend Submitter}
\label{subsec:notEnoughSubmitter}

Ein unerwarteter Fehler trat bei der Ausführung des Testfalls 31.1 auf.
Hierbei kam es vor, dass die für die anderen \glspl{Test} genutzten acht Submitter des Connectors zum Starten von \glspl{Anwendung} (vgl. \cref{subsec:implementedConnectors}) nicht ausreichend waren.
Der \gls{Test} wurde hierbei im achten \gls{Testfall} abgebrochen, weil keine weiteren freien Submitter zur Verfügung standen.
Daher wurde der \gls{Test} zur Konfiguration 31 mit zehn Submittern erneut ausgeführt, wodurch dieser wie in \cref{subsec:noReconf3132} erläutert im achten \gls{Testfall} aufgrund fehlender Rekonfigurierbarkeit abgebrochen wurde.
Die Gründe für den Abbruch des \glspl{Test} 31.1 liegen darin, dass die Docker"=Container der Benchmarks (vgl. \cref{sec:realCluster}) nicht korrekt beendet wurden und die Submitter daher auf weitere Ausgaben der gestarteten \glspl{Anwendung} gewartet haben.
\todo{Warten der submitter (auch im Treiber) nochmal genauer erklären}


\section{Nicht erkannte oder gespeicherte Daten des Clusters}
\label{sec:notDetectedData}

Einige Daten des Clusters wurden nicht im Testsystem gespeichert bzw. im Programmlog ausgegeben.
Dies verstößt damit gegen die in \autoref{sec:predictions} definierte Anforderung an das Testsystem, dass der jeweils aktuelle Status des Clusters erkannt und im Modell gespeichert werden muss.
Vorgekommen ist das auf zwei Arten, die im folgenden erläutert werden.

\subsection{Nicht erkannte Nodes auf Host 2}
\label{sec:notDetectedHost2}

Einer der beiden Fälle ist, dass ausführende Nodes von Anwendungen bzw. Attempts nicht erkannt bzw. ausgegeben wurden, wodurch vom Oracle auch Verletzungen gegen die Anforderung aus \autoref{sec:clusterRequirements} und \autoref{sec:predictions} erkannt wurden, wonach die Konfiguration des Clusters aktualisiert, und der aktuelle Status im Cluster erkannt und im Testmodell gespeichert werden muss.
Hier geht es jedoch nicht um Anwendungen bzw. Attempts, die zwar bereits gestartet wurden, für die aber noch kein \ac{AppMstr} allokiert werden konnte.
In diesen Fällen ist es daher das normale Verhalten von Hadoop, keinen ausführenden Node anzugeben, da keiner vorhanden ist.
Wenn dieser Status zu lange anhält, wurden die Attempts bzw. \ac{AppMstr} durch Hadoop mit einem Timeout beendet (vgl. \autoref{sec:failedApps}).

Anders sieht das jedoch in den sechs Tests 7.1, 8 und 23 bis 26 aus.
In diesen Tests wurden zwar regulär die Daten der Nodes ermittelt und auch in den Logdateien ausgegeben, jedoch nicht alle ausführenden Nodes von Anwendungen und Attempts.
Konkret betrifft das hier die beiden auf Host 2 ausgeführten Nodes der betroffenen \ac{AppMstr}.
In allen sechs betroffenen Tests wurden nur die vier auf Host 1 ausgeführten Nodes als ausführende Nodes der Attempts bzw. Anwendungen erkannt und auch in den Logdateien ausgegeben.
Die auf Host 2 ausgeführten Nodes wurden gemäß des SSH"=Logs allerdingsn ebenfalls übertragen, sofern den Attempts bzw. Anwendungen ein Node zugewiesen wurde, jedoch wurden diese nicht im Programmlog ausgegeben.
Zwar tritt hierbei ein gewisses Muster auf (pro Seed die jeweils zuerst ausgeführten Tests mit Nodes auf beiden Hosts), allerdings konnte dieser Fehler nicht gezielt reproduziert werden.
Bei der erneuten Ausführung der Testkonfiguration 7 (Test 7.2) wurden alle Nodes korrekt erkannt und vom Testsystem im Programmlog gespeichert.
Zum gegenwärtigen Zeitpunkt kann daher nicht gesagt werden, weshalb die ausführenden Nodes in den betroffenen Testfällen nicht immer gespeichert wurden.
Es kann nur vermutet werden, dass während dem Parsen der übertragenen Daten mit diesen Daten die betroffenen Nodes im Modell nicht gefunden werden konnten (vgl. \todo{Parser/Node-Speicherung, da erklären, dass Objekt vom Node gespeichert wird und nicht ID}).
Dennoch lässt sich sagen, dass die beiden verletzten Anforderungen nach einer genaueren Begutachtung der Gründe dafür ein falscher Alarm des Oracles war.

\subsection{Diagnostic"=Daten von Anwendungen}
\label{sec:notSavedAppDiagnostics}

Bei allen Tests ist zudem aufgefallen, dass die Diagnostic"=Daten von Anwendungen nicht im Programmlog enthalten sind.
Genauso wie bei den nicht erkannten Nodes auf Host 2 (vgl. \autoref{sec:notDetectedHost2}) wurden alle Diagnostic"=Daten von Hadoop an das Testsystem übertragen, die der Anwendungen im Gegensatz zu denen der Attempts jedoch nicht gespeichert.
Zur Auswertung der Daten im Rahmen der Evaluation ist dies zwar irrelevant, da dies auch aufgrund der Daten der Attempts geschehen konnte, allerdings wird dadurch die in \autoref{sec:predictions} definierte Anforderung an das Testsystem nur teilweise erfüllt, wonach der jeweils aktuelle Status des Clusters erkannt und gespeichert wird.

Eine Analyse ergab, dass die Diagnostic"=Daten der Anwendungen aufgrund eines falsch gesetzten Attributs in der \texttt{ApplicationResult}"=Klasse des Parsers nicht im Testsystem gespeichert werden konnten. \todo{Reflexion: Hätte bei vorabtests erkannt werden müssen!}
Da die Diagnostic"=Daten der Anwendungen eine Zusammenfassung der gesamten Anwendung darstellen, und alle Diagnostic"=Daten bereits durch die der Attempts vorhanden waren, wurde hier auf erneute Testausführungen verzichtet.

