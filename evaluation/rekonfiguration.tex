\section{Analyse der Testabbrüche}
\label{sec:noReconfig}

Insgesamt 13 der 42 ausgeführten \glspl{Test} wurden vorzeitig abgebrochen, da eine Rekonfiguration des Clusters nicht möglich war.
Dies entspricht dem in \cref{subsec:testRequirements} gefordertem und in \cref{subsec:oracleImpl} implementierten Verhalten, wenn alle Nodes im Cluster aufgrund eines Komponentenfehlers defekt sind.
Im Folgenden wird für die betroffenen Konfigurationen und Tests (vgl. \cref{app:overviewExecutedTestCases}) betrachtet, weshalb es dazu kam.

\subsection{Testkonfigurationen 3 bis 6}
\label{subsec:noReconf36}

Erstmalig ist ein Abbruch im \gls{Test} 4 aufgetreten, auch die weiteren korrespondierende \glspl{Test} der Konfigurationen 5 und 6 wurden abgebrochen.
Hier waren bereits beim dritten ausgeführten \gls{Testfall} alle verfügbaren Nodes beendet, was auffällig ist, vor allem da somit auch die Hälfte aller \glspl{Test} mit dem ersten Seed und dem Cluster auf einem Host vorzeitig abgebrochen wurden.
Das liegt primär darin begründet, dass im Gegensatz zu den beiden Konfigurationen mit nur zwei Clients hier bis zu vier \glspl{Anwendung} gleichzeitig gestartet werden, was die Last auf den Nodes deutlich erhöht.
In \gls{Test} 3, welcher somit theoretisch ebenfalls abgebrochen werden hätte müssen, wurden 11 \glspl{Anwendung} im Cluster gestartet.
Dies liegt an der geringeren Auslastung eines einzelnen Nodes im Gegensatz zu den anderen Tests.
In den abgebrochenen \glspl{Test} hatte Node 4 im ersten \gls{Testfall} eine hohe bzw. sehr hohe Auslastung, im \gls{Test} 3 jedoch nur eine mittlere.
Diese mittlere Auslastung reichte jedoch aus, um den Node im dritten ausgeführten \gls{Testfall} wieder zu aktivieren, während bei den anderen noch aktiven Nodes spätestens in diesem \gls{Testfall} aufgrund der hohen Last ein Komponentenfehler injiziert wurde (vgl. \cref{subsec:faultActivation}).
Durch diesen einen nun weiterhin ausgeführten Node ist es dem Cluster daher möglich gewesen, sich im \gls{Test} 3 zu rekonfigurieren.

\subsection{Testkonfigurationen 15 und 16}
\label{subsec:noReconf1516}

Die Ausführung der \glspl{Test} 13 bzw. 14 und 15 bzw. 16 unterscheidet sich nur in der Anzahl der \glspl{Testfall} der jeweiligen Testkonfiguration.
Dementsprechend wurden die äquivalenten \glspl{Test} 13 und 14 im Gegensatz zu den beiden anderen vollständig ausgeführt, da der Abbruch der \glspl{Test} 15 und 16 im sechsten ausgeführten \gls{Testfall} stattfand.
Die Nodes hatten im fünften \gls{Testfall} der vier \glspl{Test} folgende Auslastung:

\begin{table}[h]
    \begin{tabular}{c|cccc}
    	        \gls{Test}          & 13 & 14 & 15 & 16 \\ \hline
    	  Fehlerhafte Nodes   & 2  & 2  & 3  & 1  \\
    	Auslastung in Prozent & 47 & 97 & 96 & 98
    \end{tabular}
    \caption[Status der Nodes im fünften \gls{Testfall} der \glspl{Test} 13 bis 16]
    {Status der Nodes im fünften \gls{Testfall} der \glspl{Test} 13 bis 16.
    Der Wert der Auslastung ist die kumulierte Auslastung aller noch aktiven Nodes.
    Eine Übersicht aller Tests findet sich in \cref{app:overviewExecutedTestCases}.}
    \label{tab:loadTests1316}
\end{table}

Bei den beiden betroffenen \glspl{Test} 15 und 16 führte die sehr hohe Auslastung der noch aktven Nodes im fünften \gls{Testfall} daher im darauf folgenden \gls{Testfall} dazu, dass bei allen noch aktiven Nodes ein Komponentenfehler injiziert wurde.
Daher wurden die beiden \glspl{Test} im jeweils sechsten ausgeführten \gls{Testfall} abgebrochen.
Es ist auch davon auszugehen, dass der \gls{Test} 14 aufgrund der ebenfalls sehr hohen Auslastung im sechsten \gls{Testfall} wahrscheinlich ebenfalls abgebrochen worden wäre.

\subsection{Testkonfigurationen 19 bis 22}
\label{subsec:noReconf1922}

Bei den Konfigurationen 19 bis 22 verhält es sich ähnlich wie bei den Konfigurationen 3 bis 6.
Analog dazu wurde auch \gls{Test} 19 nicht vorzeitig abgebrochen, die \glspl{Test} 20 bis 22 im vierten \gls{Testfall} dagegen schon.

Alle vier \glspl{Test} haben gemeinsam, dass im jeweils dritten \gls{Testfall} lediglich Node 1 inaktiv ist.
Bei den beiden \glspl{Test} ohne Mutationsszenario wurde hierbei jeweils die Verbindung zum Node im \gls{Testfall} zuvor getrennt, bei den Mutationstests wurde der Node durch einen Komponentenfehler beendet.
Dies liegt in der Historie des Nodes innerhalb des \glspl{Test} begründet:

\begin{table}[h]
    \begin{tabu}{c|[1.5pt]cccc}
    	   \gls{Test}    &                       19                       &                  20                  &                     21                      &                  22                  \\ \tabucline[1.5pt]{-}
    	Testfall 1 &                  Ausl.: 93 \%                  &             Ausl.: 0 \%              &                Ausl.: 100 \%                &             Ausl.: 0 \%              \\ \hline
    	Testfall 2 &  \makecell{Injiziert:\\Verbindung\\getrennt}   &            Ausl.: 100  \%            & \makecell{Injiziert:\\Verbindung\\getrennt} &            Ausl.: 93  \%             \\ \hline
    	Testfall 3 &                       -                        & \makecell{Injiziert:\\Node\\beendet} &                      -                      & \makecell{Injiziert:\\Node\\beendet} \\ \hline
    	Testfall 4 & \makecell{Repariert:\\Verbunden\\Ausl.: 93 \%} &                  -                   &   \emph{\makecell{Repariert:\\Verbunden}}   &                  -
    \end{tabu} 
    \caption[Auslastung und Komponentenfehler in Node 1 der Tests 19 bis 22]
    {Auslastung und Komponentenfehler in Node 1 der Tests 19 bis 22.
    Eine Übersicht aller Tests findet sich in \cref{app:overviewExecutedTestCases}.}
    \label{tab:loadNode1Tests1922}
\end{table}

Die Aktivierung und Deaktivierung der Komponentenfehler bei den anderen Nodes ist in allen \glspl{Test} gleich und daher zur Ermittlung der Gründe des Abbruchs der Testausführung nicht relevant.
Durch die unterschiedliche Auslastung im ersten \gls{Testfall} der \glspl{Test} zwischen \glspl{Test} ohne Mutationen (19 und 21) und mit Mutationen (20 und 22) wurden unterschiedliche Komponentenfehler aktiviert.
Dies führte dazu, dass der relevante Node 1 bei den Mutationsstests nicht gestartet wurde, während die anderen Nodes beendet wurden wie in den \glspl{Test} 19 und 21.

Eine Besonderheit bildet hier zudem \gls{Test} 21, bei dem der Komponentenfehler vom Testsystem deaktiviert wurde, jedoch nicht repariert werden konnte.
Dies liegt darin, dass der Docker"=Container nicht mit dem Docker"=Netzwerk verbunden werden konnte.
Aus diesem Grund wurde vom Oracle bei der Prüfung der Rekonfigurierbarkeit des Clusters der \gls{Test} entsprechend beendet, da der Node nicht verbunden war.
Zwar wurde der Fehler von Docker nicht absichtlich bzw. durch das Testsystem herbeigeführt, hat jedoch eine positive, als auch eine negative Seite.
So wurde auch ein externer, nicht spezifizierter Fehler erkannt, jedoch auf Kosten der Anforderung, dass im Modell implementierte Komponentenfehler im realen Cluster repariert werden (vgl. \cref{sec:requirements}).

\subsection{Testkonfigurationen 27 und 28}
\label{subsec:noReconf2728}

In den beiden \glspl{Test} 28.1 und 28.2 wird der \gls{Test} im 8. \gls{Testfall} abgebrochen, während \gls{Test} 27 nach allen 10 Testfällen regulär beendet wird.
Das liegt daran, dass im achten \gls{Testfall} bei den beiden Mutationstests in fünf der sechs Nodes ein Komponentenfehler injiziert wird: von Node 1 wird die Verbindung getrennt, die Nodes 3 bis 6 werden komplett beendet.
Im \gls{Test} 27 ohne Mutationsszenario wird dagegen zwar auch die Verbindung von Node 1 getrennt, aber zusätzlich nur Node 3 beendet, sodass die Nodes 4 bis 6 weiterhin aktiv sind.
Node 2 wird in allen drei \glspl{Test} bereits im dritten \gls{Testfall} beendet, da die Auslastung des Nodes im zweiten \gls{Testfall} bei jeweils über 90 Prozent liegt.
Die übrigen der 19 bzw. 20 aktivierten und zwischen 10 und 13 wieder deaktivierten Komponentenfehler unterschieden sich in den drei \glspl{Test} bis auf einzelne, hier nicht relevante, Ausnahmen nicht.

Der Grund für die Injizierung von Komponentenfehlern bei noch allen aktiven Nodes im achten \gls{Testfall} liegt in der Auslastung der Nodes im siebten Testfall.
Diese beträgt im \gls{Test} 27 ohne Mutationen bei den beiden betroffenen Nodes jeweils 100 Prozent, bei den übrigen Nodes ist jedoch keine bzw. eine geringe Auslastung vorhanden.
In den beiden \glspl{Test} der Konfiguration 28 ist das Cluster jeweils vollständig ausgelastet, wodurch die Wahrscheinlichkeit zur Aktivierung der Komponentenfehler im folgenden \gls{Testfall} stark ansteigt (vgl. \cref{subsec:faultActivation}).
Dadurch war es möglich, dass alle noch aktiven Nodes vom Cluster getrennt bzw. beendet wurden und der \gls{Test} aufgrund fehlender Rekonfigurationsmöglichkeiten abgebrochen wurde (vgl. \cref{subsec:oracleImpl}).

\subsection{Testkonfigurationen 31 und 32}
\label{subsec:noReconf3132}

Die \glspl{Test} der Konfigurationen 31 und 32 verliefen ähnlich zueinander.
Die hohe Anzahl der 19 bzw. 20 aktivierten Komponentenfehler reichten bei jeweils 11 wieder deaktivierten Fehlern aus, um die \glspl{Test} 31.2 und 32 im achten \gls{Testfall} abzubrechen.
Der \gls{Test} 31.1 verlief zwar ebenfalls ähnlich zu den beiden anderen Tests, wurde jedoch aufgrund fehlender, verfügbaren Submitter des Connectors beendet.
Die Gründe dafür sind in \cref{subsec:notStartedApps} erläutert, weshalb der \gls{Test} 31.1 hier nicht genauer betrachtet wird.

Bei den beiden \glspl{Test} 31.2 und 32 fällt auf, dass bei jeweils mehreren Testfällen mehr als 3 Komponentenfehler aktiviert bzw. deaktiviert wurden.
So kam es vor, dass \zB in dritten ausgeführten \gls{Testfall} bereits eine Rekonfiguration nur deshalb möglich war, weil der zuvor vom Cluster getrennte Node 1 wieder mit dem Cluster verbunden wurde, während die Nodes 2 und 4 bis 6 getrennt oder beendet wurden, während Node 3 bereits im \gls{Testfall} zuvor beendet wurde.
Ebenso verlief der dritte \gls{Testfall} auch in den beiden \glspl{Test} der Konfigurationen 29 und 30, bei denen nur fünf \glspl{Testfall} ausgeführt wurden.

Bis auf den beendeten Node 2 wurden spätestens im sechsten ausgeführten \gls{Testfall} die im dritten \gls{Testfall} injizierten Komponentenfehler wieder repariert.
Zwar wurde im siebten \gls{Testfall} je ein Komponentenfehler repariert, jedoch im \gls{Test} ohne Mutationen auch ein weiterer injiziert.
In Kombination mit den drei bzw. vier aktivierten Komponentenfehlern im achten \gls{Testfall} führte das daher dazu, dass kein aktiver Node im Cluster mehr vorhanden war und der \gls{Test} entsprechend abgebrochen wurde.
