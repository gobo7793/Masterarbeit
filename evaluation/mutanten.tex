\section{Erkennung der Mutanten}
\label{sec:killingMutants}

Da die Selfbalancing"=Komponente den \ac{MARP}"=Wert basierend auf der aktuellen Auslastung des Clusters anpasst, konnte anhand der Betrachtung der \ac{MARP}"=Werte auch geprüft werden, ob die implementierten Mutationen vom Testsystem erkannt wurden.
Um das zu bewerkstelligen wurden von den 16 Testkonfigurationen mit einem Mutationsszenario 22 Testausführungen durchgeführt (vgl. \autoref{app:overviewExecutedTestCases}).

Zunächst wurde das Mutationsszenario genutzt, in dem alle vier Mutationen enthalten sind (vgl. \autoref{sec:implMutationTests}).
Bei jeder der 17 Testausführungen mit allen Mutanten wurden diese basierend auf dem Constraint zur Erkennung des \ac{MARP}"=Wertes erkannt.
Eine Besonderheit bildet hier jedoch der Test 2, der den korrespondierenden Mutationstest zur Konfiguration 1 darstellt, bei der bei beiden Ausführungen der \ac{MARP}"=Wert nicht verändert wurde.
Bei Test 2 kann daher nicht eindeutig festgestellt werden, ob der Mutant erkannt wurde, oder ob aufgrund der gestarteten Anwendungen der \ac{MARP}"=Wert nicht verändert wurde (vgl. Vermutungen zu Testkonfiguration 1 in \autoref{tab:finalMarpValues}).

Anders ist dies im Vergleich der Ausführungen der Testkonfigurationen 7 und 8.
Durch die massive Veränderung des \ac{MARP}"=Wertes im Test 7.2 auf den finalen Wert von 1 kann davon ausgegangen werden, dass die Mutanten der Konfiguration 8 erkannt wurden.
Dies wird dadurch gestützt, dass bei Konfiguration 8 im Gegensatz zur korrespondieren Testkonfiguration 3 der 4 gestarteten Anwendungen gefailt sind.
Zudem stellt das ein Indiz dafür dar, dass der Mutant in Konfiguration 2 erkannt worden sein könnte.

Während bei jeder Konfiguration ein Mutationsszenario mit jeweils allen vier Mutanten genutzt wurde, wurde die Testkonfiguration 10 zusätzlich mit jeweils einem Mutanten ausgeführt.
Ziel hierbei war es zu validieren, ob einzelne Mutanten ebenfalls vom Testsystem erkannt werden oder zur Erkennung der Mutanten vom Testsystem eine Kombination aus mehreren Mutaten nötig ist.
Hierzu wurde die Testkonfiguration 10 mit unterschiedlichen Mutationsszenarien der Plattform Hadoop"=Benchmark ausgeführt, bei denen jeweils ein Mutant aktiv ist.
Die Auswahl dieser Testkonfiguration hierfür liegt darin begründet, dass hier das Cluster auf beiden Hosts mit sechs Nodes gestartet wird, auf denen bis zu vier Anwendungen gleichzeitig gestartet werden.
Zudem wurde beim Test 9.2 festgestellt, dass sich der \ac{MARP}"=Wert auch direkt im ersten ausgeführten Testfall ändern kann und es gab bei den Ausführungen der Konfiguration 9 keine weiteren ungültigen Constraints als fehlerhafte Anwendungen.

Einige Ergebnisse der hierfür 5 ausgeführten Tests sind sehr unterschiedlich.
So variiert die Anzahl der aktivierten und deaktivierten Komponentenfehler zwischen 7 und 11 bzw. 5 und 9, sowie die Anzahl der gefailten Anwendungen zwischen 1 und 3.
Gemein haben alle Tests jedoch, dass der \ac{MARP}"=Wert bei allen 5 Tests nicht verändert wurde, womit alle Mutationen erkannt wurden.
Damit kann festgestellt werden, dass jeder der vier in \autoref{sec:implMutationTests} beschriebenen Mutanten durch das entwickelte Testsystem erkannt wird.
