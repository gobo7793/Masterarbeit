\section{Erkenntnisse der Evaluation}
\label{sec:evaluationResults}

\todo{Zusammenfassung}

\subsection{Betrachtung der \ac{MARP}"=Werte}
\label{sec:marpValueResults}

Bei der Betrachtung der \ac{MARP}"=Werte lässt sich generell sagen, dass die Selfbalancing"=Komponente den \ac{MARP}"=Wert entsprechend der Auslastung des Clusters anpasst.
Während bei allen Testkonfigurationen, bei denen alle 4 Mutationen aktiv waren, der \ac{MARP}"=Wert unverändert blieb, wurde er bei 16 von 20 Ausführungen der 16 Konfigurationen ohne Mutationen verändert:

\begin{table}[h]
    \begin{tabular}{l|c|c|c|c|c|c|c|c|c|c}
    	Konf. &  1.1  &  1.2  &   3   &  5.1  &  5.2  &  7.1  &  7.2  &  9.1  &  9.2  &  11   \\ \hline
    	Wert  & 0,100 & 0,100 & 0,474 & 0,242 & 0,100 & 0,100 & 1,000 & 0,269 & 0,175 & 0,539 \\
    	\multicolumn{11}{c}{} \\
    	Konf. &  13   &  15   &  17   &  19   &  21   &  23   &  25   &  27   &  29   &  31   \\ \hline
    	Wert  & 0,356 & 0,368 & 0,731 & 0,430 & 0,335 & 0,498 & 0,521 & ,0819 & 0,273 & 0,333
    \end{tabular}
    \caption{Finale \ac{MARP}"=Werte der Testkonfigurationen ohne Mutanten}
    \label{tab:finalMarpValues}
\end{table}

Da er in den Konfigurationen 1 und 7 bei der jeweils ersten Ausführung nicht verändert wurde, wurden beide Konfigurationen erneut ausgeführt, wobei der \ac{MARP}"=Wert bei letzterem mehrmals erhöht wurde, bevor er im finalen Clusterstatus auf 1 gesetzt wurde.
Bei der Konfiguration 1 wurde der Wert dagegen bei keiner der beiden Ausführungen verändert.

Dies liegt sehr wahrscheinlich daran, dass in Konfiguration 1 maximal zwei Anwendungen gleichzeitig gestartet werden.
In den beiden konkreten Ausführungen wurden die jeweiligen Anwendungen jedoch vergleichsweise schnell beendet.
Zudem wurden in den zusammen 10 Testfällen nur 8 Anwendungen gestartet, weshalb die Anwendungen auch ohne eine Anpassung des \ac{MARP}"=Wertes genügend Ressourcen zur Verfügung hatten.
Bei der Testkonfiguration 7 ist dies ähnlich, wobei die gesamte Last auf mehrere Nodes verteilt wird.
Der Test 7.2 zeigt jedoch auch, dass es stark abhängig davon ist, wie die Last im Cluster verteilt wird.

\subsection{Erkennung der Mutanten}
\label{sec:killingMutants}

% Generelles, Mutanten wurden fast immer erkannt

% Warum wurden mutanten in #10 wohl erkannt?

\subsection{Aktivierung und Deaktivierung der Komponentenfehler}
\label{sec:faultInjectionEval}

% Verhalten der Faults?

\subsection{Nicht erfolgreich abgeschlossene Anwendungen}
\label{sec:failedAppsEval}

% Was ist bei der Anzahl der gefailten Apps aufgefallen?

% Warum sind Apps gefailt?

% Kuriose Steps: AppMstr auf zb Node1 -> Stop -> AppMstr zb Node 5 -> Stop -> App Failed

% DFSIO Read Mapfail

\subsection{Rekonfiguration nicht möglich}
\label{sec:noReconfig}

% Rekonffehler #3-6

% Rekonffehler #15/16

% Rekonffehler #19-22

% Rekonffehler #27/28

% Rekonffehler #31/32

\subsection{Nicht erkannte oder gespeicherte \ac{AppMstr}"=Nodes}
\label{sec:noDetectedHost2}

% NotSaved #7/8

% NoAllok #8

% NoSaved #23-26

\subsection{Nicht erkannte, injizierte bzw. reparierte Komponentenfehler}
\label{sec:noDetectedFault}

% SuT/Test-Constraint StartNode 4 bei #17-28

\subsection{Nicht gestartete Anwendungen}
\label{sec:notStartedApps}

% Multiapp-Constraint bei #29-32

% Zu wenig Submitter bei #31/32
