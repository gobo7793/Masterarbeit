\section{Übersicht der ausgeführten Tests}
\label{sec:overviewExecTestCases}

In \autoref{sec:selectTestcases} wurden 32 Testkonfigurationen ermittelt, die mithilfe des in dieser Fallstudie entwickelten Testansatzes insgesamt 42 mal ausgeführt wurden.
Die meisten Konfigurationen wurden hierbei jeweils einmal ausgeführt, 6 Konfigurationen wurden auch mehrmals ausgeführt.
Die Gründe für eine mehrfache Ausführung einzelner Konfigurationen sind in im Rahmen der entsprechenden Auffälligkeiten bzw. Fehler beschrieben.
Eine Übersicht aller genutzten Testkonfigurationen und deren Ausführungen findet sich in \autoref{app:overviewExecutedTestCases}.

Bei der Auswertung der Programmlogs der einzelnen Tests musste zudem beachtet werden, dass die jeweiligen Monitoring"=Informationen nur Momentaufnahmen bilden.
Vor allem bei längeren Schritten werden vom \ac{RM} sehr viele Anpassungen vorgenommen, die aufgrund der Struktur eines Simulations"=Schrittes (vgl. \autoref{sec:simulationStep}) nicht erkannt werden können.

Zur Auswertung der Evaluation dienten die in \todo{Constraintabschnitt} implementierten und nicht implementierten Constraints der in \autoref{sec:clusterRequirements} und \autoref{sec:predictions} definierten Anforderungen.

In diesem Kapitel wurden folgende Begriffe mit folgenden Bedeutungen genutzt:

\begin{description}
    \item [Bessere Lastverteilung] \hfill \\
        Die gesamte Last im Cluster ist für den jeweiligen Kontext besser verteilt.
        Meist bedeutet dies, dass die Last auf allen Nodes gleichmäßig verteilt ist anstatt einer vollständigen Auslastung einzelner Nodes.
    \item [Gefailte Anwendung] \hfill \\
        Nicht vollständig abgeschlossene Anwendung, die aufgrund eines Fehler vorzeitig beendet wurde.
    \item [Testkonfiguration] \hfill \\
        Eine Konfiguration bestehend aus mehreren Parametern, die einen Test definieren.
        Die Nummerierung der in \autoref{sec:selectTestcases} definierten Konfigurationen erfolgte fortlaufend.
    \item [Testfall] \hfill \\
        Ein ausgeführter Simulations"=Schritt.
        Ein Testfall wird während der Laufzeit basierend auf einer zugrundeliegenden Testkonfiguration sowie den Ereignissen und Ergebnissen zuvor ausgeführter Testfälle der zugrundeliegenden Konfiguration generiert.
        In einem Testfall können daher unterschiedliche Komponentenfehler aktiviert und deaktiviert sowie unterschiedliche Anwendungen gestartet werden, auch wenn sie durch die gleiche Testkonfiguration generiert wurden.
    \item [Test] \hfill \\
        Eine Ausführung einer Testkonfiguration.
        Um mehrmalige Ausführungen einer Testkonfiguration zu kennzeichnen, wurde der jeweiligen Konfiguration eine weitere Ziffer angehängt.
\end{description}
